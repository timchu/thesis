\section{Exactly Computing the nearest neighbor metric}
\label{sec:NN}
In this section, we prove Theorem~\ref{thm:NN} on finite point sets, and
explain in Section~\ref{sec:bodies} that our proof strategy applies to
finite collections of path-connected compact bodies.

  First, lets observe what happens when $P$ has only two points $a$ and
$b$, $\dist_2(a,b) = \dist_N(a,b)$.  This reduces to a high school calculus
exercise as the minimum path $\gamma$ will be a straight line between the
points and the nearest neighbor metric is
\begin{align*}
  \dist_N(a,b) &= 4\int_0^1 \distto_P(\gamma(t))\|\gamma'(t)\|dt \\
  & = 8\int_0^{\frac{1}{2}} t \|a - b\|^2 dt = \|a - b\|^2 = \dist_2(a,b).
\end{align*}
  Now it is easy to observe that the
  nearest neighbor metric is never greater than the
  edge-squared distance, as proven in the following lemma.

  \begin{lemma}\label{lem:dist_N_le_dist}
    For all $s,p\in P$, we have $\dist_N(s,p)\le \dist_2(s,p)$.
  \end{lemma}
  \begin{proof}
    Fix any points $s,p\in P$.
    Let $q_0,\ldots, q_k \in P$ be such that $q_0 = s$, $q_k = p$ and
    \[
      \dist_2(s,p) = \sum_{i=1}^k \|q_i - q_{i-1}\|^2.
    \]
    Let $\psi_i(t) = tq_i + (1-t)q_{i-1}$ be the straight line segment from $q_{i-1}$ to $q_i$.
    Observe that $\len(\psi_i) = \|q_i - q_{i-1}\|^2 / 4$, by the same argument as in the two point case.
    Then, let $\psi$ be the concatenation of the $\psi_i$ and it follows that
    \[
      \dist_2(s,p) = 4 \len(\psi) \ge 4 \inf_{\gamma\in \ourpath(s,p)} \len(\gamma) = \dist_N(s,p).
    \]
  \end{proof}

  By Lemma~\ref{lem:dist_N_le_dist}, it suffices to show that $\dist_N(a, b) \geq \dist_2(a,b)$ for all $a, b \in P$.
% This allows us to compute of the nearest neighbor metric exactly by instead computing the edge-squared metric.
%\subsection{Equivalence} % (fold)
\label{sec:the_proof}

  Let $P\subset \R^d$ be a set of $n$ points.
  Pick any \emph{source} point $s\in P$.
  Order the points of $P$ as $p_1,\ldots ,p_n$ so that
  \[
    \dist_2(s,p_1) \le \cdots \le \dist_2(s, p_n).
  \]
  This will imply that $p_1 = s$.
  It will suffice to show that for all $p_i\in P$, we have $\dist_2(s,p_i) = \dist_N(s,p_i)$.
  There are three main steps:
  \begin{enumerate}
    \item We first show that when $P$ is a subset of the vertices of an axis-aligned box, $\dist = \dist_N$.  In this case, shortest paths for $\dist$ are single edges and shortest paths for $\dist_N$ are straight lines.
    \item We then show how to lift the points from $\R^d$ to $\R^n$ by a Lipschitz map $m$ that places all the points on the vertices of a box and preserves $\dist_2(s,p)$ for all $p\in P$.
    \item Finally, we show how the Lipschitz extension of $m$ is also Lipschitz as a function between nearest neighbor metrics.  We combine these pieces to show that $\dist \le \dist_N$.  As $\dist \ge \dist_N$ (Lemma~\ref{lem:dist_N_le_dist}), this will conclude the proof that $\dist = \dist_N$.
  \end{enumerate}
The key to the second step, to be elaborated in Section~\ref{sec:lifting},
is that if you take points on a line and raise the pairwise distances to
the $1/2$ power, you get points on a box. This is a special case of the
general theory on screw functions developed by Von Neumann and Schoenberg,
which asserts a far more general criterion on when functions applied to
pairwise distances between points on a line can be embedded into Euclidean
space~\cite{VonNeumann41}.
  % !TeX root = main.tex

\subsubsection{Boxes} % (fold)
\label{sec:boxes}

  Let $Q$ be the vertices of a box in $\R^n$.
  That is, there exist some positive real numbers $\alpha_1,\ldots , \alpha_n$ such that each $q\in Q$ can be written as $q = \sum_{i\in I} \alpha_i e_i$, for some $I\subseteq [n]$.

  Let the source $s$ be the origin.
  Let $\distto_Q:\R^n\to \R$ be the distance function to the set $Q$.
  Setting $r_i(x) := \min\{x_i, \alpha_i - x_i\}$ (a lower bound on the difference in the $i$th coordinate to a vertex of the box), it follows that
  \begin{equation}
    \label{eq:distto_bded_by_ris}
    \distto_Q(x) \ge \sqrt{\sum_{i= 1}^n r_i(x)^2}.
  \end{equation}

  Let $\gamma:[0,1]\to \R^n$ be a curve in $\R^n$.
  Define $\gamma_i(t)$ to be the projection of $\gamma$ onto its $i$th coordinate.
  Thus,
  \begin{equation}\label{eq:ri_as_min}
    r_i(\gamma(t)) = \min\{\gamma_i(t), \alpha_i - \gamma_i(t)\}
  \end{equation}
  and
  \begin{equation}\label{eq:gamma_decomposed}
    \|\gamma'(t)\| = \sqrt{\sum_{i = 1}^n \gamma_i'(t)^2}.
  \end{equation}
  %
  We can bound the length of $\gamma$ as follows. For simplicity of exposition we only present the case
  of a path from the origin to the far corner, $p = \sum_{i=1}^n \alpha_i e_i$. 
  %
  \begin{align*}
    \len(\gamma)
      &= \int_0^1 \distto_Q(\gamma(t))\|\gamma'(t)\|dt \\
      & \text{[by definition]}\\
      &\ge \int_0^1 \left(\sqrt{\sum_{i= 1}^n r_i(\gamma(t))^2} \sqrt{\sum_{i = 1}^n \gamma_i'(t)^2}\right) dt \\
      & \text{[by \eqref{eq:distto_bded_by_ris} and \eqref{eq:gamma_decomposed}]}\\
      &\ge \sum_{i=1}^n \int_0^1 r_i(\gamma(t)) \gamma_i'(t) dt \\
      & \text{[by Cauchy-Schwarz]}\\
      &\ge \sum_{i=1}^n \left(\int_0^{\ell_i} \gamma_i(t) \gamma_i'(t)dt + \int_{\ell_i'}^1 (\alpha_i - \gamma_i(t)) \gamma_i'(t) dt\right) \\
      & \text{[by \eqref{eq:ri_as_min} where $\gamma_i(\ell_i) = \alpha_i/2$ for the first time} \\
      & \text{and $\gamma_i(\ell_i') = \alpha_i/2$ for the last time.]}\\
      &= \sum_{i=1}^n 2\int_0^{\ell_i} \gamma_i(t) \gamma_i'(t) dt \\
      & \text{[by symmetry]}\\
      &\ge \sum_{i=1}^n \frac{\alpha_i^2}{4} \\
      & \text{[by basic calculus]}
  \end{align*}

  It follows that if $\gamma$ is any curve that starts at $s$ and ends at $p = \sum_{i=1}^n \alpha_i e_i$, then $\dist_N(s,p) = \dist_2(s,p)$.

% section boxes (end)

  \subsubsection{Lifting the points to $\R^n$} % (fold)
\label{sec:lifting}

  Define a mapping $m: P \to \R^n$.  We do this by adding the points $p_1, \ldots, p_n$, as defined above, one point at a time.
  For each new point we will introduce a new dimension. We start by setting $m(p_1) = 0$ and by induction:
  \begin{equation}\label{eq:defn_of_m}
    m(p_i) = m(p_{i-1}) + \sqrt{\dist_2(s, p_i) - \dist_2(s, p_{i-1})} e_i,
  \end{equation}
  where the vectors $e_i$ are the standard basis vectors in $\R^n$.
  A similar embedding works for some other functions and was extensively studied by Schoenberg and Von Neumann in the theory of screw functions.

  \begin{lemma}\label{lem:m_and_dist}
    For all $p_i, p_j\in P$, we have
    \begin{enumerate}
      \item[(i)] $\|m(p_j) - m(p_i)\| = \sqrt{|\dist_2(s,p_j) - \dist_2(s,p_i)|}$, and
      \item[(ii)]$\|m(s) - m(p_j)\|^2 \le \|m(p_i)\|^2 + \|m(p_i) - m(p_j)\|^2$.
    \end{enumerate}
  \end{lemma}
  \begin{proof}
    \emph{Proof of (i).}
    Without loss of generality, let $i \le j$.
    Then, by the definition of $m$, expanding the norm, and telescoping the sum, we get the following.
    \begin{align*}
      & \|m(p_j) - m(p_i)\| \\
      &= \left\|\sum_{k=i+1}^j \sqrt{\dist_2(s, p_k) - \dist_2(s, p_{k-1})} e_k \right\| \\
      &= \sqrt{\sum_{k=i+1}^j (\dist_2(s, p_k) - \dist_2(s, p_{k-1}))}\\
      &= \sqrt{\dist_2(s, p_j) - \dist_2(s, p_i)}.
    \end{align*}

    \noindent\emph{Proof of (ii).}
    As $m(s) = 0$, it suffice to observe that
    \begin{align*}
      \|m(p_j)\|^2
        &= \dist_2(s, p_j) \because{by \emph{(i)}}\\
        % & \text{[by \emph{(i)}]}\\
        &\le \dist_2(s, p_i) + |\dist_2(s,p_j) - \dist_2(s,p_i)| \\
        % & \text{[by basic arithmetic]}\\
        &= \|m_(p_i)\|^2 + \|m(p_i) - m(p_j)\|^2  \because{by \emph{(i)}}
        % & \text{[by \emph{(i)}]}
    \end{align*}
  \end{proof}

  We can now show that $m$ has all of the desired properties.

  \begin{prop}\label{prop:m_is_good}
    Let $P\subset\R^d$ be a set of $n$ points, let $s\in P$ be a designated source point, and let $m:P\to \R^n$ be the map defined as in \eqref{eq:defn_of_m}.
    Let $\dist'$ denote the edge squared metric for the point set $m(P)$ in $\R^n$.
    Then,
    \begin{enumerate}
      \item[(i)] $m$ is $1$-Lipschitz as a map between Euclidean metrics,
      \item[(ii)] $m$ maps the points of $P$ to the vertices of a box, and
      \item[(iii)] $m$ preserves the edge squared distance to $s$, i.e.\ $\dist'(m(s), m(p)) = \dist_2(s,p)$ for all $p\in P$.
    \end{enumerate}
  \end{prop}
  \begin{proof}
    \emph{Proof of (i).} To prove the Lipschitz condition, fix any $a,b\in P$ and bound the distance as follows.
    \begin{align*}
      \|m(a) - m(b)\|
        &= \sqrt{|\dist_2(s,a) - \dist_2(s,b)|} \because{Lem.~\ref{lem:m_and_dist}(i)}\\
        % & \text{[by Lemma~\ref{lem:m_and_dist}(i)]}\\
        &\le \sqrt{\dist_2(a,b)} \because{triangle ineq.}\\
        % & \text{[by triangle inequality]}\\
        &\le \|a-b\|. \because{by def. of $\dist_2$}\\
        % & \text{[$\dist_2(a,b)\le \|a-b\|^2$ by the definition of $\dist$]}
    \end{align*}

    \noindent
    \emph{Proof of (ii).} That $m$ maps $P$ to the vertices of a box is immediate from the definition.
    The box has side lengths $\|m_i - m_{i-1}\|$ for all $i>1$ and $p_i = \sum_{k=1}^i \|m_k - m_{k-1}\| e_k$.

    \noindent
    \emph{Proof of (iii).} We can now show that the edge squared distance to $s$ is preserved.
    Let $q_0,\ldots, q_k$ be the shortest sequence of points of $m(P)$ that realizes the edge-squared distance from $m(s)$ to $m(p)$, i.e., $q_0 = m(s)$, $q_k = m(p)$, and
    \[
      \dist'(m(s), m(p)) = \sum_{i = 1}^k \|m(q_i) - m(q_{i-1})\|^2.
    \]
    If $k> 1$, then Lemma~\ref{lem:m_and_dist}(ii) implies that removing $q_1$ gives a shorter sequence.
    Thus, we may assume $k = 1$ and therefore, by Lemma~\ref{lem:m_and_dist}(i),
    \[
      \dist'(m(s), m(p)) = \|m(s) - m(p)\|^2 = \dist_2(s,p). %\qedhere
    \]
  \end{proof}


% section lifting (end)

  \subsubsection{The Lipschitz Extension} % (fold)
\label{sec:lip_extension}

  Proposition~\ref{prop:m_is_good} and the Kirszbraun theorem on Lipschitz extensions imply that we can extend $m$ to a $1$-Lipschitz function $f: \R^d\to \R^n$ such that $f(p) = m(p)$ for all $p\in P$ \cite{Kirszbraun1934,Valentine1945,brehm1981}.

  \begin{lemma}\label{lem:dist_N_lipschitz}
    The function $f$ is also $1$-Lipschitz as mapping from $\R^d\to \R^n$ with both spaces endowed with the nearest neighbor metric.
  \end{lemma}
  \begin{proof}
    We are interested in two distance functions $\distto_P:\R^d \to \R$ and $\distto_{f(P)}: \R^n\to \R$.
    Recall that each is the distance to the nearest point in $P$ or $f(P)$ respectively.
    \begin{align*}
      \distto_{f(P)}(f(x))
        &= \min_{q\in f(P)} \|q - f(x)\| \because{by definition}\\
        % & \text{[by definition]}\\
        &= \min_{p\in P} \|f(p) - f(x)\| \because{$q\in f(P)$}\\
        % & \text{[$q = f(p)$ for some $p$]}\\
        &\le \min_{p\in P} \|p - x\|\because{$f$ is $1$-Lipschitz}\\
        % & \text{[$f$ is $1$-Lipschitz]}\\
        &= \distto_P(x). \because{by definition}\\
        % & \text{[by definition]}
    \end{align*}
    For any curve $\gamma:[0,1]\to \R^d$ and for all $t\in [0,1]$, we have $\|(f\circ \gamma)'(t)\| \le \|\gamma'(t)\|$.
    It then follows that

    \begin{align}
      \len'(f\circ \gamma) &= \int_0^1 \distto_{f(P)}(f(\gamma(t)))\|(f\circ\gamma)'(t)\|dt \nonumber\\
              & \le \int_0^1 \distto_{P}(\gamma(t))\|\gamma'(t)\|dt = \len(\gamma),\label{eq:curves_shorten}
      \end{align}
    where $\len'$ denotes the length with respect to $\distto_{f(P)}$.
    Thus, for all $a,b\in P$,
    \begin{align*}
      \dist_N(a,b)
        &= 4 \inf_{\gamma\in \ourpath(a,b)} \len(\gamma) \because{by definition}\\
        &\ge 4 \inf_{\gamma\in \ourpath(a,b)} \len'(f\circ\gamma) \because{by \eqref{eq:curves_shorten}}\\
        % & \text{[by \eqref{eq:curves_shorten}]}\\
        &\ge 4 \inf_{\gamma'\in \ourpath(f(a),f(b))} \len'(\gamma') \because{$f\circ\gamma$ is a path}\\
        % & \text{[because $f\circ\gamma\in \ourpath(f(a), f(b))$]}\\
        &= \dist_N(f(a), f(b)). \because{by definition}\\
        % & \text{[by definition]}
    \end{align*}
  \end{proof}

  We now restate Theorem~\ref{thm:NN} for convenience, and prove
  it.
  \begin{theorem}\label{thm:equality}
    For any point set $P\subset\R^d$, the edge squared metric $\dist$ and the nearest neighbor metric $\dist_N$ are identical.
  \end{theorem}
  \begin{proof}
    Fix any pair of points $s$ and $p$ in $P$.
    Define the Lipschitz mapping $m$ and its extension $f$ as in \eqref{eq:defn_of_m}.
    Let $\dist'$ and $\dist_N'$ denote the edge-squared and nearest neighbor metrics on $f(P)$ in $\R^n$.
    \begin{align*}
      \dist_2(s,p)
        &= \dist'(m(s), m(p)) \because{Proposition~\ref{prop:m_is_good}(iii)}\\
        &= \dist_N'(m(s), m(p)) \because{$f(P)$ are vertices of a box}\\
        &\le \dist_N(s, p) \because{Lemma~\ref{lem:dist_N_lipschitz}}
    \end{align*}
    We have just shown that $\dist\le \dist_N$ and Lemma~\ref{lem:dist_N_le_dist} states that $\dist\ge \dist_N$, so we conclude that $\dist = \dist_N$ as desired.
  \end{proof}

% section lip_extension (end)

\subsection{From Finite Sets to Finite Collections of Compact Path-Connected Bodies}
\label{sec:bodies}
All of our proof steps hold for finite collections of compact,
path-connected bodies in arbitrarily large dimension. Our Lipschitz map $m$ can
still be extended to a Lipschitz map $f$ in this setting, largely due to the
generality of the Kirszbraun theorem. In this case, the pre-image of the
contractive map is the set of all points belonging to some body. Meanwhile, the image is a finite set of points, the corners of a multi-dimensional box.
Thus our construction of $m$ contracts each convex
body into a single point, and the image of our
compact bodies under $f$ is still a finite point set on the corners of a
box. Therefore, the remainder of our theorem proof goes through unchanged.

This result is rather remarkable: path-connected compact sets in high
dimensional space can have extremely convoluted geometry, and the Voronoi diagrams
on these collections (on which the nearest neighbor metric depends) can be
massively complex.  The key is that our Lipschitz map is robust enough to
handle objects of considerable geometric complexity.
  
  
% section the_proof (end)

