\section{Introduction}

A foundational hypothesis in non-linear dimension reduction and machine
learning is that data can be represented as points in Euclidean space,
and that graphs on these points can be generated to solve a variety of
problems on the data, including classification, regression, and clustering \cite{Daitch09, Chen11,vL09, Dong11, Hashi15,
vonluxburg13density, mcqueen16megaman}.
Such graphs can induce similarity measures or distances on the point set,
which are used as the key model when generating a clustering or
classification~\cite{Daitch09, bijral11semiSupLearningDBD, tenenbaum00global}

The problem of generating meaningful graphs and metrics on data is an active research topic in machine learning, statistics, and
geometry~\cite{hein07graph, amenta99surface, sajama05estimatingDBDM, vincent03,
bijral11semiSupLearningDBD, Hashi15, alamgir12shortest, ting10analysis,
gorban07principal, bookDey2007, cavanna16adaptive, Gottlieb14, vL04,
agarwal16efficient, SridharMaster, hwang2016}.
% We believe two important problems are: comparing the
% various methods of generating distances from point sets, and
% building data structures to quickly compute these distances.
%
In this paper, we study a distance generated from a graph
on points in Euclidean space called the edge-squared metric. It
is defined by taking the squared Euclidean distance between every pair of
points, and finding the shortest path on the resulting graph.
This metric has the property that two points
in a dense cluster can be close, even if their Euclidean
distance is far~\cite{bijral11semiSupLearningDBD}.
This property is known as \textbf{density-sensitivity}
~\cite{vincent03}, and is
desirable for clustering and classification
~\cite{sajama05estimatingDBDM, alamgir12shortest, hwang2016,
cohen15approximating}. The edge-squared metric has been studied
before in the
context of machine learning~\cite{bijral11semiSupLearningDBD, vincent03, hwang2016, alamgir12shortest,
cohen15approximating} and
power-efficient wireless
networks~\cite{LiWan2001, LiWan2002}.
Squared Euclidean distances have been
examined before as an optimization objective, and occur in
natural settings including $k$-means clustering and RMS
matching~\cite{macqueen67,Agarwal12}.
We compare the edge-squared metric to another
distance on point sets known as the nearest neighbor metric,
first introduced in~\cite{cohen15approximating} as the nearest neighbor
distance. Close variants of this metric have been studied in
probability and machine learning~\cite{sajama05estimatingDBDM, cohen15approximating,
hwang2016}. We then give efficient algorithms constructing sparse
spanners for both metrics.

% Close variants of both these metrics have been studied in machine
% learning~\cite{bijral11semiSupLearningDBD, LiWan2001,
% alamgir12shortest,sajama05estimatingDBDM, cohen15approximating,
% vincent03, hwang2016}.

In this paper, we show that the edge-squared metric and nearest-neighbor
geodesic distance are
identical. It was not previously known or
suspected that the two might be the same. This is the
first work we know of that equates a discrete metric with a
continuous geodesic, and gives the first nontrivial example of a
so-called \textbf{density-based distance}~\cite{sajama05estimatingDBDM} that can
be computed exactly. This considerably improves a result
in~\cite{cohen15approximating}, which showed that the two were
$3$-approximations of each other. Previous works computing
geodesics or density-based distances generally use approximate
methods whose running time grows as the approximation
quality improves~\cite{Kimmel98, tenenbaum00global,
alamgir12shortest, agarwal16efficient, hwang2016,
cohen15approximating}, or
calculus of variations~\cite{bernoulli, Schwarzschild, Sussmann97}.
We don't follow any of the previous existing approaches. Our proof employs
the Kirszbraun theorem, also known as the Lipschitz
Extension Theorem~\cite{Kirszbraun1934, brehm1981}. This theorem has been
widely used in mathematics, classification, and metric
embeddings~\cite{Naor06, Lee05, Naor15, Gottlieb14, Gottlieb17}.
Our result lets us compute the persistent
homology of the nearest neighbor metric, a general problem of interest for
many continuous metrics in the computational geometry
setting~\cite{edelsbrunner02topological,gasparovic17complete,chazal08towards,chazal13persistence,carlsson09topology}.

In order to perform clustering or classification with the edge-squared
metric or nearest neighbor metric, we
provide data structures that admit fast, practical
computation of these metrics with theoretical guarantees. The
edge-squared metric can have high doubling
dimension even if the underlying points are in $2$
dimensions~\cite{cohen15approximating}. This
means that many data structures suitable for low-doubling dimension
metrics will not immediately
work on it. Despite this, if the underlying point set is in low dimension,
we compute a sparse $(1+\eps)$-spanner of
the edge-squared metric quickly.
This spanner is sparser and faster to compute than the
best known $(1+\eps)$-spanners for Euclidean metrics, and uses
techniques from well-separated pair
decompositions~\cite{Callahan1993} and
approximate Euclidean MSTs~\cite{arya95euclid, Arya2016,
Callahan1995}.

 A foundational assumption of
machine learning is that most data points are samples from a
well-behaved probability
distribution with low intrinsic dimension~\cite{tenenbaum00global,
gorban07principal, vandermaaten08visualizing, Hashi15, mcqueen16megaman,
hwang2016}.
We compute a sparse
$1$-spanner for our metrics in such a setting. Our $1$-spanner is a $k$-nearest neighbor graph ($k$-NN graph)
with edge weights equal to Euclidean distance squared, with $k = O(2^d \log n)$. Here, $d$ is the
intrinsic dimension of the probability density.
Note that a sparse $1$-spanner of
Euclidean distance is not possible in this setting.
Our result may allow for fast computation of edge-squared spanners in
practice, given the breadth of
literature on the $k$-NN graph~\cite{Brito97, Dong11, Chen11}
and its widespread
use in practice~\cite{tenenbaum00global, vL09, Dong11}.
If intrinsic dimension $d$ is constant,
our $k$ is nearly optimal:
it is believed $k=\Omega(\log n)$ is necessary for
connectivity of the $k$-nearest neighbor
graph~\cite{Balister05,Gonzales2003, vL09}.

We also show how spanners of the edge-squared metric can be seen as a generalization of
Euclidean spanners~\cite{Vaidya1991, Elkin2013, arya95euclid,
Callahan1993}, approximate Euclidean
MSTs~\cite{Callahan1995, Andoni2014, Arya2016, Alman2016, Yaro2017}, and
single linkage clustering~\cite{Gower1969, Yaro2017}.% !TeX root = main.tex

\subsection{Definitions and Preliminaries} % (fold)
\label{sec:definitions}
In this section, we establish additional definitions for this chapter. These
are mostly of interest for our spanner and persistent homology results, and are not strictly
necessary for Theorem~\ref{thm:NN}.

%  \tim{Anything on wireless networks here? Or other prelims?}
\vspace{3 mm}

\noindent \textbf{Spanners:} For real value $t \geq 1$, a $t$-spanner of
a weighted graph $G$ is a subgraph $S$ such that $d_G(x,y) \leq d_S(x,y)
\leq t\cdot d_G(x,y)$ where $d_G$ and $d_S$ represent the shortest path
distance functions between vertex pairs in $G$ and $S$. Spanners
of Euclidean distances, and general graph distances, have been
studied extensively, and their importance as a data structure is
well established.
~\cite{Chew1986, Vaidya1991, Callahan1993,HarPeled13}.

\vspace{3 mm}
\noindent \textbf{$k$-nearest neighbor graphs:} The $k$-nearest neighbor graph
($k$-NN graph) for a set of objects $V$ is a graph with vertex set $V$
and an edge from $v\in V$ to its $k$ most similar objects in $V$, under
a given distance measure. In this chapter, the underlying distance
measure is Euclidean, and the edge weights are Euclidean distance
squared.
$k$-NN
graph constructions are a key data structure in machine
learning~\cite{Dong11, Chen11}, clustering~\cite{vL09}, and manifold learning~\cite{tenenbaum00global}.

\vspace{3 mm}
\noindent \textbf{Gabriel Graphs:} The Gabriel graph is a graph where
two vertices $p$ and $q$ are joined by an edge if and only if the disk
with diameter $pq$ has no other points of $S$ in the interior. The
Gabriel graph is a subgraph of the Delaunay
triangulation~\cite{SridharMaster}, and a
$1$-spanner of the edge-squared metric~\cite{SridharMaster}. Gabriel
graphs will be used in the proof of
Theorem~\ref{thm:distribution-spanner}.

\vspace{3 mm}
\noindent \textbf{Persistent Homology:}
  Persistent homology is a popular tool in computational geometry and topology to ascribe quantitative topological invariants to spaces that are stable with respect to perturbation of the input.
  In particular, it's possible to compare the so-called persistence diagram of a function defined on a sample to that of the complete space~\cite{chazal08towards}.
  These two aspects of persistence theory---the intrinsic nature of topological invariants and the ability to rigorously compare the discrete and the continuous---are both also present in our theory of nearest neighbor metrics.
  Indeed, our primary motivation for studying these metrics was to use them as inputs to persistence computations for problems such as persistence-based clustering~\cite{chazal13persistence} or metric graph reconstruction~\cite{aanjaneya12metric}.

% section definitions (end)

% To illustrate one finding, we prove in this section that:
% \begin{lemma} \label{lem:one-span} A $1$-spanner of the edge-squared metric
%   contains every Euclidean MST.
% \end{lemma}
% This leads us to an alternate proof of the following corollary, which was already known in the
% literature~\cite{Gonzales2003}:
% \begin{corollary} For a Lipschitz distribution of bounded support bounded
%   above and below, with constant intrinsic
%   dimension $d$, the $k$-NN graph from
%   points drawn from this distribution contains all Euclidean MSTs w.h.p
%   when $k = O(\log n)$. The Euclidean MST is unique almost always.
% \end{corollary}
% \begin{proof} This follows from Theorem~\ref{thm:distribution-spanner}
%   and Lemma~\ref{lem:one-span}.
% \end{proof}
%%%%%% Two techniques!!!

%% In this
%% paper, we provide two techniques for computing accurate and sparse
%% spanners of the edge-squared metric (thus also computing it for the
%% nearest neighbor metric), when the point set lives in constant dimensional space.

%%%%% Two techniques continued!!!!
%% One technique
%% is useful in the setting where the point-set comes from an underlying
%% probability
%% distribution, an assumption common in the machine learning setting. This
%% uses $k$-NN graphs, a common tool in machine learning. The
%% second technique can be used for any point set, and uses well-separated
%% pair decompositions, and other techniques used in computing fast
%% Euclidean spanning trees.

%%%%%%%% Below: Some exposition on the edge^2 metric and NN metric, and what they
%% are.

%% The edge-squared metric is a metric built by taking the graph of
%% Euclidean distances between points, squaring them, and taking the
%% shortest path on the resulting graph. The
%% nearest neighbor metric is a distance based on geodesics through the
%% ambient space, and will be fully
%% defined in Section~\ref{sec:definitions}.
