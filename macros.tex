\newcommand{\e}{\varepsilon}
%\newcommand{\eps}{\varepsilon}
\newcommand{\volt}{\overrightarrow{v}}
%\newcommand{\EE}{\textbf{E}}
\newcommand{\LGinv}{L_G^\dag}
\newcommand{\inv}{(\LGinv)^{1/2}}
\newcommand{\Br}{\overline{B}}
\newcommand{\vr}{\overline{v}}
\newcommand{\xr}{\overline{x}}
%\newcommand{\pr}{\overline{p}}
\newcommand{\Rr}{\overline{R}}
\newcommand{\Gr}{\overline{G}}
\newcommand{\Conv}{\mathcal{C}}
\newcommand{\Convr}{\overline{\mathcal{C}}}
\newcommand{\g}{\textbf{g}}
\newcommand{\B}{\mathbb{B}}
% \DeclarePairedDelimiter\ceil{\lceil}{\rceil}
% \DeclarePairedDelimiter\floor{\lfloor}{\rfloor}

\DeclareMathOperator{\volscrew}{Vol}
\DeclareMathOperator{\kNN}{kNN}
\DeclareMathOperator{\NN}{NN}


\newcommand{\len}{\ell}
\newcommand{\R}{\mathbb{R}}
\newcommand{\RR}{\mathbb{R}}
\newcommand{\ourpath}{\mathrm{path}}
\newcommand{\dist}{\mathbf{d}}
\newcommand{\distto}{\mathbf{r}}
\renewcommand{\because}[1]{&\left[\text{\small{#1}}\right]}

\newcommand\prob[2]{\operatorname*{\mathbb{P}}_{#1}\left[ #2 \right]}
\newcommand{\M}{M}

\newtheorem{claim}{Claim}
\newtheorem{observation}{Observation}
\newtheorem{problem}{Problem}
\newtheorem{theorem}{Theorem}[section]
\newtheorem{prop}[theorem]{Proposition}
\newtheorem{corollary}{Corollary}[theorem]
\newtheorem{remark}{Remark}[theorem]
\newtheorem{lemma}[theorem]{Lemma}
\newtheorem{definition}[theorem]{Definition}
\newtheorem{conjecture}[theorem]{Conjecture}


\newenvironment{fminipage}%
  {\begin{Sbox}\begin{minipage}}%
  {\end{minipage}\end{Sbox}\fbox{\TheSbox}}

\newenvironment{algbox}[0]{\vskip 0.2in
\noindent 
\begin{fminipage}{6.3in}
}{
\end{fminipage}
\vskip 0.2in
}


%%%%%%%%%%%%%% Use for definitions
\newcommand{\defeq}{\stackrel{\textup{def}}{=}}

%%%%%%%%%%%%%% Probability stuff
\DeclareMathOperator*{\pr}{\bf Pr}
\DeclareMathOperator*{\av}{\bf E}
\DeclareMathOperator*{\var}{\bf Var}

%%%%%%%%%%%%%% Matrix stuff
\newcommand{\tr}[1]{\mathop{\mbox{Tr}}\left({#1}\right)}
\newcommand{\diag}[1]{{\bf Diag}\left({#1}\right)}

%% Notation for integers, natural numbers, reals, fractions, sets, cardinalities
%%and so on
\newcommand{\nfrac}[2]{\nicefrac{#1}{#2}}
\def\abs#1{\left| #1 \right|}
\newcommand{\norm}[1]{\ensuremath{\left\lVert #1 \right\rVert}}

\newcommand{\floor}[1]{\left\lfloor\, {#1}\,\right\rfloor}
\newcommand{\ceil}[1]{\left\lceil\, {#1}\,\right\rceil}

\newcommand{\pair}[1]{\left\langle{#1}\right\rangle} %for inner product

%\newcommand\B{\{0,1\}}      % boolean alphabet  use in math mode
\newcommand\bz{\mathbb Z}
\newcommand\nat{\mathbb N}
\newcommand\rea{\mathbb R}
\newcommand\com{\mathbb{C}}
\newcommand\plusminus{\{\pm 1\}}
\newcommand\Bs{\{0,1\}^*}   % B star use in math mode

\newcommand{\V}[1]{\mathbf{#1}} % Used to denote bold commands
                                % e.g. vectors, matrices
\DeclareRobustCommand{\fracp}[2]{{#1 \overwithdelims()#2}}
\DeclareRobustCommand{\fracb}[2]{{#1 \overwithdelims[]#2}}
\newcommand{\marginlabel}[1]%
{\mbox{}\marginpar{\it{\raggedleft\hspace{0pt}#1}}}
\newcommand\card[1]{\left| #1 \right|} %cardinality of set S; usage \card{S}
\newcommand\set[1]{\left\{#1\right\}} %usage \set{1,2,3,,}
\renewcommand\complement{\ensuremath{\mathsf{c}}}
%\newcommand\poly{{\textrm{poly}}}  %usage \poly(n)
\newcommand{\comp}[1]{\overline{#1}}
\newcommand{\smallpair}[1]{\langle{#1}\rangle}
\newcommand{\ol}[1]{\ensuremath{\overline{#1}}\xspace}

%%%%%%%%%%%%%% Mathcal shortcuts
\newcommand\calF{\mathcal{F}}
\newcommand\calS{\mathcal{S}}
\newcommand\calG{\mathcal{G}}
\newcommand\calH{\mathcal{H}}
\newcommand\calC{\mathcal{C}}
\newcommand\calD{\mathcal{D}}
\newcommand\calI{\mathcal{I}}
\newcommand\calV{\mathcal{V}}
\newcommand\calK{\mathcal{K}}
\newcommand\calN{\mathcal{N}}
\newcommand\calX{\mathcal{X}}
\newcommand\calU{\mathcal{U}}
\newcommand\calE{\mathcal{E}}
\newcommand\calA{\mathcal{A}}

%%%%%%%%%%%%%% {{{ authornotes }}}
\definecolor{Mygray}{gray}{0.8}

 \ifcsname ifcommentflag\endcsname\else
  \expandafter\let\csname ifcommentflag\expandafter\endcsname
                  \csname iffalse\endcsname
\fi

% \ifnum\showauthornotes=1
% \newcommand{\todo}[1]{\colorbox{Mygray}{\color{red}\parbox{\textwidth}{#1}}}
% \else
% \newcommand{\todo}[1]{}
% \fi


%%%%%%%%%%%%%% Logical operators
%\newcommand\true{\mbox{\sc True}}
%\newcommand\false{\mbox{\sc False}}
\def\scand{\mbox{\sc and}}
\def\scor{\mbox{\sc or}}
\def\scnot{\mbox{\sc not}}
\def\scyes{\mbox{\sc yes}}
\def\scno{\mbox{\sc no}}

%% Parantheses
\newcommand{\paren}[1]{\left({#1}\right)}
\newcommand{\sqparen}[1]{\left[{#1}\right]}
\newcommand{\curlyparen}[1]{\left\{{#1}\right\}}
\newcommand{\smallparen}[1]{({#1})}
\newcommand{\smallsqparen}[1]{[{#1}]}
\newcommand{\smallcurlyparen}[1]{\{{#1}\}}

%% short-hands for relational simbols

\newcommand{\from}{:}
\newcommand\xor{\oplus}
\newcommand\bigxor{\bigoplus}
\newcommand{\logred}{\leq_{\log}}
\def\iff{\Leftrightarrow}
\def\implies{\Rightarrow}




%% macros to write pseudo-code

\newlength{\pgmtab}  %  \pgmtab is the width of each tab in the
\setlength{\pgmtab}{1em}  %  program environment
 \newenvironment{program}{\renewcommand{\baselinestretch}{1}%
\begin{tabbing}\hspace{0em}\=\hspace{0em}\=%
\hspace{\pgmtab}\=\hspace{\pgmtab}\=\hspace{\pgmtab}\=\hspace{\pgmtab}\=%
\hspace{\pgmtab}\=\hspace{\pgmtab}\=\hspace{\pgmtab}\=\hspace{\pgmtab}\=%
\+\+\kill}{\end{tabbing}\renewcommand{\baselinestretch}{\intl}}
\newcommand {\BEGIN}{{\bf begin\ }}
\newcommand {\ELSE}{{\bf else\ }}
\newcommand {\IF}{{\bf if\ }}
\newcommand {\FOR}{{\bf for\ }}
\newcommand {\TO}{{\bf to\ }}
\newcommand {\DO}{{\bf do\ }}
\newcommand {\WHILE}{{\bf while\ }}
\newcommand {\ACCEPT}{{\bf accept}}
\newcommand {\REJECT}{\mbox{\bf reject}}
\newcommand {\THEN}{\mbox{\bf then\ }}
\newcommand {\END}{{\bf end}}
\newcommand {\RETURN}{\mbox{\bf return\ }}
\newcommand {\HALT}{\mbox{\bf halt}}
\newcommand {\REPEAT}{\mbox{\bf repeat\ }}
\newcommand {\UNTIL}{\mbox{\bf until\ }}
\newcommand {\TRUE}{\mbox{\bf true\ }}
\newcommand {\FALSE}{\mbox{\bf false\ }}
\newcommand {\FORALL}{\mbox{\bf for all\ }}
\newcommand {\DOWNTO}{\mbox{\bf down to\ }}

% Theorem-type environments
% \theoremstyle{break} 
% \theoremheaderfont{\scshape}
% \theorembodyfont{\slshape}
% \newtheorem{Thm}{Theorem}[section]
% \newtheorem{Lem}[Thm]{Lemma}
% \newtheorem{Cor}[Thm]{Corollary}
% \newtheorem{Prop}[Thm]{Proposition}
% % \theoremstyle{plain} 
% % \theorembodyfont{\rmfamily} 
% \newtheorem{Ex}[Thm]{Exercise}
% \newtheorem{Exa}[Thm]{Example}
% \newtheorem{Rem}[Thm]{Remark}
% % \theorembodyfont{\itshape}
% \newtheorem{Def}[Thm]{Definition}
% \newtheorem{Conj}[Thm]{Conjecture}
% \newtheorem{Obs}[Thm]{Observation}
% \newtheorem{Ques}[Thm]{Question}
%\newenvironment{proof}{\noindent {\sc Proof:}}{$\Box$ \medskip} 
\newenvironment{problems} % Definition of problems
 {\renewcommand{\labelenumi}{\S\theenumi}
	\begin{enumerate}}{\end{enumerate}}


%%%%%%%%%%%%%%%%% Proof Environments

\def\FullBox{\hbox{\vrule width 6pt height 6pt depth 0pt}}
%
%\def\qed{\ifmmode\qquad\FullBox\else{\unskip\nobreak\hfil
%\penalty50\hskip1em\null\nobreak\hfil\FullBox
%\parfillskip=0pt\finalhyphendemerits=0\endgraf}\fi}

\def\qedsketch{\ifmmode\Box\else{\unskip\nobreak\hfil
\penalty50\hskip1em\null\nobreak\hfil$\Box$
\parfillskip=0pt\finalhyphendemerits=0\endgraf}\fi}

%\newenvironment{proof}{\begin{trivlist} \item {\bf Proof:~~}}
 %  {\qed\end{trivlist}}

\newenvironment{proofsketch}{\begin{trivlist} \item {\bf
Proof Sketch:~~}}
  {\qedsketch\end{trivlist}}

\newenvironment{proofof}[1]{\begin{trivlist} \item {\bf Proof
#1:~~}}
  {\qed\end{trivlist}}

\newenvironment{claimproof}{\begin{quotation} \noindent
{\bf Proof of claim:~~}}{\qedsketch\end{quotation}}


%%%%%%%%%%%%%%%%%%%%%%%%%%%%%%%%%%%%%%%%%%%%%%%%%%%%%%%%%%%%%%%%%%%%%%%%%%%
%%%%%%%%%%%%%%%%%%%%%%%%%%%%%%%%%%%%%%%%%%%%%%%%%%%%%%%%%%%%%%%%%%%%%%%%%%%




\newlength{\tpush}
\setlength{\tpush}{2\headheight}
\addtolength{\tpush}{\headsep}

\newcommand{\handout}[5]{
   \noindent
   \begin{center}
   \framebox{ \vbox{ \hbox to \textwidth { {\bf \coursenum\ :\  \coursename} \hfill #5 }
       \vspace{3mm}
       \hbox to \textwidth { {\Large \hfill #2  \hfill} }
       \vspace{1mm}
       \hbox to \textwidth { {\it #3 \hfill #4} }
     }
   }
   \end{center}
   \vspace*{4mm}
   \newcommand{\lecturenum}{#1}
   \addcontentsline{toc}{chapter}{Lecture #1 -- #2}
}

\newcommand{\lecturetitle}[4]{\handout{#1}{#2}{Lecturer: \courseprof
  }{Scribe: #3}{Lecture #1 : #4}}
\newcommand{\guestlecturetitle}[5]{\handout{#1}{#2}{Lecturer:
    #4}{Scribe: #3}{Lecture #1 - #5}}


%%%%%%%%%%%%%%%%%%%%%%%%%%%%%%%%%%%%%%%%%%%%%%%%%%%%%%%%%
%%% Commands to include figures


%% PSfigure

\newcommand{\PSfigure}[3]{\begin{figure}[t] 
  \centerline{\vbox to #2 {\vfil \psfig{figure=#1.eps,height=#2} }} 
  \caption{#3}\label{#1} 
  \end{figure}} 
\newcommand{\twoPSfigures}[5]{\begin{figure*}[t]
  \centerline{%
    \hfil
    \begin{tabular}{c}
        \vbox to #3 {\vfil\psfig{figure=#1.eps,height=#3}} \\ (a)
    \end{tabular}
    \hfil\hfil\hfil
    \begin{tabular}{c}
        \vbox to #3 {\vfil\psfig{figure=#2.eps,height=#3}} \\ (b)
    \end{tabular}
    \hfil}
  \caption{#4}
  \label{#5}
%  \sublabel{#1}{(a)}
%  \sublabel{#2}{(b)}
  \end{figure*}}


\newcounter{fignum}

% fig
%command to insert figure. usage \fig{name}{h}{caption}
%where name.eps is the postscript file and h is the height in inches
%The figure is can be referred to using \ref{name}
\newcommand{\fig}[3]{%
\begin{minipage}{\textwidth}
\centering\epsfig{file=#1.eps,height=#2}
\caption{#3} \label{#1}
\end{minipage}
}%


% ffigure
% Usage: \ffigure{name of file}{height}{caption}{label}
\newcommand{\ffigure}[4]{\begin{figure} 
  \centerline{\vbox to #2 {\hfil \psfig{figure=#1.eps,height=#2} }} 
  \caption{#3}\label{#4} 
  \end{figure}} 

% ffigureh
% Usage: \ffigureh{name of file}{height}{caption}{label}
\newcommand{\ffigureh}[4]{\begin{figure}[!h] 
  \centerline{\vbox to #2 {\vfil \psfig{figure=#1.eps,height=#2} }} 
  \caption{#3}\label{#4} 
  \end{figure}} 


% {{{ draftbox }}}

%% Complexity classes
\newcommand\p{\mbox{\bf P}\xspace}
\newcommand\np{\mbox{\bf NP}\xspace}
\newcommand\cnp{\mbox{\bf coNP}\xspace}
\newcommand\sigmatwo{\mbox{\bf $\Sigma_2$}\xspace}
\newcommand\ppoly{\mbox{\bf $\p_{\bf /poly}$}\xspace}
\newcommand\sigmathree{\mbox{\bf $\Sigma_3$}\xspace}
\newcommand\pitwo{\mbox{\bf $\Pi_2$}\xspace}
\newcommand\rp{\mbox{\bf RP}\xspace}
\newcommand\zpp{\mbox{\bf ZPP}\xspace}
\newcommand\bpp{\mbox{\bf BPP}\xspace}
\newcommand\ph{\mbox{\bf PH}\xspace}
\newcommand\pspace{\mbox{\bf PSPACE}\xspace}
\newcommand\npspace{\mbox{\bf NPSPACE}\xspace}
\newcommand\dl{\mbox{\bf L}\xspace}
\newcommand\ma{\mbox{\bf MA}\xspace}
\newcommand\am{\mbox{\bf AM}\xspace}
\newcommand\nl{\mbox{\bf NL}\xspace}
\newcommand\conl{\mbox{\bf coNL}\xspace}
\newcommand\sharpp{\mbox{\#{\bf P}}\xspace}
\newcommand\parityp{\mbox{$\oplus$ {\bf P}}\xspace}
\newcommand\ip{\mbox{\bf IP}\xspace}
\newcommand\pcp{\mbox{\bf PCP}}
\newcommand\dtime{\mbox{\bf DTIME}}
\newcommand\ntime{\mbox{\bf NTIME}}
\newcommand\dspace{\mbox{\bf SPACE}\xspace}
\newcommand\nspace{\mbox{\bf NSPACE}\xspace}
\newcommand\cnspace{\mbox{\bf coNSPACE}\xspace}
\newcommand\exptime{\mbox{\bf EXP}\xspace}
\newcommand\nexptime{\mbox{\bf NEXP}\xspace}
\newcommand\genclass{\mbox{$\cal C$}\xspace}
\newcommand\cogenclass{\mbox{\bf co$\cal C$}\xspace}
\newcommand\size{\mbox{\bf SIZE}\xspace}
\newcommand\sig{\mathbf \Sigma}
\newcommand\pip{\mathbf \Pi}

%%Computational problems
\newcommand\sat{\mbox{SAT}\xspace}
\newcommand\tsat{\mbox{3SAT}\xspace}
\newcommand\tqbf{\mbox{TQBF}\xspace}


%\newcommand{\Ccal}{\mathcal{C}}
\newcommand{\Mcal}{\mathcal{M}}
\newcommand{\chat}{\widehat{c}}
\newcommand{\fhat}{\widehat{f}}
\newcommand{\khat}{\widehat{k}}
\newcommand{\lhat}{\widehat{l}}
\newcommand{\gns}{\gamma_{\textsc{NS}}}
\newcommand{\gst}{\gamma_{\textsc{ST}}}

\newcommand{\eps}{\varepsilon}
\newcommand{\schur}{\ensuremath{\mathsf{Schur}}}
\newcommand{\ent}[1]{\mathop{\mbox{Ent}}\left({#1}\right)}
\newcommand{\lap}{\ensuremath{\boldsymbol{L}}}

\newcommand\Ctil{\widetilde{\mathit{C}}}
\newcommand\Otil[1]{\ensuremath{\widetilde{O}\left(#1\right)}}
\newcommand\otil[1]{\ensuremath{\widetilde{\mathit{O}}(#1)}}

\newcommand\Ccal{\mathcal{C}}
\newcommand\Hcal{\mathcal{H}}


\renewcommand\AA{\boldsymbol{\mathit{A}}}
\newcommand\DD{\boldsymbol{\mathit{D}}}
\newcommand\EE{\boldsymbol{\mathit{E}}}
\newcommand\MM{\boldsymbol{\mathit{M}}}
\newcommand\MMcal{\boldsymbol{\mathcal{M}}}
\newcommand\MMbar{\boldsymbol{\overline{\mathit{M}}}}
\newcommand\MMhat{\boldsymbol{\widehat{\mathit{M}}}}
\newcommand\II{\boldsymbol{\mathit{I}}}
\newcommand\LL{\boldsymbol{\mathit{L}}}
\newcommand\LLtil{\widetilde{\boldsymbol{L}}}
\newcommand\PP{\boldsymbol{\mathit{P}}}
\newcommand\QQ{\boldsymbol{\mathit{Q}}}
\renewcommand\SS{\boldsymbol{\mathit{S}}}
\newcommand\TT{\boldsymbol{\mathit{T}}}
\newcommand\UU{\boldsymbol{\mathit{U}}}
\newcommand\XX{\boldsymbol{\mathit{X}}}
\newcommand\XXcal{\boldsymbol{\mathcal{X}}}
\newcommand\XXJcal{\boldsymbol{\mathcal{X}}\mathcal{J}}
\newcommand\YY{\boldsymbol{\mathit{Y}}}
%\newcommand\YYcal{\mathcal{Y}}
\newcommand\ZZ{\boldsymbol{\mathit{Z}}}
\newcommand\ZZhat{\boldsymbol{\widehat{\mathit{Z}}}}

\newcommand\cchi{\boldsymbol{\chi}}

\newcommand\simuniform{{\sim_{{\rm uniform}}}}

\newcommand\Ical{\mathcal{I}}

\newcommand\nhat{\widehat{n}}
\newcommand\mhat{\widehat{m}}

\newcommand\xhat{\widehat{x}}
\newcommand\yhat{\widehat{y}}

\newcommand\Hhat{\widehat{H}}

\newcommand\Gtil{\widetilde{G}}
\newcommand\Ktil{\widetilde{K}}

\newcommand\Kcal{\mathcal{K}}
\newcommand\Kcaltil{\widetilde{\mathcal{K}}}

\newcommand\Scal{\mathcal{S}}
\newcommand\Scalhat{\widehat{\mathcal{S}}}

\newcommand\epsbar{\overline{\eps}}
\newcommand\epshat{\widehat{\eps}}

\newcommand\AbsMatrix[1]{\mbox{Abs}_{2}\left| #1 \right|}

\newcommand\dd{\boldsymbol{\mathit{d}}}
\newcommand\ww{\boldsymbol{\mathit{w}}}
\newcommand\xx{\boldsymbol{\mathit{x}}}
\newcommand\yy{\boldsymbol{\mathit{y}}}

\newcommand\rrtil{\boldsymbol{\widetilde{\mathit{r}}}}
\newcommand\rr{\boldsymbol{\mathit{r}}}

\newcommand\PPi{\boldsymbol{\mathit{\Pi}}}
\newcommand\one{\vec{1}}

\newcommand{\cupdot}{\mathbin{\mathaccent\cdot\cup}}

\newcommand{\DegPreserveSparsify}{{\textsc{DegreePreservingSparsify}}}
\newcommand{\SpectralSketch}{{{\textsc{SpectralSketch}}}}
\newcommand{\MoveEdges}{{\textsc{MoveEdges}}}
\newcommand{\MoveEdgesExpander}{{\textsc{MoveEdgesExpander}}}
\newcommand{\ExtractBoundedDegreeGraph}{{\textsc{ExtractBoundedDegreeGraph}}}
\newcommand{\NaiveCycleDecomposition}{{\textsc{NaiveCycleDecomposition}}}
\newcommand{\ShortCycleAlgo}{{\textsc{ShortCycleDecomposition}}}
\newcommand{\CycleDecomposition}{{\textsc{CycleDecomposition}}}
\newcommand{\ExpanderDecompose}{{\textsc{ExpanderDecompose}}}
\newcommand{\NSExpanderDecompose}{{\textsc{NSExpanderDecompose}}}
\newcommand{\SparsifyOnce}{{\textsc{SparsifyOnce}}}
\newcommand{\SampleMatchings}{{\textsc{SampleMatchings}}}
\newcommand{\ImplicitPartitionAndSample}{{\textsc{ImplicitPartitionAndSample}}}
\newcommand{\SC}{{\textsc{SC}}}
\newcommand{\MakeBalanced}{{\textsc{MakeBalanced}}}
\newcommand{\DecomposeAndSample}{{\textsc{DecomposeAndSample}}}
\newcommand{\SampleBiCliques}{{\textsc{SampleBiCliques}}}
\newcommand{\SampleMatching}{{\textsc{SampleMatching}}}
\newcommand{\EulerianSparsify}{{\textsc{EulerianSparsify}}}
\newcommand{\DirectedSparsifyOnce}{{\textsc{DirectedSparsifyOnce}}}
\newcommand{\ImplicitSketchUnweightedBiCliques}{{\textsc{ImplicitSketchUnweightedBiCliques}}}
\newcommand{\UnweightedDecompose}{{\textsc{UnweightedDecompose}}}
\newcommand{\SchurSparse}{{\textsc{SchurSparse}}}

\newcommand{\vol}[1]{\operatorname{vol}\left({#1}\right)}
\newcommand{\reff}{\mathop{R}_{\textrm{eff}}}

\newcommand{\trace}{\mbox{Trace}}

\newcommand{\dir}[1]{{\vec{#1}}}
\newcommand{\undir}[1]{\mathit{undir}\left( #1 \right)}
\newcommand{\undirInline}[1]{\mathit{undir}\left( #1 \right)}



\newcommand\expec[2]{\operatorname*{\mathbb{E}}_{#1}\left[ #2 \right]}

\newcommand{\etal}{\emph{et al.}}


\newcommand{\LCycle}{\textsc{L}_{\textsc{CycleDecomp}}}
\newcommand{\TCycle}{\textsc{T}_{\textsc{CycleDecomp}}}

\newenvironment{tight_enumerate}{
\begin{enumerate}
  \setlength{\itemsep}{2pt}
  \setlength{\parskip}{1pt}
  \setlength{\partopsep}{1pt}
}{\end{enumerate}}
\newenvironment{tight_itemize}{
\begin{itemize}
  \setlength{\itemsep}{2pt}
  \setlength{\parskip}{1pt}
  \setlength{\partopsep}{1pt}
}{\end{itemize}}

\newcommand{\Err}{\mathrm{Err}}
\newcommand{\poly}{\mathrm{poly}}
\newcommand{\sign}{\mathrm{sign}}
\renewcommand{\tr}{\mathrm{tr}}
\newcommand{\polylog}{\mathrm{polylog}}
\newcommand{\Ceff}{\mathtt{Ceff}}
\newcommand{\Ham}{\mathrm{Ham}}
\newcommand{\wh}{\widehat}
\newcommand{\wt}{\widetilde}
\newcommand{\ov}{\overline}
\newcommand{\mc}{\mathcal}
\renewcommand{\tilde}{\wt}
\renewcommand{\hat}{\wh}
\renewcommand{\bar}{\ov}
\renewcommand{\k}{\mathsf{K}}
\newcommand{\Reff}{\mathtt{Reff}}
\newcommand{\true}{\mathtt{true}}
\newcommand{\false}{\mathtt{false}}
\newcommand{\KAdjE}{\mathsf{KAdjE}}
\newcommand{\KLapE}{\mathsf{KLapE}}
\newcommand{\KDF}{\mathrm{KDF}}
\newcommand{\KDE}{\mathrm{KDE}}
\newcommand{\cts}{\mathrm{cts}}
\newcommand{\dis}{\mathrm{dis}}
\newcommand{\WSPD}{\mathrm{WSPD}}
%\newcommand{\dist}{\mathrm{dist}}
\renewcommand{\d}{\mathrm{d}}
\newcommand{\bdeg}{\mathrm{bdeg}}
\newcommand{\SETH}{\mathsf{SETH}}
\newcommand{\ANN}{\mathsf{ANN}}
\newcommand{\RecoverExpander}{\textsc{RecoverExpander}}
\newcommand{\SparsifyInner}{\textsc{SparsifyInner}}
\newcommand{\LowDiamSet}{\textsc{LowDiamSet}}
\newcommand{\ReffQuery}{\textsc{ReffQuery}}
\newcommand{\ReffPreproc}{\textsc{ReffPreproc}}
\newcommand{\UnweightedCover}{\textsc{UnweightedCover}}
\newcommand{\BoundedCover}{\textsc{BoundedCover}}
\newcommand{\TwoBoundedCover}{\textsc{TwoBoundedCover}}
\newcommand{\LogCover}{\textsc{LogCover}}
\newcommand{\DesiredCover}{\textsc{DesiredCover}}
\newcommand{\IntervalFamily}{\textsc{IntervalFamily}}
\newcommand{\OversamplingWithCover}{\textsc{OversamplingWithCover}}
\newcommand\X{{\cal X}}
\newcommand\Y{{\cal Y}}
\newcommand{\rank}{\operatorname{rank}}
\newtheorem{question}[theorem]{Question}
\newtheorem{fact}[theorem]{Fact}

\makeatletter
\newcounter{savesection}
\newcounter{apdxsection}
\renewcommand\appendix{\par
  \setcounter{savesection}{\value{section}}%
  \setcounter{section}{\value{apdxsection}}%
  \setcounter{subsection}{0}%
  \gdef\thesection{\@Alph\c@section}}
\newcommand\unappendix{\par
  \setcounter{apdxsection}{\value{section}}%
  \setcounter{section}{\value{savesection}}%
  \setcounter{subsection}{0}%
  \gdef\thesection{\@arabic\c@section}}
\makeatother
    \newcommand{\boundary}{\partial}
    \newcommand{\smid}{\,\middle|\,}

\newcommand{\Lone}{{L^1}}
%\newcommand{\Lone}{{L^1(\Omega)}}
\newcommand{\lone}[1]{\norm{#1}_\Lone}
\newcommand{\Ltwo}{{L^2}}
%\newcommand{\Ltwo}{{L^2(\Omega)}}
\newcommand{\ltwo}[1]{\norm{#1}_\Ltwo}
\newcommand{\Ltwoo}{{L^2(\Omega)/\Re}}
\newcommand{\ltwoo}[1]{\norm{#1}_\Ltwoo}
\newcommand{\lltwo}[1]{\left\|#1\right\|_\Ltwo}
\newcommand{\Linf}{{L^\infty}}
%\newcommand{\Linf}{{L^\infty(\Omega)}}
\newcommand{\linf}[1]{\norm{#1}_{\Linf}}
\newcommand{\Lp}{{L^p(\Omega)}}
\newcommand{\lp}[1]{\norm{#1}_\Lp}
\newcommand{\Lpp}{{L^{p'}(\Omega)}}
\newcommand{\lpp}[1]{\norm{#1}_\Lpp}
\newcommand{\vph}{\vphantom{A^{A}_{A}}}
\newcommand{\ubar}{{\bar{u}}}
\newcommand{\eqnref}[1]{(\ref{eqn#1})}
    \newcommand{\grad}{\nabla}
    \DeclareMathOperator{\supp}{supp}
\newcommand{\bbR}{\mathbb{R}}
\newcommand{\Lonea}{{L^1_\alpha}}
%\newcommand{\Lonea}{{L^1_\alpha(\Omega)}}
\newcommand{\lonea}[1]{\norm{#1}_\Lonea}
\newcommand{\Ltwoa}{{L^2_\alpha}}
%\newcommand{\Ltwoa}{{L^2_\alpha(\Omega)}}
\newcommand{\ltwoa}[1]{\norm{#1}_\Ltwoa}

\newcommand{\alphahat}{\hat{\alpha}}
\newcommand{\betahat}{\hat{\beta}}
\newcommand{\gammahat}{\hat{\gamma}}
\newcommand{\deltahat}{\hat{\delta}}
\newcommand{\epsilinhat}{\hat{\epsilon}}
\newcommand{\varepsilinhat}{\hat{\varepsilon}}
\newcommand{\zetahat}{\hat{\zeta}}
\newcommand{\etahat}{\hat{\eta}}
\newcommand{\thetahat}{\hat{\theta}}
\newcommand{\varthetahat}{\hat{\vartheta}}
\newcommand{\iotahat}{\hat{\iota}}
\newcommand{\kappahat}{\hat{\kappa}}
\newcommand{\lambdahat}{\hat{\lambda}}
\newcommand{\muhat}{\hat{\mu}}
\newcommand{\nuhat}{\hat{\nu}}
\newcommand{\xihat}{\hat{\xi}}
\newcommand{\pihat}{\hat{\pi}}
\newcommand{\varpihat}{\hat{\varpi}}
\newcommand{\rhohat}{\hat{\rho}}
\newcommand{\varrhohat}{\hat{\varrho}}
\newcommand{\sigmahat}{\hat{\sigma}}
\newcommand{\varsigmahat}{\hat{\varsigma}}
\newcommand{\tauhat}{\hat{\tau}}
\newcommand{\upsilonhat}{\hat{\upsilon}}
\newcommand{\phihat}{\hat{\phi}}
\newcommand{\varphihat}{\hat{\varphi}}
\newcommand{\chihat}{\hat{\chi}}
\newcommand{\psihat}{\hat{\psi}}
\newcommand{\omegahat}{\hat{\omega}}

\newcommand{\Gammahat}{\hat{\Gamma}}
\newcommand{\Deltahat}{\hat{\Delta}}
\newcommand{\Thetahat}{\hat{\Theta}}
\newcommand{\Lambdahat}{\hat{\Lambda}}
\newcommand{\Xihat}{\hat{\Xi}}
\newcommand{\Pihat}{\hat{\Pi}}
\newcommand{\Sigmahat}{\hat{\Sigma}}
\newcommand{\Upsilonhat}{\hat{\Upsilon}}
\newcommand{\Phihat}{\hat{\Phi}}
\newcommand{\Psihat}{\hat{\Psi}}
\newcommand{\Omegahat}{\hat{\Omega}}
\newcommand{\sst}{\,\mid\,}

\newcommand{\ahat}{\hat{a}}
\newcommand{\bhat}{\hat{b}}
\newcommand{\dhat}{\hat{d}}
\newcommand{\ehat}{\hat{e}}
\newcommand{\ghat}{\hat{g}}
\newcommand{\hhat}{\hat{h}}
\newcommand{\ihat}{\hat{i}}
\newcommand{\jhat}{\hat{j}}
\newcommand{\ellhat}{\hat{\ell}}
\newcommand{\ohat}{\hat{o}}
\newcommand{\phat}{\hat{p}}
\newcommand{\qhat}{\hat{q}}
\newcommand{\rhat}{\hat{r}}
\newcommand{\shat}{\hat{s}}
\newcommand{\that}{\hat{t}}
\newcommand{\uhat}{\hat{u}}
\newcommand{\vhat}{\hat{v}}
\newcommand{\what}{\hat{w}}
\newcommand{\zhat}{\hat{z}}

\newcommand{\Ahat}{\hat{A}}
\newcommand{\Bhat}{\hat{B}}
\newcommand{\Chat}{\hat{C}}
\newcommand{\Dhat}{\hat{D}}
\newcommand{\Ehat}{\hat{E}}
\newcommand{\Fhat}{\hat{F}}
\newcommand{\Ghat}{\hat{G}}
\newcommand{\Ihat}{\hat{I}}
\newcommand{\Jhat}{\hat{J}}
\newcommand{\Khat}{\hat{K}}
\newcommand{\Lhat}{\hat{L}}
\newcommand{\Mhat}{\hat{M}}
\newcommand{\Nhat}{\hat{N}}
\newcommand{\Ohat}{\hat{O}}
\newcommand{\Phat}{\hat{P}}
\newcommand{\Qhat}{\hat{Q}}
\newcommand{\Rhat}{\hat{R}}
\newcommand{\Shat}{\hat{S}}
\newcommand{\That}{\hat{T}}
\newcommand{\Uhat}{\hat{U}}
\newcommand{\Vhat}{\hat{V}}
\newcommand{\What}{\hat{W}}
\newcommand{\Xhat}{\hat{X}}
\newcommand{\Yhat}{\hat{Y}}
\newcommand{\Zhat}{\hat{Z}}

\newcommand{\dbydp}[2]{\frac{\partial #1}{\partial #2}}
