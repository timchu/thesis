\section{Conclusion and Future Directions}\label{sec:conclusion}
We define a new notion of spectral sweep cuts, eigenvalues,
Rayleigh quotients, and sparsity for probability densities. We
present the first known Cheeger and Buser inequality on Lipschitz
probability density functions, and use this to show an
$(\alpha=1, \gamma=3)$ spectral
sweep cut on a $L$-Lipschitz probability density function has
provably low $(\alpha=1,\gamma=2)$-sparsity. This work is the first spectral sweep
cut algorithm on non-parametric probability densities with any guarantees on the cut quality.

Further, we show that existing spectral sweep cut methods (such
as those implicit in spectral clustering) compute
$(1, 1)$ or $(1,2)$ spectral sweep cuts, neither of which
has any sparsity guarantees. We prove that $(1,2)$ spectral sweep
cuts, which are implicitly used in traditional spectral
clustering, can lead to undesirable partitions of simple 
$1$-Lipschitz probability densities. Meanwhile, our work showed
that using $(1,3)$ spectral sweep cuts give provably good $(1,2)$
sparse cuts.

For future directions, we conjecture that $\beta =
\alpha+1$ and $\gamma = \alpha+2$ is the only settings of
$(\alpha, \beta, \gamma)$ in which both Cheeger and Buser
inequalities are provable. This would be a stronger theorem than
we currently have for Lemma~\ref{lem:cheeger-converse} and
Lemma~\ref{lem:buser-converse}.

In the Buser inequality, we would like to iron out the exact dimensional
dependence on the dimension,
$d$ (Theorem~\ref{thm:buser_n}). The
authors believe that this dependence can be
reduced to $\sqrt{d}$. It is an open question whether
\textit{any}
dimension dependence is required.  In particular, the latest
version of
Buser's inequality for manifolds has no dimension
dependence~\cite{ledoux2004spectral}. It is an open question how to
generalize their techniques into the density setting, as the
Bochner formula does not easily generalize to densities.

Another open question is whether multi-way Cheeger and Buser inequalities
can be proven on densities, mirroring the work on
graphs~\cite{Louis12, kw16, LeeMultiway14, Lee2014}. This would
allow our clustering algorithms to generalize into $k$-way
clusterings. We additionally would like to know whether one can
understand \textbf{balanced cuts} on proability densities for our
new definitions of sparsity. Balanced cuts in this setting may
have applications to machine learning.

Finally, we would like to know whether Buser and Cheeger inequalities may exist for $L$-Lipschitz
probability densities supported on manifolds with bounded
curvature. If true, this would fully generalize the work
of Cheeger and Buser on manifolds, which may lead to deeper
insight into manifold theory. Moreover, it could have
foundational impact:
a fundamental assumption underlying modern machine learning is
that most data comes from probability density supported on a manifold, and a Cheeger and Buser inequality in this
setting would give provable sparsity guarantees about
spectral sweep cuts in this setting.
