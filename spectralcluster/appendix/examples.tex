\section{Calculating Eigenvalues and Isoperimetry constants for Simple
  Examples}\label{app:examples}
Recall from Section~\ref{sec:examples} the definitions of $\rho_1$ and
$\rho_2$. This section is devoted to computing the eigenvalues and
isoperimetric constants of these densities. We note that the eigenvalue
and isoperimetry computation for $\rho_1$ is straightforward, so we omit
it. The isoperimetry constant for $\rho_2$ is also straightforward, as
the isoperimetric cut will be at $x = 0$. The only non-trivial
computation is the $(\alpha,\beta)$-eigenvalue for $\rho_2$.

\subsection{Notation}
We will write $a\gtrsim b$ if $a\geq cb$ for some absolute constant $0<c<\infty$. Similarly define $a \lesssim b$. We will write $a\asymp b$ if both relations hold.

\subsection{A Lipschitz weight}
\label{subsec:lipschitz_example}

It is clear that
\begin{align*}
\Phi \asymp \epsilon^\beta.
\end{align*}
Next, we apply the Hardy-Muckenhoupt inequality~\cite{MillerHardy18} to estimate $\lambda_2$
for $\rho_2$. 

We upper bound $\H$ as:

\begin{align*}
\H &\leq\R(1)\M(0)\\
  &\asymp  \int_0^1 \frac{1}{(x+\epsilon)^{\gamma}}\,dx\\
& \lesssim \begin{cases}
  1 & \text{if } \gamma<1\\
  \ln\left(1/\epsilon\right)& \text{if } \gamma = 1\\
  O\epsilon^{1-\gamma}& \text{if } \gamma>1.
\end{cases}
\end{align*}
By the Hardy-Muckenhoupt inequality, we can lower bound $\lambda_2$ with
the inverse of an upper bound on $\H$. Thus, as claimed in
Section~\ref{sec:examples}, we can lower bound $\lambda_2$ with
$\epsilon^{\gamma - 1}$ when $\gamma \geq 1$.

Thus, if we want a Buser-type inequality to hold, then $(\alpha,\beta,\gamma)$ needs to satisfy,
\begin{align*}
\begin{cases}
  1\lesssim \lambda_2 \lesssim \max(\Phi,\Phi^2)  \asymp \epsilon^\beta & \text{if } \gamma<1\\
  \frac{1}{\ln\left(1/\epsilon\right)}\lesssim \lambda_2 \lesssim \max(\Phi,\Phi^2)  \asymp \epsilon^\beta& \text{if } \gamma = 1\\
  \epsilon^{\gamma-1}\lesssim \lambda_2 \lesssim \max(\Phi,\Phi^2)  \asymp \epsilon^\beta& \text{if } \gamma>1.
\end{cases}
\end{align*}
By letting $\epsilon$ go to zero, it is clear that $\gamma-1\geq \beta$,
   as desired.

