\section{Cheeger and Buser for Density Functions does not easily follow
  from Graph or Manifold Cheeger and Buser}\label{app:notgraph}

\subsection{Comments on Graph Cheeger-Buser}
The most natural method of proving distributional Cheeger-Buser
inequality using the graph Cheeger-Buser inequality is to
generate a vertex and edge weighted graph approximating the distribution, and write down
graph Cheeger-Buser. Then, one would generate a sequence of graphs with
an increasing number of vertices. Ideally, the graph Cheeger-Buser inequality
on these graphs would converge to a
Cheeger-Buser inequality on the underlying distribution. This
discretization approach follows a standard paradigm of
approximating distributions with graphs, present in
numerical methods, finite element methods,
         and machine learning
         ~\cite{TrillosRate15,TrillosVariational15,SPIELMAN2007284}.

Such an approach cannot work (no matter how the
    eigenvalues and isoperimetric cuts are defined for
    distributions). The easiest way to see this is to attempt to execute
    this strategy for a simple uniform distribution in $1$ dimension, on
    the interval $[0,1]$. One would naively approximate this
    distribution
    with a line graph with $n$ vertices, with edge weights $w_n$ and vertex
    weights $m_n$. Then one would take $n$ to go to infinity.

    If one writes down the Cheeger and Buser inequalities for graphs in
    this example, we get:
    
    \[\frac{w_n}{m_n n^2} \leq \Phi_{OPT} \leq \frac{w_n}{m_n n} \]

    No matter what $m_n$ and $w_n$ are, the ratio between the upper and
    lower bound is $n$, which diverges. Thus, either the Cheeger
    inequality or the Buser inequality becomes meaningless: either the
    lower bound goes to $0$ or the upper bound goes to $\infty$, or
    both, depending on
    how $w_n$ and $m_n$ are set.

    Thus, even for the simple case of a uniform distribution on $[0,1]$
    the natural strategy for deriving probability density Cheeger/Buser
    from graph Cheeger/Buser fails.

\subsection{Comments on Manifold Cheeger-Buser}
Distributional Buser does not easily follow from an
application of
the manifold Buser inequality. We recall that manifold Buser only applies for
    manifolds with bounded Ricci curvature. The natural way to parlay manifold Buser into
distributional Buser on $\RR^d$ is to change the underlying metric tensor
on $\RR^d$ to factor in the probability density function at that
point. However, the authors are unaware of any method of doing
this for which one can recover a meaningful Cheeger and
Buser inequality. Moreover, it is unclear how to obtain any Ricci
curvature bounds when we change the metric tensor.

Most modern approaches to proving Buser's inequality for
    manifolds rely on the
Li-Yau inequality, which in turn depends on the Bochner identity
    for manifolds on bounded Ricci
    curvature~\cite{ledoux2004spectral}.
The authors are unaware of a clean Bochner-like identity for
distributions. Older techniques use Almgren's minimizing currents
and/or Epsilon nets~\cite{Buser82}. For the former, we do not know of any
analog for distributions. For the latter, the corresponding
Buser inequality has a $2^{d}$ multiplicative dependence, which
is significantly worse than our $d$ dependence.
