\begin{abstract}
Cheeger and Buser inequalities relate fundamental eigenvalues with
isoperimetric constants. These inequalities for graphs are the
foundation of spectral graph theory. 

In this chapter, we introduce Cheeger and Buser inequalities for Lipschitz
probability density functions. To do this, we create new definitions of
isoperimetry and eigenvalues in this setting. For past
definitions, one of the inequalities must fail. We apply our work to
give a new spectral algorithm for partitioning probability densities,
which is a variant of classical spectral clustering. Classical spectral
clustering can fail when data comes from a nicely behaved Lipschitz
  density function, and our new variant will overcome traditional
  problems in spectral clustering when points are drawn from general
Lipschitz probability densities.  \end{abstract}
