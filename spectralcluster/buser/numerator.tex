\subsection{Upper Bounding the Numerator}
To bound the $L^2$ norm in
  the numerator of the Rayleigh quotient by the $L^1$ norm in the
  numerator of the expression for $\Phi(A)$ it is necessary to obtain
  uniform bound on $\rho(x) \nabla u_\theta(x)$. 
  
  \begin{lemma} \label{lem:rep1}
  Let $u$ be any function, and let $u_{\theta}$ be defined as in
  Equation~\ref{eqn:utheta}. Let $\rho: \Re^d \to \Re_{\geq 0}$ be an
  $L$-Lipschitz function.
  \begin{align}
  \linf{\rho (x) \nabla u_\theta(x)}
    \leq \linf{u} \frac{d(2+3L)}{\theta}
  \end{align}
  \end{lemma}

  \begin{proof}
  In order to prove this lemma, we first need to get a handle on
  $\nabla u_\theta(x)$, which is the gradient of $u$ after
  mollification by $\theta$.

  We take the
  the second representation of $u_\theta$ in equation \eqnref{:utheta}
  to get
  \begin{align}
  \nabla u_\theta(x) 
  = \int_{\bbR^d} u(z) \left\{
    \frac{-d}{\theta \rho(x)} \phi_{\theta \rho}(x-z) \nabla \rho
    + \frac{1}{(\theta \rho(x))^{d+1}} 
    \left( I 
      + \nabla \rho(x) \otimes \frac{x-z}{\theta \rho(x)} \right)
    \nabla \phi \left(\frac{x-z}{\theta \rho(x)} \right) \right\}
    \, dz,
  \end{align}
  which is a consequence of the multivariable chain rule. Here,
  $v \otimes u$ refers to the outer product of $v$ and $u$.

  Multiplying by $\rho$ gives:
  \begin{align}
  \rho (x) \nabla u_\theta(x) 
  = \int_{\bbR^d} u(z) \left\{
    \frac{-d}{\theta} \phi_{\theta \rho(x)}(x-z) \nabla
    \rho(x)
    + \frac{1}{(\theta^{d+1} \rho(x)^{d})} 
    \left( I 
      + \nabla \rho(x) \otimes \frac{x-z}{\theta \rho(x)} \right)
    \nabla \phi \left(\frac{x-z}{\theta \rho(x)} \right) \right\} \, dz.
  \end{align}
  Now, we can bound the above equation by carefully bounding each
  part. We note:
  \begin{align} \label{eq:Cphi-calculation}
    & \int_{\bbR^d} \frac{1}{(\theta^{d+1} \rho(x)^d)} \nabla
    \phi\left(\frac{x-z}{\theta \rho(x)} \right) dz
    \\
    & = \int_{\bbR^d} \frac{1}{(\theta^{d+1} \rho(x)^d)} \nabla
    \phi\left(\frac{-z}{\theta \rho(x)} \right) dz
    \\
     & = \frac{1}{\theta} \int_{\bbR^d} \nabla \phi(-y) dy
  \end{align}
  where the last step follows by a simple change of variable.
  Here, we note that $\nabla \phi(y)$ is a vector, and the
  integral is over $\bbR^d$, which is how we eliminated
  $\frac{1}{(\theta \rho(x))^d}$ from the expression.

  Next, we examine the term: 
  \begin{align}
    I + \nabla \rho(x) \otimes \frac{x-z}{\theta \rho(x)}
  \end{align}
  Here, we aim to bound the operator norm of this matrix. Here,
  we note that 
  \[ |x - z| \leq \theta \rho(x) \]
  when 
  \[
    \nabla \phi\left(\frac{x-z}{\theta \rho(x)}\right) \not= 0
  \]
  and thus, when the latter equation holds, we can say:

  \[
    \left| \frac{x-z}{\theta \rho(x)}\right| < 1.
  \]
  Since $|\nabla \rho(x) < L|$, we now have:
  \begin{align}\label{eqn:num-matrix-norm-bound}
    |I + \nabla \rho(x) \otimes \frac{x-z}{\theta \rho(x)}|_2 <
    3/2
  \end{align}
  Combining Equation~\ref{eqn:num-matrix-norm-bound} Equation~\ref{eq:Cphi-calculation} to show:
  \begin{align}
    & \left | \int_{\bbR^d}
    \frac{1}{(\theta^{d+1} \phi(x)^{d})} \left( I + \nabla \rho(x) \otimes \frac{x-z}{\theta \rho(x)}
    \right) \nabla \phi \left( \frac{x-z}{\theta \rho(x)}
    \right) dz \right| 
    \\
    & \leq \frac{(1 + L)}{\theta} \int_{\bbR^d} |\nabla\phi(y) | dy,
  \end{align}
  where $L = \linf{\nabla \rho}$ is the Lipschitz constant for $\rho$. 
  We note that Section~\ref{sec:dimension-dep} shows that 
  \begin{align}
    \int_{\bbR^d} | \nabla \phi(y) dy| \leq 2d.
  \end{align}
  and therefore:
  \begin{align}
    & \left | \int_{\bbR^d}
    \frac{1}{(\theta^{d+1} \phi(x)^{d})} \left( I + \nabla \rho(x) \otimes \frac{x-z}{\theta \rho(x)}
    \right) \nabla \phi \left( \frac{x-z}{\theta \rho(x)}
    \right) dz \right| 
    \\ \label{eq:rho-grad-utheta-2}
    & \leq \frac{2d(1+L)}{\theta}
  \end{align}

  Now we turn our attention to the first term, which is:
  \begin{align}
    \int_{\bbR^d} \frac{-d}{\theta} \phi_{\theta \rho(x)}(x-z)
    \nabla\rho(x) dz
  \end{align}
  We note that 
  \[ 
    \int_{\bbR^d}\left| \phi_{\theta \rho(x)}(x-z)\right| dz
    = 1
  \]
  by our definition of $\phi$ (which was defined when we defined
      $u_\theta$). Combining this
  with $|\nabla \rho(x)| < L$, we get:
  \begin{align}
    & \int_{\bbR^d} \left | \frac{-d}{\theta} \phi_{\theta \rho(x)}(x-z)
    \nabla\rho(x) \right| dz
    \\ \label{eq:rho-grad-utheta-1}
    & < \frac{dL}{\theta}
  \end{align}
  Therefore, 
  \begin{align}
  &  \nonumber \left| \int_{\bbR^d} 
    \frac{-d}{\theta \rho(x)} \phi_{\theta \rho}(x-z) \nabla \rho
    + \frac{1}{(\theta \rho(x))^{d+1}} 
    \left( I 
      + \nabla \rho(x) \otimes \frac{x-z}{\theta \rho(x)} \right)
    \nabla \phi \left(\frac{x-z}{\theta \rho(x)} \right) \, dz
    \right|
    \\
    & \nonumber \leq \frac{d}{\theta}( L + 2(1+L)) 
    \\
  & = \frac{d(2+3L)}{\theta}.
  \label{eqn:}
  \end{align}
  where the first inequality comes from combining
  Equations~\ref{eq:rho-grad-utheta-2}
  and~\ref{eq:rho-grad-utheta-1}.

  This allows us to bound $\linf{\rho(x) \nabla u_{\theta}(x)}$:
  \begin{align}
  & \linf{\rho (x) \nabla u_\theta(x)}
  \\
  & = \linf{ \left| \int_{\bbR^d} u(z) \left\{
    \frac{-d}{\theta} \phi_{\theta \rho(x)}(x-z) \nabla
    \rho(x)
    + \frac{1}{(\theta^{d+1} \rho(x)^{d})} 
    \left( I 
      + \nabla \rho(x) \otimes \frac{x-z}{\theta \rho(x)} \right)
    \nabla \phi \left(\frac{x-z}{\theta \rho(x)} \right) \right\}
  \, dz \right| }
  \\
  & \leq \linf{u} \linf{ \int_{\bbR^d} \left| 
    \frac{-d}{\theta} \phi_{\theta \rho(x)}(x-z) \nabla
    \rho(x)
    + \frac{1}{(\theta^{d+1} \rho(x)^{d})} 
    \left( I 
      + \nabla \rho(x) \otimes \frac{x-z}{\theta \rho(x)} \right)
    \nabla \phi \left(\frac{x-z}{\theta \rho(x)} \right)
  \right| \, dz}
  \\
  & \leq \linf{u} \frac{d(2+3L)}{\theta}
  \end{align}
  where we make use of the fact that
    $\linf{ab} < \linf{a}\lone{b}.$
  This completes our proof.
  \end{proof}

  Next, we want an $L_1$ bound on $\rho^{\beta}(x) \nabla
  u_{\theta}(x)$.

  \begin{lemma} \label{lem:rep2}
  Let $u$ be any function, and let $u_{\theta}$ be defined as in
  Equation~\ref{eqn:utheta}. Let $\rho:\Re^d \to \Re_{\geq 0}$ be an
  $L$-Lipschitz function, and
  let $\theta L < 1/2$.

  Then:
  \begin{align}
      \lone{\rho^\beta(x) \nabla u_{\theta}(x)} \leq C_{\beta}
      \lone{\rho^\beta(x) \nabla u(x)}
  \end{align}
  \end{lemma}
  \begin{proof}
  First, we take the gradient 
  first representation of $u_\theta$
  in equation \eqnref{:utheta}. Using the chain rule gives us an
  alternate form for $\nabla u_\theta(x)$:

  \begin{align}
  \nabla u_\theta(x) 
  = \int_{\Re^d} 
  (I - \theta \nabla \rho \otimes y) \nabla u(x-\theta \rho y) \phi(y) \, dy,
  \end{align}
  so
  \begin{align}\label{eqn:rhoNablaUtheta}
  \rho^\beta(x) \nabla u_\theta(x) 
  = \int_{\Re^d} 
  (I - \theta \nabla \rho \otimes y)
  \frac{\rho^\beta(x)}{\rho^\beta(x-\theta \rho y)}
  \rho^\beta(x-\theta \rho y) \nabla u(x-\theta \rho y) \phi(y) \, dy.
  \end{align}

  The ratio in the integrand is bounded using the Lipschitz assumption
  on $\rho$ (and $|y| \leq 1$),
  \begin{equation} \label{eqn:rhoRatio}
    \frac{\rho(x)}{\rho(x-\theta \rho y)}
    \leq \frac{\rho(x)}{\rho(x) - L \theta \rho(x)}
    = \frac{1}{1 - L \theta} \leq 2,
    \qquad \text{ when } \theta < 1 / (2L).
  \end{equation}
  Note that 
  \begin{align} \label{eqn:matrix-norm} 
  \left\| I - \theta \nabla \rho \otimes y \right\|_2 \leq 3/2
  \end{align}
  where $\|M\|_2$ represents the $\ell^2$ matrix norm of $M$. This is
  because $|\nabla \rho(x) | \leq L$, and
  $\theta L < 1/2$, and $|y| \leq
  1$ every time $\phi(y) \not= 0$, and thus 
  \[ 
   \frac{I}{2} \preceq  I - \theta \nabla \rho \otimes y
   \preceq \frac{3I}{2}.
    \]
  
 Therefore, we can now apply Corollary~\ref{cor:lonetheta}
 to Equation~\eqnref{:rhoNablaUtheta} to show:
  \begin{align}
  \nonumber 
  & \lone{\rho^\beta(x) \nabla u_\theta(x)} 
  \\  \nonumber
  & \leq \|I - \theta \nabla \rho(x) \otimes y\|_2
  \cdot \max_x\left(\frac{\rho(x)}{\rho(x - \theta \rho
        y)}\right) \cdot
  \int_{\bbR^d} \left| \int_{\bbR^d} \rho^{\beta}(x - \theta \rho y) \nabla u(x - \theta \rho y)
    \phi(y) dy \right|
  \\ \nonumber 
  & \leq 3 \cdot 2^{\beta - 1} \int_{\bbR^d} \rho^\beta(x -
      \theta \rho(x) y) \nabla u (x - \theta \rho (x) y 
  \\ \nonumber
  & 
  \leq 3 \cdot 2^\beta \lone{\rho^\beta \nabla u},
  \qquad \text{ when } \theta < 1 / (2L).
  \end{align}
  here, the first inequality comes from the equation $\lone{abc}
  \leq \linf{a}\linf{b}\lone{c}$, the second inequality comes
  from Equations~\ref{eqn:rhoRatio} and~\ref{eqn:matrix-norm}, and the
  third inequality comes
  from Corollary~\ref{cor:lonetheta} assuming $\theta L \leq
  1/2$. 
  \end{proof}

  \begin{lemma}\label{lem:num}
  For any $L$-Lipschitz distribution $\rho$, any function $u$, and any $\theta$ such that $\theta L <
  1/2$:

  \begin{align}
  \int_{\bbR^d} \rho^\gamma |\nabla u_\theta|^2
  \leq  C_\beta \linf{\rho^{\gamma-\beta-1}} \frac{d(2+3L)}{\theta} 
  \linf{u} \lone{\rho^\beta \nabla u},
  \end{align}
  \end{lemma}

  \begin{proof}
  Combining the two estimates from Lemma~\ref{lem:rep1}
  and~\ref{lem:rep2} gives an upper bound for the Rayleigh
  quotient 
  \begin{align}
  \int_{\bbR^d} \rho^\gamma |\nabla u_\theta|^2
  = \int_{\bbR^d} \rho^{\gamma-\beta-1} \,
  \rho |\nabla u_\theta| \, \rho^\beta |\nabla u_\theta|
  \leq 3 \cdot 2^{\beta + 1} \linf{\rho^{\gamma-\beta-1}} \frac{d(2+3L)}{\theta} 
  \linf{u} \lone{\rho^\beta \nabla u},
  \end{align}
  \end{proof}

  We note that in the case where $\gamma = \beta + 1$, and if $u$ is a step
  function, the expression would
  simplify to:
  \[ \int_{\bbR^d}
  |\rho^{\gamma} |\nabla u_\theta|^2
  \leq 3 \cdot 2^{\beta + 1} \frac{d(2+3L)}{\theta} 
  \linf{u} \lone{\rho^\beta \nabla u},
  \]
