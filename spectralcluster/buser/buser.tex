%!TEX root = ms.tex

\section{Buser Inequality for Probability Density Functions} \label{sec:buser}

In this section the theory of functions of bounded variation is used to
justify certain formal calculations. The key step is to define the geometric
quantities variationally.

\begin{definition}
  For a measurable set $A\subseteq \Re^d$ and $\rho:\Re^d \rightarrow
  \Re_{\geq 0}$ integrable define
  \begin{align*}
    \abs{A}_\alpha \coloneqq \int_A \rho^\alpha(x)\,dx.
  \end{align*}
\end{definition}

We define the weighted boundary area
variationally~\cite{betta2008weighted,parini2011introduction}.

\begin{definition}
\label{def:betaBdy}
For $\rho:\Re^d \rightarrow \Re_{\geq 0}$ integrable and $A\subseteq\Re^d$
the weighted perimeter of $A$ is
\begin{align*}
\abs{\boundary A}_\beta \coloneqq \sup\set{
\int_A \div(\rho^\beta (x) \phi(x))\,dx
\smid \phi\in C_c^1(\Re^d),\, \norm{\phi}_{\infty}\leq 1}.
\end{align*}
\end{definition}

\begin{remark}
  This definition corresponds to the intuitive definition of the
  boundary integral of $\rho^\beta$ when $\boundary A$ is sufficiently
  regular.  Specifically, if $A\subseteq\Re^d$ has smooth boundary
  and then,
  $$
  \abs{\boundary A}_\beta = \int_{\boundary A} \rho^\beta(x)\,d \mc H^{n-1}(x),
  $$
  where $\mc H^{n-1}$ denotes surface (Hausdorff) measure.
\end{remark}

\begin{definition}
  Let $A\subseteq\Re^d$ be a set of finite perimeter such that
  $\abs{A}_\alpha,\abs{\Re^d\setminus A}_\alpha >0$. The
  \textit{isoperimetric ratio} of the cut induced by $A$ is
\begin{align*}
\Phi(A) &\coloneqq \frac{\abs{\boundary A}_\beta}
{\min\left(\abs{A}_\alpha,\abs{\Re^d\setminus A}_\alpha\right)}
\end{align*}
and the \textit{isoperimetric constant} with weight $\rho$ is
\begin{align*}
  \Phi \coloneqq \inf_{A\subseteq \Re^d} \Phi(A).
\end{align*}
Here, $A$ is taken over sets of finite perimeter such that
$\abs{A}_\alpha,\abs{\Re^d \setminus A}_\alpha>0$.
\end{definition}

\subsection{Weighted Buser-type Inequality}

We now prove our weighted Buser-type inequality
Theorem~\ref{thm:Cheeger-Buser}. We state our result in terms of
general $(\alpha, \beta, \gamma)$.

\begin{theorem}
  \label{thm:buser_n}
  Let $\rho: \Re^d \to \Re_{\geq 0}$ be an $L$-Lipschitz function,
  $\lambda_2$ be a $(\alpha, \gamma)$-principal eigenvalue, and
  $\Phi$ the $(\alpha, \beta)$ isoperimetric cut.

  Then:
  \begin{align*}
    \lambda_2 \leq 3 \cdot 2^{\beta + 1} d \linf{\rho^{\gamma-\beta-1}}
    \max\left(L \Phi, 2^{\beta+1}\linf{\rho^{\alpha+1-\beta}} \Phi^2 \vph\right)
\end{align*}
\end{theorem}

We note that when setting $(\alpha, \beta, \gamma) = (1,2,3)$,
the above expression simplifies into:

\begin{align*}
  \lambda_2 \leq 24 d 
  \max\left(L \Phi, 8 \Phi^2 \vph\right).
\end{align*}

\subsection{Proof Strategy: Mollification by Disks of Radius
  Proportional to \texorpdfstring{$\rho$}{rho}}
To prove Theorem \ref{thm:buser_n}, for $A \subset \Re^d$ fixed, we
construct an approximation $u_\theta$ of the characteristic function,
$u$, of $A$ for which the numerator and denominator of the Rayleigh
quotient, $R(u_\theta)$, approximate respectively the numerator and
denominator of this expression.  Specifically, $u_\theta$ will
constructed as a mollification of $u$, Recall the following two
equivalent definitions of a mollification.  They are equivalent by the
change of variables $z = x-\theta \rho(x) y$.

\begin{equation} \label{eqn:utheta}
u_\theta(x) 
\coloneqq \int_{B(0,1)} \!\!\! u(x-\theta \rho(x) y) \phi(y) \, dy
= \int \! u(z) \phi_{\theta \rho(x)}(x-z) \, dz,
\quad \text{ where } \quad
\phi_{\eta}(z) 
= \frac{1}{\eta^d} \phi\left(\frac{z}{\eta}\right),
\end{equation}

with $\theta > 0$ a parameter to be chosen and $\phi:\Re^d \rightarrow
[0,\infty)$ a smooth radially symmetric function supported in the unit
open ball $B(0,1)=\{x \in \Re^d \;| \; |x| < 1\}$ with unit mass
$\int_{\Re^d} \phi = 1$. When $\rho$ is constant it follows from the
Tonelli theorem that $\lone{u_\theta} = \lone{u}$; when $\rho$ is not
constant the following lemma shows that the latter still bounds the
former.


\subsection{Key Technical Lemma: Bounding $L_1$ norm of a
  function with the $L_1$ norm of its
    mollification}\label{sec:key_lemma}
The following is our primary technical lemma, which roughly bounds the
$L_1$ norm of a mollified function $f$ by the $L_1$
norm of the original $f$. Here, the mollification radius is
determined by a function $\delta(x)$.
\begin{lemma} \label{lem:lonetheta} 

  Let
  $\delta:\RR^d\to\RR$ be Lipschitz continuous with Lipschitz constant
  $|\grad\delta(x)| \leq c < 1$ for almost every $x \in \bbR^d$. Let
  $\phi:\RR^d\to\RR_{\geq 0}$ be smooth, $\int_{\RR^d}\phi = 1$, and
  $\supp(\phi)\subseteq B(0,1)$.  Then
  \[
  \frac{1}{1+c}\norm{f}_{L^1}
  \leq \int_{\RR^d} \int_{B(0,1)} 
  \abs{f(x-\delta(x) y)}\phi(y)\,dy\,dx\leq
  \frac{1}{1-c}\norm{f}_{L^1},
  \qquad
  f \in L^1(\RR^d).
  \]
\end{lemma}

\begin{proof} (of Lemma~\ref{lem:lonetheta})
An application of Tonelli's theorem shows
\begin{align}
\int_{\RR^d} \int_{B(0,1)} \abs{f(x-\delta(x) y)}\phi(y)\,dy\,dx
=  \int_{B(0,1)}\phi(y)\int_{\RR^d} \abs{f(x-\delta(x)y)}\,dx\,dy.
\end{align}
Fix $y\in B(0,1)$ and consider the change of variables $z = x-\delta(x)y$. The
Jacobian of this mapping is $I - y \otimes \nabla \delta(x)$ which by Sylvester's determinant theorem has
determinant $1-y.\nabla \delta(x) > 0$. It follows that
\begin{align}
\int_{\RR^d} \int_{B(0,1)} 
\abs{f(x-\delta(x) y)}\phi(y)\,dy\,dx
=  \int_{B(0,1)}\phi(y)\int_{\RR^d} 
\frac{|f(z)|}{1-y.\nabla \delta(x)} \, dx \,dy,
\end{align}
and the lemma follows since $1-c \leq 1-y.\nabla \delta(x) \leq 1+c$.
\end{proof}
(Here, $a.b$ denotes the dot product between $a$ and $b$.)

We present the following simple corollaries, which is the primary
way our proof makes use of Lemma~\ref{lem:lonetheta}

\begin{corollary}\label{cor:lonetheta}
For any Lipschitz continuous function $\rho: \bbR^d \rightarrow \bbR^{\geq 0}$
with Lipschitz constant $L$ and any $\theta$ with $0 < \theta L < 1$, we have:
$$
\frac{1}{1+\theta L} \lone{\rho^\beta\nabla u}
\leq \int_{\bbR^d} \int_{B(0,1)} \rho^\beta ( x - \theta \rho(x) y)
  |\nabla u(x-\theta \rho(x)y)| \phi(y)\,dy\,dx
 \leq \frac{1}{1-\theta L} \lone{ \rho^\beta\nabla u},
$$
when $\rho^\beta |\nabla u| \in \Lone$.
\end{corollary}

\begin{proof} (of Corollary~\ref{cor:lonetheta}) Apply
  Lemma~\ref{lem:lonetheta} with $\delta(x) = \theta \rho(x)$, and
  $f(x) = \rho^\beta(x) \nabla u(x)$.
\end{proof}

This corollary will be used to bound the numerator of our
Rayleigh quotient.
Note that the expression
\[ \int_{\bbR^d} \int_{B(0,1)} \rho^\beta ( x - \theta \rho(x) y)
|\nabla u(x-\theta \rho(x)y)| \phi(y) dy dx
\]
is close to $\int_{\bbR^d} \rho^\beta(x) \nabla |u_\theta(x)| dx$ when
$\theta \leq \frac{1}{2L}$. This is the guiding intuition behind how
Corollary~\ref{cor:lonetheta} and Lemma~\ref{lem:lonetheta} will be
used, and will be formalized later in our proof of
Theorem~\ref{thm:buser_n}.

We present another simple corollary whose proof is equally
straightforward. This corollary will be used to bound the denominator,
and is a small generalization of Corollary~\ref{cor:lonetheta}. We
write down both corollaries anyhow, since this will make it easier to
interpret our bounds on the Rayleigh quotient.

\begin{corollary}\label{cor:lone-t-theta}
For any Lipschitz continuous function $\rho: \bbR^d \rightarrow \bbR^{\geq 0}$
with Lipschitz constant $L$, any $0 < t < 1$, and any $\theta$ with $0 < \theta L < 1$, we have:
$$
\frac{1}{1+\theta L} \lone{\rho^\beta \nabla u}
\leq \int_{\bbR^d} \int_{B(0,1)} \rho^\beta ( x - \theta t \rho(x) y)
|\nabla u(x-\theta t \rho(x)y)| \phi(y)\,dy\,dx
\leq \frac{1}{1-\theta L} \lone{ \rho^\beta \nabla u}
$$
\end{corollary}

\begin{proof} (of Corollary~\ref{cor:lone-t-theta})
    Apply Lemma~\ref{lem:lonetheta} with
    $\delta(x) = \theta t \rho(x)$, and $f(x) = \rho^\beta(x) \nabla u(x)$.
\end{proof}

A key technical step is to observe that the above two corollaries hold
when $u$ is a function of bounded variation provided the left and right
terms are interpreted as their variation. In particular, consider $A
\subset \Re^d$ with finite perimeter and let $u$ be the
characteristic function of $A$. ($u(x) = 1$ if $x \in A$ and zero
otherwise). Then there exists a sequence of functions
$\{u_n\}_{n=1}^\infty \subset C^\infty(\Re^d)$ with $u_n \rightarrow
u$ in $\Lone$ for which \cite{EvansMeasure15}
\begin{equation} \label{eqn:Aapprox}
\abs{\boundary A}_\beta
= \lim_{n \rightarrow \infty} \int_\Omega  \rho^\beta|\nabla u_n|
\eqqcolon  \int_\Omega  \rho^\beta|\nabla u|.
\end{equation}
Interchanging $A$ and $\Omega \setminus A$ if necessary, it follows that
$\Phi(A)$ defined in Definition \ref{def:betaBdy} can be written as
$$
\Phi(A) 
= \frac{\int_\Omega \rho^\beta |\nabla u|}{\int_\Omega \rho^\alpha u}
= \lim_{n\to\infty}\frac{\int_\Omega \rho^\beta\abs{\grad u_n}}{\int_\Omega\rho^\alpha \abs{u_n}}.
$$

Now we are ready to prove our main Theorem, which is the Buser
inequality for probability densities stated in
Theorem~\ref{thm:buser_n}.
\begin{proof} (of Theorem \ref{thm:buser_n})

Fix $A \subset \RR^d$ with $|A|_\alpha \leq |1|_\alpha / 2$ and let
$u(x) = \chi_A(x)$ be the characteristic function of $A$. Setting 
$\ubar$ to be the weighted average of $u$,
\[
\ubar 
= \frac{\int \rho^\alpha u}{\int \rho^\alpha}
= \frac{\int_A \rho^\alpha}{\int \rho^\alpha}
= \frac{|A|_\alpha}{|1|_\alpha} \in [0,1/2],
\qquad \text{ then } \qquad
\int \rho^\alpha (u-\ubar) = 0,
\]
and
  \begin{equation}\label{eqn:blah}
\lone{\rho^\alpha(u-\ubar)} 
= \int \rho^\alpha |u-\ubar| = 2 |A|_\alpha (1-\ubar)
= 2 \int \rho^\alpha |u-\ubar|^2.
  \end{equation}

Since $|A|_\alpha = \lonea{u}$ and $1-\ubar \in [0,1/2]$ it follows that

\begin{equation} \label{eqn:PhiA}
  (1/2) \frac{\lone{\rho^\beta \nabla u}}{\lone{\rho^\alpha(u-\ubar)}} 
  \leq \Phi(A) = \frac{\lone{\rho^\beta \nabla u}}{\lone{\rho^\alpha u}}
  \leq \frac{\lone{\rho^\beta \nabla u}}{\lone{\rho^\alpha(u-\ubar)}}.
\end{equation}

In the calculations below we omit the limiting argument with smooth
approximations of $u$ in equation \eqnref{:Aapprox} which justify
formula involving $\nabla u$. In particular, only the $\Lone$ norm
of $\rho^\beta |\nabla u|$ will appear in the estimates since this
has meaning while the $\Ltwo$ norm is undefined.

Next, let $u_\theta$ be the mollification of (an extension of) $u$
given by equation \eqnref{:utheta}. Then $u_\theta(x)$ is a local
average average of $u$ so $u_\theta(x) \geq 0$, $\linf{u_\theta} \leq
1$ and $\linf{u-u_\theta} \leq 1$.  Letting $L$ denote the Lipschitz
constant of $\rho$, the parameter $\theta$ will to be chosen 
less than $1/(2L)$ so that that Lemma \ref{lem:lonetheta} is applicable
with constant $c = 1/2$.

The remainder of the proof constructs an upper bound on the numerator
  $\int_{\bbR^d} \rho^\gamma |\nabla u_\theta|^2$ of the Rayleigh quotient
for $u_\theta - \ubar_\theta$ by $\lone{\rho^\beta \nabla u}$ and to
  lower bound the denominator $\int_{\bbR^d} \rho^\alpha (u_\theta -
\ubar_\theta)^2$ by $\lone{\rho^\alpha (u-\ubar)}$. The conclusion
of the theorem then follows from equation \eqnref{:PhiA}.

\subsection{Upper Bounding the Numerator}
To bound the $L^2$ norm in
  the numerator of the Rayleigh quotient by the $L^1$ norm in the
  numerator of the expression for $\Phi(A)$ it is necessary to obtain
  uniform bound on $\rho(x) \nabla u_\theta(x)$. 
  
  \begin{lemma} \label{lem:rep1}
  Let $u$ be any function, and let $u_{\theta}$ be defined as in
  Equation~\ref{eqn:utheta}. Let $\rho: \Re^d \to \Re_{\geq 0}$ be an
  $L$-Lipschitz function.
  \begin{align}
  \linf{\rho (x) \nabla u_\theta(x)}
    \leq \linf{u} \frac{d(2+3L)}{\theta}
  \end{align}
  \end{lemma}

  \begin{proof}
  In order to prove this lemma, we first need to get a handle on
  $\nabla u_\theta(x)$, which is the gradient of $u$ after
  mollification by $\theta$.

  We take the
  the second representation of $u_\theta$ in equation \eqnref{:utheta}
  to get
  \begin{align}
  \nabla u_\theta(x) 
  = \int_{\bbR^d} u(z) \left\{
    \frac{-d}{\theta \rho(x)} \phi_{\theta \rho}(x-z) \nabla \rho
    + \frac{1}{(\theta \rho(x))^{d+1}} 
    \left( I 
      + \nabla \rho(x) \otimes \frac{x-z}{\theta \rho(x)} \right)
    \nabla \phi \left(\frac{x-z}{\theta \rho(x)} \right) \right\}
    \, dz,
  \end{align}
  which is a consequence of the multivariable chain rule. Here,
  $v \otimes u$ refers to the outer product of $v$ and $u$.

  Multiplying by $\rho$ gives:
  \begin{align}
  \rho (x) \nabla u_\theta(x) 
  = \int_{\bbR^d} u(z) \left\{
    \frac{-d}{\theta} \phi_{\theta \rho(x)}(x-z) \nabla
    \rho(x)
    + \frac{1}{(\theta^{d+1} \rho(x)^{d})} 
    \left( I 
      + \nabla \rho(x) \otimes \frac{x-z}{\theta \rho(x)} \right)
    \nabla \phi \left(\frac{x-z}{\theta \rho(x)} \right) \right\} \, dz.
  \end{align}
  Now, we can bound the above equation by carefully bounding each
  part. We note:
  \begin{align} \label{eq:Cphi-calculation}
    & \int_{\bbR^d} \frac{1}{(\theta^{d+1} \rho(x)^d)} \nabla
    \phi\left(\frac{x-z}{\theta \rho(x)} \right) dz
    \\
    & = \int_{\bbR^d} \frac{1}{(\theta^{d+1} \rho(x)^d)} \nabla
    \phi\left(\frac{-z}{\theta \rho(x)} \right) dz
    \\
     & = \frac{1}{\theta} \int_{\bbR^d} \nabla \phi(-y) dy
  \end{align}
  where the last step follows by a simple change of variable.
  Here, we note that $\nabla \phi(y)$ is a vector, and the
  integral is over $\bbR^d$, which is how we eliminated
  $\frac{1}{(\theta \rho(x))^d}$ from the expression.

  Next, we examine the term: 
  \begin{align}
    I + \nabla \rho(x) \otimes \frac{x-z}{\theta \rho(x)}
  \end{align}
  Here, we aim to bound the operator norm of this matrix. Here,
  we note that 
  \[ |x - z| \leq \theta \rho(x) \]
  when 
  \[
    \nabla \phi\left(\frac{x-z}{\theta \rho(x)}\right) \not= 0
  \]
  and thus, when the latter equation holds, we can say:

  \[
    \left| \frac{x-z}{\theta \rho(x)}\right| < 1.
  \]
  Since $|\nabla \rho(x) < L|$, we now have:
  \begin{align}\label{eqn:num-matrix-norm-bound}
    |I + \nabla \rho(x) \otimes \frac{x-z}{\theta \rho(x)}|_2 <
    3/2
  \end{align}
  Combining Equation~\ref{eqn:num-matrix-norm-bound} Equation~\ref{eq:Cphi-calculation} to show:
  \begin{align}
    & \left | \int_{\bbR^d}
    \frac{1}{(\theta^{d+1} \phi(x)^{d})} \left( I + \nabla \rho(x) \otimes \frac{x-z}{\theta \rho(x)}
    \right) \nabla \phi \left( \frac{x-z}{\theta \rho(x)}
    \right) dz \right| 
    \\
    & \leq \frac{(1 + L)}{\theta} \int_{\bbR^d} |\nabla\phi(y) | dy,
  \end{align}
  where $L = \linf{\nabla \rho}$ is the Lipschitz constant for $\rho$. 
  We note that Section~\ref{sec:dimension-dep} shows that 
  \begin{align}
    \int_{\bbR^d} | \nabla \phi(y) dy| \leq 2d.
  \end{align}
  and therefore:
  \begin{align}
    & \left | \int_{\bbR^d}
    \frac{1}{(\theta^{d+1} \phi(x)^{d})} \left( I + \nabla \rho(x) \otimes \frac{x-z}{\theta \rho(x)}
    \right) \nabla \phi \left( \frac{x-z}{\theta \rho(x)}
    \right) dz \right| 
    \\ \label{eq:rho-grad-utheta-2}
    & \leq \frac{2d(1+L)}{\theta}
  \end{align}

  Now we turn our attention to the first term, which is:
  \begin{align}
    \int_{\bbR^d} \frac{-d}{\theta} \phi_{\theta \rho(x)}(x-z)
    \nabla\rho(x) dz
  \end{align}
  We note that 
  \[ 
    \int_{\bbR^d}\left| \phi_{\theta \rho(x)}(x-z)\right| dz
    = 1
  \]
  by our definition of $\phi$ (which was defined when we defined
      $u_\theta$). Combining this
  with $|\nabla \rho(x)| < L$, we get:
  \begin{align}
    & \int_{\bbR^d} \left | \frac{-d}{\theta} \phi_{\theta \rho(x)}(x-z)
    \nabla\rho(x) \right| dz
    \\ \label{eq:rho-grad-utheta-1}
    & < \frac{dL}{\theta}
  \end{align}
  Therefore, 
  \begin{align}
  &  \nonumber \left| \int_{\bbR^d} 
    \frac{-d}{\theta \rho(x)} \phi_{\theta \rho}(x-z) \nabla \rho
    + \frac{1}{(\theta \rho(x))^{d+1}} 
    \left( I 
      + \nabla \rho(x) \otimes \frac{x-z}{\theta \rho(x)} \right)
    \nabla \phi \left(\frac{x-z}{\theta \rho(x)} \right) \, dz
    \right|
    \\
    & \nonumber \leq \frac{d}{\theta}( L + 2(1+L)) 
    \\
  & = \frac{d(2+3L)}{\theta}.
  \label{eqn:}
  \end{align}
  where the first inequality comes from combining
  Equations~\ref{eq:rho-grad-utheta-2}
  and~\ref{eq:rho-grad-utheta-1}.

  This allows us to bound $\linf{\rho(x) \nabla u_{\theta}(x)}$:
  \begin{align}
  & \linf{\rho (x) \nabla u_\theta(x)}
  \\
  & = \linf{ \left| \int_{\bbR^d} u(z) \left\{
    \frac{-d}{\theta} \phi_{\theta \rho(x)}(x-z) \nabla
    \rho(x)
    + \frac{1}{(\theta^{d+1} \rho(x)^{d})} 
    \left( I 
      + \nabla \rho(x) \otimes \frac{x-z}{\theta \rho(x)} \right)
    \nabla \phi \left(\frac{x-z}{\theta \rho(x)} \right) \right\}
  \, dz \right| }
  \\
  & \leq \linf{u} \linf{ \int_{\bbR^d} \left| 
    \frac{-d}{\theta} \phi_{\theta \rho(x)}(x-z) \nabla
    \rho(x)
    + \frac{1}{(\theta^{d+1} \rho(x)^{d})} 
    \left( I 
      + \nabla \rho(x) \otimes \frac{x-z}{\theta \rho(x)} \right)
    \nabla \phi \left(\frac{x-z}{\theta \rho(x)} \right)
  \right| \, dz}
  \\
  & \leq \linf{u} \frac{d(2+3L)}{\theta}
  \end{align}
  where we make use of the fact that
    $\linf{ab} < \linf{a}\lone{b}.$
  This completes our proof.
  \end{proof}

  Next, we want an $L_1$ bound on $\rho^{\beta}(x) \nabla
  u_{\theta}(x)$.

  \begin{lemma} \label{lem:rep2}
  Let $u$ be any function, and let $u_{\theta}$ be defined as in
  Equation~\ref{eqn:utheta}. Let $\rho:\Re^d \to \Re_{\geq 0}$ be an
  $L$-Lipschitz function, and
  let $\theta L < 1/2$.

  Then:
  \begin{align}
      \lone{\rho^\beta(x) \nabla u_{\theta}(x)} \leq C_{\beta}
      \lone{\rho^\beta(x) \nabla u(x)}
  \end{align}
  \end{lemma}
  \begin{proof}
  First, we take the gradient 
  first representation of $u_\theta$
  in equation \eqnref{:utheta}. Using the chain rule gives us an
  alternate form for $\nabla u_\theta(x)$:

  \begin{align}
  \nabla u_\theta(x) 
  = \int_{\Re^d} 
  (I - \theta \nabla \rho \otimes y) \nabla u(x-\theta \rho y) \phi(y) \, dy,
  \end{align}
  so
  \begin{align}\label{eqn:rhoNablaUtheta}
  \rho^\beta(x) \nabla u_\theta(x) 
  = \int_{\Re^d} 
  (I - \theta \nabla \rho \otimes y)
  \frac{\rho^\beta(x)}{\rho^\beta(x-\theta \rho y)}
  \rho^\beta(x-\theta \rho y) \nabla u(x-\theta \rho y) \phi(y) \, dy.
  \end{align}

  The ratio in the integrand is bounded using the Lipschitz assumption
  on $\rho$ (and $|y| \leq 1$),
  \begin{equation} \label{eqn:rhoRatio}
    \frac{\rho(x)}{\rho(x-\theta \rho y)}
    \leq \frac{\rho(x)}{\rho(x) - L \theta \rho(x)}
    = \frac{1}{1 - L \theta} \leq 2,
    \qquad \text{ when } \theta < 1 / (2L).
  \end{equation}
  Note that 
  \begin{align} \label{eqn:matrix-norm} 
  \left\| I - \theta \nabla \rho \otimes y \right\|_2 \leq 3/2
  \end{align}
  where $\|M\|_2$ represents the $\ell^2$ matrix norm of $M$. This is
  because $|\nabla \rho(x) | \leq L$, and
  $\theta L < 1/2$, and $|y| \leq
  1$ every time $\phi(y) \not= 0$, and thus 
  \[ 
   \frac{I}{2} \preceq  I - \theta \nabla \rho \otimes y
   \preceq \frac{3I}{2}.
    \]
  
 Therefore, we can now apply Corollary~\ref{cor:lonetheta}
 to Equation~\eqnref{:rhoNablaUtheta} to show:
  \begin{align}
  \nonumber 
  & \lone{\rho^\beta(x) \nabla u_\theta(x)} 
  \\  \nonumber
  & \leq \|I - \theta \nabla \rho(x) \otimes y\|_2
  \cdot \max_x\left(\frac{\rho(x)}{\rho(x - \theta \rho
        y)}\right) \cdot
  \int_{\bbR^d} \left| \int_{\bbR^d} \rho^{\beta}(x - \theta \rho y) \nabla u(x - \theta \rho y)
    \phi(y) dy \right|
  \\ \nonumber 
  & \leq 3 \cdot 2^{\beta - 1} \int_{\bbR^d} \rho^\beta(x -
      \theta \rho(x) y) \nabla u (x - \theta \rho (x) y 
  \\ \nonumber
  & 
  \leq 3 \cdot 2^\beta \lone{\rho^\beta \nabla u},
  \qquad \text{ when } \theta < 1 / (2L).
  \end{align}
  here, the first inequality comes from the equation $\lone{abc}
  \leq \linf{a}\linf{b}\lone{c}$, the second inequality comes
  from Equations~\ref{eqn:rhoRatio} and~\ref{eqn:matrix-norm}, and the
  third inequality comes
  from Corollary~\ref{cor:lonetheta} assuming $\theta L \leq
  1/2$. 
  \end{proof}

  \begin{lemma}\label{lem:num}
  For any $L$-Lipschitz distribution $\rho$, any function $u$, and any $\theta$ such that $\theta L <
  1/2$:

  \begin{align}
  \int_{\bbR^d} \rho^\gamma |\nabla u_\theta|^2
  \leq  C_\beta \linf{\rho^{\gamma-\beta-1}} \frac{d(2+3L)}{\theta} 
  \linf{u} \lone{\rho^\beta \nabla u},
  \end{align}
  \end{lemma}

  \begin{proof}
  Combining the two estimates from Lemma~\ref{lem:rep1}
  and~\ref{lem:rep2} gives an upper bound for the Rayleigh
  quotient 
  \begin{align}
  \int_{\bbR^d} \rho^\gamma |\nabla u_\theta|^2
  = \int_{\bbR^d} \rho^{\gamma-\beta-1} \,
  \rho |\nabla u_\theta| \, \rho^\beta |\nabla u_\theta|
  \leq 3 \cdot 2^{\beta + 1} \linf{\rho^{\gamma-\beta-1}} \frac{d(2+3L)}{\theta} 
  \linf{u} \lone{\rho^\beta \nabla u},
  \end{align}
  \end{proof}

  We note that in the case where $\gamma = \beta + 1$, and if $u$ is a step
  function, the expression would
  simplify to:
  \[ \int_{\bbR^d}
  |\rho^{\gamma} |\nabla u_\theta|^2
  \leq 3 \cdot 2^{\beta + 1} \frac{d(2+3L)}{\theta} 
  \linf{u} \lone{\rho^\beta \nabla u},
  \]

\subsection{Lower Bound on the Denominator}
Let $\ubar$ and
  $\ubar_\theta$ be the $\rho^\alpha$--weighted averages of $u$ and
  $u_\theta$. For any function $f$, let $\ltwoa{f}$ denote the $L^2$ norm of $\rho^\alpha
  f$, and let $\lone{f}$ denote the $L^1$ norm of $\rho^\alpha f$ for
  functions $f$ where these two quantities are well defined.
  Our core lemma is a bound on $\ltwoa{u_\theta-\ubar_\theta}$ in
  terms of $l_1$ and weighted $l_1$ norms of $\nabla u$ and $u -
  \ubar$ respectively.
  \begin{lemma}\label{lem:denom}
  Let $\rho$ be an $L$-Lipschitz function $\rho: \Re^d \to
  \Re_{\geq 0}$, and let $\theta$ be such
  that $\theta L < 1/2$.
  Let $u$ be an indicator function of a set $A$ with finite
  $\beta$-perimeter. Let $\ubar$ be defined as $\ubar(x):=u(x) - \int u(y) dy$
    $u_\theta$ be defined as in Equation~\ref{eqn:utheta}, and
    $\ubar_\theta$ be defined as $\ubar_\theta(x) := u_\theta(x)
    - \int
    u_\theta(y) dy$.
    Then:
  \begin{align}
  \ltwoa{u_\theta - \ubar_\theta}^2
  \geq (1/4) \lonea{u-\ubar} 
  - C(\beta) \theta \linf{\rho^{\alpha+1-\beta}} \lone{\rho^\beta \nabla u},
  \qquad \text{ when } \theta < 1 / (2L).
  \end{align}
  \end{lemma}

  Note that when $\alpha + 1 = \beta$, as is true when $(\alpha,
      \beta, \gamma) = (1,2,3)$, the inequality in
  Lemma~\ref{lem:denom} becomes:

  \[
  \ltwoa{u_\theta - \ubar_\theta}^2
  \geq (1/4) \lonea{u-\ubar} 
  - C(\beta) \theta \lone{\rho^\beta \nabla u},
  \qquad \text{ when } \theta < 1 / (2L).
  \]
  The estimate in Lemma~\ref{lem:denom} will be combined with the
  estimate in Lemma~\ref{lem:num} to prove
  Theorem~\ref{thm:buser_n}
  in Section~\ref{sec:rayleigh-bound}.


  \begin{proof} The key to this proof is to upper bound the
  quantity $\ltwoa{u_\theta - \ubar_\theta}$ with the expression
  appearing in Corollary~\ref{cor:lone-t-theta}. We will do so by
  a series of inequalities, application of the fundamental
  theorem of calculus, and more.

  Using the property that subtracting the average from a
  function reduces the $L^2$ norm it follows that
  \begin{align}
  \nonumber
  & \ltwoa{u_\theta - \ubar_\theta}
  \\
    \nonumber
  & \geq \ltwoa{u-\ubar} - \ltwoa{u_\theta - u - (\ubar_\theta-\ubar)}
  \\
    \nonumber
  &\geq \ltwoa{u-\ubar} - \ltwoa{u_\theta - u}.
  \end{align}
  If $a \geq b-c$ then $a^2 \geq b^2/2 - c^2$, so a lower bound for
  the denominator of the Rayleigh quotient
  \begin{align} \label{eqn:utmu}
    & \ltwoa{u_\theta - \ubar_\theta}^2
    \\
    \nonumber
    &\geq (1/2) \ltwoa{u-\ubar}^2 - \ltwoa{u_\theta - u}^2
    \\
    \nonumber
    & \geq (1/4) \lonea{u-\ubar} - \lonea{u_\theta - u},
  \end{align}
  where the identity $\ltwoa{u-\ubar}^2 = \lonea{u-\ubar}/2$ from
  Equation~\ref{eqn:blah}, and the bound
  $\linf{u_\theta - u} \leq 1$, were used in the last step.

  It remains to estimate the difference $\lonea{u_\theta - u}$. To
  do this, we use the multivariable fundamental
  theorem of calculus to write
  \begin{eqnarray*}
    u_\theta(x) - u(x)
    &=& \int (u(x - \theta \rho y) - u(x)) \phi(y) \, dy \\
    &=& \int \! \int_0^1
    -\theta \rho(x) \nabla u(x - t \theta \rho(x) y).y \phi(y) \, dt \, dy \\
    &=& \int \! \int_0^1
    \frac{-\theta \rho(x)}{\rho^\beta(x-t\theta \rho(x) y)} 
    \rho^\beta(x - t\theta \rho(x) y) \nabla u(x - t \theta
        \rho(x) y).y \phi(y) \, dt \, dy,
  \end{eqnarray*}
  where the first and second equalities came from application of
  the multivariable fundamental theorem of calculus, and the last
  equation is straightforward. This tells us that:
  \begin{align}
  \notag
  & \rho^{\alpha}(x) ( u_\theta \rho(x) - u(x) )
  \\
  \notag
  & = \int \! \int_0^1
  \frac{-\theta \rho^{\alpha+1}(x)}{\rho^\beta(x-t\theta \rho y)} 
  \rho^\beta(x - t\theta \rho(x) y) 
  \nabla u(x - t \theta \rho(x) y).y \phi(y) \, dt \, dy.
  \\
    \label{eqn:utheta-l1-alpha}
  &= \int \! \int_0^1
  \frac{\rho^\beta(x)}{\rho^\beta (x- t \theta \rho(x)y)}
  \frac{-\theta \rho^{\alpha+1}(x)}{\rho^\beta(x)}
  \rho^{\beta}(x - t \theta \rho(x) y )
  \nabla u(x - t \theta \rho(x) y).y \phi(y) \, dt \, dy.
  \end{align}
  Equation \eqnref{:rhoRatio} bounds the ratio $\rho(x)/ \rho(x -
    t\theta \rho(x) y)$ as less than $2$ when $\theta L < 1/2$,
  so Equation~\eqnref{:utheta-l1-alpha} is always less than or
  equal to:
  \begin{align}
  \int \! \int_0^1
  2^{\beta}
  \frac{-\theta \rho^{\alpha+1}(x)}{\rho^\beta(x)}
  \rho^{\beta}(x - t \theta \rho(x) y )
  \nabla u(x - t \theta \rho(x) y).y \phi(y) \, dy \, dt.
  \end{align}

  An application of Corollary \ref{cor:lone-t-theta} then shows
  \begin{align}
  & \int \! \int_0^1
  2^{\beta}
  \frac{-\theta \rho^{\alpha+1}(x)}{\rho^\beta(x)}
  \rho^{\beta}(x - t \theta \rho(x) y )
  \nabla u(x - t \theta \rho(x) y).y \phi(y) \, dy \, dt.
  \\
  & \leq \int \! \int_0^1
  2^{\beta}
  \frac{-\theta \rho^{\alpha+1}(x)}{\rho^\beta(x)}
  \rho^{\beta}(x - t \theta \rho(x) y ) \ 
  \left|\nabla u(x - t \theta \rho(x) y) \phi(y)\right| \, dy \, dt.
  \\
  &
  \leq 2^{\beta+1} \linf{\rho^{\alpha+1-\beta}} \theta
  \int \! \int_0^1
  \rho^{\beta}(x - t \theta \rho(x) y ) \ 
  \left|\nabla u(x - t \theta \rho(x) y) \phi(y)\right| \, dy \, dt.
  \qquad \text{ when } \theta < 1 / (2L).
  \\
  &
  \leq 2^{\beta+1} \linf{\rho^{\alpha+1-\beta}} \theta \lone{\rho^\beta \nabla u}
  \end{align}
  where the last inequality follows from
  Corollary~\ref{cor:lone-t-theta}.

  Using this estimate in \eqnref{:utmu} gives a lower bound on the
  denominator of the Rayleigh quotient,
  \begin{align}
  \ltwoa{u_\theta - \ubar_\theta}^2
  \geq (1/4) \lonea{u-\ubar} 
  - 2^{\beta+1} \theta \linf{\rho^{\alpha+1-\beta}} \lone{\rho^\beta \nabla u},
  \qquad \text{ when } \theta < 1 / (2L).
  \end{align}
  as desired.
  \end{proof}

  \subsection{Bounding the Rayleigh
    Quotient (Proof of Theorem~\ref{thm:buser_n})}\label{sec:rayleigh-bound}
  Combining Lemmas~\ref{lem:num} and Lemmas~\ref{lem:denom}
  provides an upper bound for the Rayleigh quotient of $u_\theta - \ubar_\theta$,
  \begin{eqnarray*}
    \lambda_2 
    &\leq& \frac{\int_{\bbR^d} \rho^\gamma |\nabla u_\theta|^2}
    {\int_{\bbR^d} \rho^\alpha (u_\theta - \ubar_\theta)^2} \\
    &\leq&  \frac{d \cdot 3 \cdot 2^{\beta}}{\theta}
    \frac{\linf{\rho^{\gamma-\beta-1}} (2+3L) \lone{\rho^\beta \nabla u}}
    {\lonea{u-\ubar} 
      - 2^{\beta+1} \theta \linf{\rho^{\alpha+1-\beta}} \lone{\rho^\beta \nabla u}} \\
    &\leq&  \frac{d \cdot 3 \cdot 2^{\beta}}{\theta}
    \frac{\linf{\rho^{\gamma-\beta-1}} (2+3L)}
    {1 - 2^{\beta + 1} \theta \linf{\rho^{\alpha+1-\beta}} \Phi(A)} \Phi(A).
  \end{eqnarray*}
  Selecting $\theta = (1/2)
  \min\left(1/\left(2^{\beta+1}\linf{\rho^{\alpha+1-\beta}}
        \Phi(A)\right), 1/L
  \right)$ shows
  \[
  \lambda_2 \leq 2 d \cdot 3 \cdot 2^{\beta}
\linf{\rho^{\gamma-\beta-1}}(2+3L) 
  \max\left(L \Phi(A), 2^{\beta+1}\linf{\rho^{\alpha+1-\beta}} \Phi(A)^2 \vph\right).
  \]
  When $\gamma = (1,2,3)$, this simplifies into:
  \[
  \lambda_2 \leq 12(2+3L) d
  \max\left(L \Phi(A), 8 \Phi(A)^2 \vph\right).
  \]
  We note that, via the work shown in
  Section~\ref{sec:scaling},  we can strengthen our inequality
  to:
  \[
  \lambda_2 \leq 24 d
  \max\left(L \Phi(A), 8 \Phi(A)^2 \vph\right).
  \]

  \qedhere


\subsection{Gradient of Mollifier}\label{sec:dimension-dep}
Let $\phi$ be a standard mollifier i.e. $\phi\in C_c^\infty(\RR^d)$
is a function from $\RR^d\to [0,\infty)$ satisfying $\int_{\RR^d}
  \phi\,dx = 1$ and $\supp(\phi)\subseteq B(0,1)$.  We will define
  $\phi$ by its profile. Namely, let $\phihat(r):[0,\infty)
    \rightarrow [0,1]$ be a fixed monotone decreasing profile with
    $\phihat(0)=1$, $0 < \phihat(r) < 1$ for $0 < r < 1$, and
    $\phihat(r) = 0$ for $r \geq 1$. Then define $\phi:\Re^d
    \rightarrow \Re$ by $\phi(x) = c \phihat(|x|)$ with $c > 0$ chosen
    so that $\int_{\Re^d} \phi(x) \, dx = 1$; that is,

\[
1 = \int_{\Re^d} \phi(x) \, dx
= c |S^{d-1}| \int_0^1 \phihat(r) r^{d-1} \, dr
\qquad \Rightarrow \qquad
c = \frac{1}{|S^{d-1}| \int_0^1 \phihat(r) r^{d-1} \, dr},
\]

where $|S^{d-1}|$ is the $(d-1)$--area of the unit sphere in $\Re^d$.
We claim the $L_1$ norm of the gradient of $\nabla \phi(x)$ is linear in $d$.
\begin{lemma}\label{lem:molli}
  \[
  \int_{\Re^d} |\nabla \phi(x)| \, dx   
\leq (d-1) \left( \frac{d 2^d}{\phihat(1/2)} \right)^{1/(d-1)}
\stackrel{d \rightarrow \infty}{\longrightarrow} 2(d-1).
\]
For the classic mollifier $\phihat(r)=\exp(-1/(1-r^2))$ we get
  \[
  \int_{\Re^d} |\nabla \phi(x)| \, dx  \leq 2d.
\]
\end{lemma}

From the formula $\nabla \phi(x) = c \phihat'(|x|) (x/|x|)$ we compute
\begin{eqnarray*}
\int_{\Re^d} |\nabla \phi(x)| \, dx
&=& c |S^{d-1}| \int_0^1 |\phihat'(r)| r^{d-1} \, dr \\
&=& c |S^{d-1}| \int_0^1 -\phihat'(r) r^{d-1} \, dr \\
&=& c |S^{d-1}| \int_0^1 \phihat(r) (d-1) r^{d-2} \, dr \\
&=& (d-1) \frac{\int_0^1 \phihat(r) r^{d-2} \, dr}
{\int_0^1 \phihat(r) r^{d-1} \, dr}.
\end{eqnarray*}
To estimate the numerator use Holder's inequality: for
$1 \leq s, s' \leq \infty$ with $1/s + 1/s' = 1$ 
$$
\int f g 
\leq \left(\int |f|^s \right)^{1/s} \left(\int |g|^{s'} \right)^{1/s'}.
$$
Set $s = (d-1)/(d-2)$ and $s' = d-1$ to get
$$
\int_0^1 \phihat(r) r^{d-2} \, dr
= \int_0^1 \phihat(r)^{1/s} r^{d-2} \times \phihat(r)^{1/s'}  \, dr
\leq \left(\int_0^1 \phihat(r) r^{d-1} \, dr \right)^{1/s}
\left(\int_0^1 \phihat(r) \, dr \right)^{1/s'}.
$$
It follows that
\begin{equation}\label{eq:molli0}
\int_{\Re^d} |\nabla \phi(x)| \, dx
\leq (d-1) \left( \frac{\int_0^1 \phihat(r) \, dr}
{\int_0^1 \phihat(r) r^{d-1} \, dr} \right)^{1/(d-1)}.
\end{equation}
Since $0 \leq \phihat(r) \leq 1$ we can bound the numerator
by $1$, and since $\phihat(r)$ is monotone decreasing we have
$\phihat(r) \geq \phihat(1/2)$ on $(0,1/2)$, so 
\begin{equation}\label{eq:molli}
\int_{\Re^d} |\nabla \phi(x)| \, dx
\leq (d-1)
\left( \frac{1}{\phihat(1/2) \int_0^{1/2} r^{d-1} \, dr} \right)^{1/(d-1)} 
\leq (d-1) \left( \frac{d 2^d}{\phihat(1/2)} \right)^{1/(d-1)}
\stackrel{d \rightarrow \infty}{\longrightarrow} 2(d-1).
\end{equation}

It will be convenient to write equation~\ref{eq:molli} as a simple
inequality. Observer that
\[
 \left( \frac{d 2^d}{\phihat(1/2)} \right)^{1/(d-1)}
\]
is monotone decreasing. We now pick the classic
$\phihat(r) = \exp(-1/(1-r^2))$ we have $\phihat(1/2) \geq 1/4$ and if
$d \geq 5$ the right hand side of equation (\ref{eq:molli} is bounded
by $2d$. If $d < 5$ explicit computations of the integrals shows the
right hand side of equation (\ref{eq:molli0} is bounded by $2d$.







\end{proof}

