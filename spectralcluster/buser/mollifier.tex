
\subsection{Gradient of Mollifier}\label{sec:dimension-dep}
Let $\phi$ be a standard mollifier i.e. $\phi\in C_c^\infty(\RR^d)$
is a function from $\RR^d\to [0,\infty)$ satisfying $\int_{\RR^d}
  \phi\,dx = 1$ and $\supp(\phi)\subseteq B(0,1)$.  We will define
  $\phi$ by its profile. Namely, let $\phihat(r):[0,\infty)
    \rightarrow [0,1]$ be a fixed monotone decreasing profile with
    $\phihat(0)=1$, $0 < \phihat(r) < 1$ for $0 < r < 1$, and
    $\phihat(r) = 0$ for $r \geq 1$. Then define $\phi:\Re^d
    \rightarrow \Re$ by $\phi(x) = c \phihat(|x|)$ with $c > 0$ chosen
    so that $\int_{\Re^d} \phi(x) \, dx = 1$; that is,

\[
1 = \int_{\Re^d} \phi(x) \, dx
= c |S^{d-1}| \int_0^1 \phihat(r) r^{d-1} \, dr
\qquad \Rightarrow \qquad
c = \frac{1}{|S^{d-1}| \int_0^1 \phihat(r) r^{d-1} \, dr},
\]

where $|S^{d-1}|$ is the $(d-1)$--area of the unit sphere in $\Re^d$.
We claim the $L_1$ norm of the gradient of $\nabla \phi(x)$ is linear in $d$.
\begin{lemma}\label{lem:molli}
  \[
  \int_{\Re^d} |\nabla \phi(x)| \, dx   
\leq (d-1) \left( \frac{d 2^d}{\phihat(1/2)} \right)^{1/(d-1)}
\stackrel{d \rightarrow \infty}{\longrightarrow} 2(d-1).
\]
For the classic mollifier $\phihat(r)=\exp(-1/(1-r^2))$ we get
  \[
  \int_{\Re^d} |\nabla \phi(x)| \, dx  \leq 2d.
\]
\end{lemma}

From the formula $\nabla \phi(x) = c \phihat'(|x|) (x/|x|)$ we compute
\begin{eqnarray*}
\int_{\Re^d} |\nabla \phi(x)| \, dx
&=& c |S^{d-1}| \int_0^1 |\phihat'(r)| r^{d-1} \, dr \\
&=& c |S^{d-1}| \int_0^1 -\phihat'(r) r^{d-1} \, dr \\
&=& c |S^{d-1}| \int_0^1 \phihat(r) (d-1) r^{d-2} \, dr \\
&=& (d-1) \frac{\int_0^1 \phihat(r) r^{d-2} \, dr}
{\int_0^1 \phihat(r) r^{d-1} \, dr}.
\end{eqnarray*}
To estimate the numerator use Holder's inequality: for
$1 \leq s, s' \leq \infty$ with $1/s + 1/s' = 1$ 
$$
\int f g 
\leq \left(\int |f|^s \right)^{1/s} \left(\int |g|^{s'} \right)^{1/s'}.
$$
Set $s = (d-1)/(d-2)$ and $s' = d-1$ to get
$$
\int_0^1 \phihat(r) r^{d-2} \, dr
= \int_0^1 \phihat(r)^{1/s} r^{d-2} \times \phihat(r)^{1/s'}  \, dr
\leq \left(\int_0^1 \phihat(r) r^{d-1} \, dr \right)^{1/s}
\left(\int_0^1 \phihat(r) \, dr \right)^{1/s'}.
$$
It follows that
\begin{equation}\label{eq:molli0}
\int_{\Re^d} |\nabla \phi(x)| \, dx
\leq (d-1) \left( \frac{\int_0^1 \phihat(r) \, dr}
{\int_0^1 \phihat(r) r^{d-1} \, dr} \right)^{1/(d-1)}.
\end{equation}
Since $0 \leq \phihat(r) \leq 1$ we can bound the numerator
by $1$, and since $\phihat(r)$ is monotone decreasing we have
$\phihat(r) \geq \phihat(1/2)$ on $(0,1/2)$, so 
\begin{equation}\label{eq:molli}
\int_{\Re^d} |\nabla \phi(x)| \, dx
\leq (d-1)
\left( \frac{1}{\phihat(1/2) \int_0^{1/2} r^{d-1} \, dr} \right)^{1/(d-1)} 
\leq (d-1) \left( \frac{d 2^d}{\phihat(1/2)} \right)^{1/(d-1)}
\stackrel{d \rightarrow \infty}{\longrightarrow} 2(d-1).
\end{equation}

It will be convenient to write equation~\ref{eq:molli} as a simple
inequality. Observer that
\[
 \left( \frac{d 2^d}{\phihat(1/2)} \right)^{1/(d-1)}
\]
is monotone decreasing. We now pick the classic
$\phihat(r) = \exp(-1/(1-r^2))$ we have $\phihat(1/2) \geq 1/4$ and if
$d \geq 5$ the right hand side of equation (\ref{eq:molli} is bounded
by $2d$. If $d < 5$ explicit computations of the integrals shows the
right hand side of equation (\ref{eq:molli0} is bounded by $2d$.






