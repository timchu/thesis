\subsection{Lower Bound on the Denominator}
Let $\ubar$ and
  $\ubar_\theta$ be the $\rho^\alpha$--weighted averages of $u$ and
  $u_\theta$. For any function $f$, let $\ltwoa{f}$ denote the $L^2$ norm of $\rho^\alpha
  f$, and let $\lone{f}$ denote the $L^1$ norm of $\rho^\alpha f$ for
  functions $f$ where these two quantities are well defined.
  Our core lemma is a bound on $\ltwoa{u_\theta-\ubar_\theta}$ in
  terms of $l_1$ and weighted $l_1$ norms of $\nabla u$ and $u -
  \ubar$ respectively.
  \begin{lemma}\label{lem:denom}
  Let $\rho$ be an $L$-Lipschitz function $\rho: \Re^d \to
  \Re_{\geq 0}$, and let $\theta$ be such
  that $\theta L < 1/2$.
  Let $u$ be an indicator function of a set $A$ with finite
  $\beta$-perimeter. Let $\ubar$ be defined as $\ubar(x):=u(x) - \int u(y) dy$
    $u_\theta$ be defined as in Equation~\ref{eqn:utheta}, and
    $\ubar_\theta$ be defined as $\ubar_\theta(x) := u_\theta(x)
    - \int
    u_\theta(y) dy$.
    Then:
  \begin{align}
  \ltwoa{u_\theta - \ubar_\theta}^2
  \geq (1/4) \lonea{u-\ubar} 
  - C(\beta) \theta \linf{\rho^{\alpha+1-\beta}} \lone{\rho^\beta \nabla u},
  \qquad \text{ when } \theta < 1 / (2L).
  \end{align}
  \end{lemma}

  Note that when $\alpha + 1 = \beta$, as is true when $(\alpha,
      \beta, \gamma) = (1,2,3)$, the inequality in
  Lemma~\ref{lem:denom} becomes:

  \[
  \ltwoa{u_\theta - \ubar_\theta}^2
  \geq (1/4) \lonea{u-\ubar} 
  - C(\beta) \theta \lone{\rho^\beta \nabla u},
  \qquad \text{ when } \theta < 1 / (2L).
  \]
  The estimate in Lemma~\ref{lem:denom} will be combined with the
  estimate in Lemma~\ref{lem:num} to prove
  Theorem~\ref{thm:buser_n}
  in Section~\ref{sec:rayleigh-bound}.


  \begin{proof} The key to this proof is to upper bound the
  quantity $\ltwoa{u_\theta - \ubar_\theta}$ with the expression
  appearing in Corollary~\ref{cor:lone-t-theta}. We will do so by
  a series of inequalities, application of the fundamental
  theorem of calculus, and more.

  Using the property that subtracting the average from a
  function reduces the $L^2$ norm it follows that
  \begin{align}
  \nonumber
  & \ltwoa{u_\theta - \ubar_\theta}
  \\
    \nonumber
  & \geq \ltwoa{u-\ubar} - \ltwoa{u_\theta - u - (\ubar_\theta-\ubar)}
  \\
    \nonumber
  &\geq \ltwoa{u-\ubar} - \ltwoa{u_\theta - u}.
  \end{align}
  If $a \geq b-c$ then $a^2 \geq b^2/2 - c^2$, so a lower bound for
  the denominator of the Rayleigh quotient
  \begin{align} \label{eqn:utmu}
    & \ltwoa{u_\theta - \ubar_\theta}^2
    \\
    \nonumber
    &\geq (1/2) \ltwoa{u-\ubar}^2 - \ltwoa{u_\theta - u}^2
    \\
    \nonumber
    & \geq (1/4) \lonea{u-\ubar} - \lonea{u_\theta - u},
  \end{align}
  where the identity $\ltwoa{u-\ubar}^2 = \lonea{u-\ubar}/2$ from
  Equation~\ref{eqn:blah}, and the bound
  $\linf{u_\theta - u} \leq 1$, were used in the last step.

  It remains to estimate the difference $\lonea{u_\theta - u}$. To
  do this, we use the multivariable fundamental
  theorem of calculus to write
  \begin{eqnarray*}
    u_\theta(x) - u(x)
    &=& \int (u(x - \theta \rho y) - u(x)) \phi(y) \, dy \\
    &=& \int \! \int_0^1
    -\theta \rho(x) \nabla u(x - t \theta \rho(x) y).y \phi(y) \, dt \, dy \\
    &=& \int \! \int_0^1
    \frac{-\theta \rho(x)}{\rho^\beta(x-t\theta \rho(x) y)} 
    \rho^\beta(x - t\theta \rho(x) y) \nabla u(x - t \theta
        \rho(x) y).y \phi(y) \, dt \, dy,
  \end{eqnarray*}
  where the first and second equalities came from application of
  the multivariable fundamental theorem of calculus, and the last
  equation is straightforward. This tells us that:
  \begin{align}
  \notag
  & \rho^{\alpha}(x) ( u_\theta \rho(x) - u(x) )
  \\
  \notag
  & = \int \! \int_0^1
  \frac{-\theta \rho^{\alpha+1}(x)}{\rho^\beta(x-t\theta \rho y)} 
  \rho^\beta(x - t\theta \rho(x) y) 
  \nabla u(x - t \theta \rho(x) y).y \phi(y) \, dt \, dy.
  \\
    \label{eqn:utheta-l1-alpha}
  &= \int \! \int_0^1
  \frac{\rho^\beta(x)}{\rho^\beta (x- t \theta \rho(x)y)}
  \frac{-\theta \rho^{\alpha+1}(x)}{\rho^\beta(x)}
  \rho^{\beta}(x - t \theta \rho(x) y )
  \nabla u(x - t \theta \rho(x) y).y \phi(y) \, dt \, dy.
  \end{align}
  Equation \eqnref{:rhoRatio} bounds the ratio $\rho(x)/ \rho(x -
    t\theta \rho(x) y)$ as less than $2$ when $\theta L < 1/2$,
  so Equation~\eqnref{:utheta-l1-alpha} is always less than or
  equal to:
  \begin{align}
  \int \! \int_0^1
  2^{\beta}
  \frac{-\theta \rho^{\alpha+1}(x)}{\rho^\beta(x)}
  \rho^{\beta}(x - t \theta \rho(x) y )
  \nabla u(x - t \theta \rho(x) y).y \phi(y) \, dy \, dt.
  \end{align}

  An application of Corollary \ref{cor:lone-t-theta} then shows
  \begin{align}
  & \int \! \int_0^1
  2^{\beta}
  \frac{-\theta \rho^{\alpha+1}(x)}{\rho^\beta(x)}
  \rho^{\beta}(x - t \theta \rho(x) y )
  \nabla u(x - t \theta \rho(x) y).y \phi(y) \, dy \, dt.
  \\
  & \leq \int \! \int_0^1
  2^{\beta}
  \frac{-\theta \rho^{\alpha+1}(x)}{\rho^\beta(x)}
  \rho^{\beta}(x - t \theta \rho(x) y ) \ 
  \left|\nabla u(x - t \theta \rho(x) y) \phi(y)\right| \, dy \, dt.
  \\
  &
  \leq 2^{\beta+1} \linf{\rho^{\alpha+1-\beta}} \theta
  \int \! \int_0^1
  \rho^{\beta}(x - t \theta \rho(x) y ) \ 
  \left|\nabla u(x - t \theta \rho(x) y) \phi(y)\right| \, dy \, dt.
  \qquad \text{ when } \theta < 1 / (2L).
  \\
  &
  \leq 2^{\beta+1} \linf{\rho^{\alpha+1-\beta}} \theta \lone{\rho^\beta \nabla u}
  \end{align}
  where the last inequality follows from
  Corollary~\ref{cor:lone-t-theta}.

  Using this estimate in \eqnref{:utmu} gives a lower bound on the
  denominator of the Rayleigh quotient,
  \begin{align}
  \ltwoa{u_\theta - \ubar_\theta}^2
  \geq (1/4) \lonea{u-\ubar} 
  - 2^{\beta+1} \theta \linf{\rho^{\alpha+1-\beta}} \lone{\rho^\beta \nabla u},
  \qquad \text{ when } \theta < 1 / (2L).
  \end{align}
  as desired.
  \end{proof}

  \subsection{Bounding the Rayleigh
    Quotient (Proof of Theorem~\ref{thm:buser_n})}\label{sec:rayleigh-bound}
  Combining Lemmas~\ref{lem:num} and Lemmas~\ref{lem:denom}
  provides an upper bound for the Rayleigh quotient of $u_\theta - \ubar_\theta$,
  \begin{eqnarray*}
    \lambda_2 
    &\leq& \frac{\int_{\bbR^d} \rho^\gamma |\nabla u_\theta|^2}
    {\int_{\bbR^d} \rho^\alpha (u_\theta - \ubar_\theta)^2} \\
    &\leq&  \frac{d \cdot 3 \cdot 2^{\beta}}{\theta}
    \frac{\linf{\rho^{\gamma-\beta-1}} (2+3L) \lone{\rho^\beta \nabla u}}
    {\lonea{u-\ubar} 
      - 2^{\beta+1} \theta \linf{\rho^{\alpha+1-\beta}} \lone{\rho^\beta \nabla u}} \\
    &\leq&  \frac{d \cdot 3 \cdot 2^{\beta}}{\theta}
    \frac{\linf{\rho^{\gamma-\beta-1}} (2+3L)}
    {1 - 2^{\beta + 1} \theta \linf{\rho^{\alpha+1-\beta}} \Phi(A)} \Phi(A).
  \end{eqnarray*}
  Selecting $\theta = (1/2)
  \min\left(1/\left(2^{\beta+1}\linf{\rho^{\alpha+1-\beta}}
        \Phi(A)\right), 1/L
  \right)$ shows
  \[
  \lambda_2 \leq 2 d \cdot 3 \cdot 2^{\beta}
\linf{\rho^{\gamma-\beta-1}}(2+3L) 
  \max\left(L \Phi(A), 2^{\beta+1}\linf{\rho^{\alpha+1-\beta}} \Phi(A)^2 \vph\right).
  \]
  When $\gamma = (1,2,3)$, this simplifies into:
  \[
  \lambda_2 \leq 12(2+3L) d
  \max\left(L \Phi(A), 8 \Phi(A)^2 \vph\right).
  \]
  We note that, via the work shown in
  Section~\ref{sec:scaling},  we can strengthen our inequality
  to:
  \[
  \lambda_2 \leq 24 d
  \max\left(L \Phi(A), 8 \Phi(A)^2 \vph\right).
  \]

  \qedhere
