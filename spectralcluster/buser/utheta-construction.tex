\subsection{Proof Strategy: Mollification by Disks of Radius
  Proportional to \texorpdfstring{$\rho$}{rho}}
To prove Theorem \ref{thm:buser_n}, for $A \subset \Re^d$ fixed, we
construct an approximation $u_\theta$ of the characteristic function,
$u$, of $A$ for which the numerator and denominator of the Rayleigh
quotient, $R(u_\theta)$, approximate respectively the numerator and
denominator of this expression.  Specifically, $u_\theta$ will
constructed as a mollification of $u$, Recall the following two
equivalent definitions of a mollification.  They are equivalent by the
change of variables $z = x-\theta \rho(x) y$.

\begin{equation} \label{eqn:utheta}
u_\theta(x) 
\coloneqq \int_{B(0,1)} \!\!\! u(x-\theta \rho(x) y) \phi(y) \, dy
= \int \! u(z) \phi_{\theta \rho(x)}(x-z) \, dz,
\quad \text{ where } \quad
\phi_{\eta}(z) 
= \frac{1}{\eta^d} \phi\left(\frac{z}{\eta}\right),
\end{equation}

with $\theta > 0$ a parameter to be chosen and $\phi:\Re^d \rightarrow
[0,\infty)$ a smooth radially symmetric function supported in the unit
open ball $B(0,1)=\{x \in \Re^d \;| \; |x| < 1\}$ with unit mass
$\int_{\Re^d} \phi = 1$. When $\rho$ is constant it follows from the
Tonelli theorem that $\lone{u_\theta} = \lone{u}$; when $\rho$ is not
constant the following lemma shows that the latter still bounds the
former.

