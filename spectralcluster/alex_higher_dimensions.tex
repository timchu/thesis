%!TEX root = ms.tex

\section{Higher dimensions}
\label{sec:Ahigher_dim}
In this section the theory of functions of bounded variaion is used to
provide the abstract notions needed to justify certain formal
calculations in the proof of Theorem \ref{:??}.

\begin{definition}
For a measurable set $A\subseteq \Omega$, define
\begin{align*}
\abs{A}_\alpha \coloneqq \int_A \rho^\alpha(x)\,dx.
\end{align*}
\end{definition}
We define the weight of the boundary distributionally~\cite{betta2008weighted,parini2011introduction}.
\begin{definition}
\label{def:betaBdy}
For $A\subseteq\Omega$ a set of finite perimeter, define
\begin{align*}
\abs{\boundary A}_\beta \coloneqq \sup\set{\int_\Omega \chi_A(x) \div(\rho^\beta (x) \phi(x))\,dx\smid \phi\in C_c^1(\Omega),\, \norm{\phi}_{\infty}\leq 1}.
\end{align*}
\end{definition}

\begin{remark}
  This definition corresponds to the intuitive definition of the
  boundary integral of $\rho^\beta$ when $\boundary A$ is sufficiently
  regular.  Specifically, if $A\subseteq\Omega$ has smooth boundary
  and then,
  $$
  \abs{\boundary A}_\beta = \int_{\boundary A} \rho^\beta(x)\,d \mc H^{n-1}(x),
  $$
  where $\mc H^{n-1}$ denotes surface (Hausdorf) measure.
\end{remark}


\begin{definition}
Let $A\subseteq\Omega$ be a set of finite perimeter such that $\abs{A}_\alpha,\abs{\Omega\setminus A}_\alpha >0$. The \textit{isoperimetric ratio} of the cut induced by $A$ is
\begin{align*}
\Phi(A) &\coloneqq \frac{\abs{\boundary A}_\beta}{\min\left(\abs{A}_\alpha,\abs{\Omega\setminus A}_\alpha\right)}
\end{align*}
and the \textit{isoperimetric constant} of $\Omega$ with weight $\rho$ is
\begin{align*}
\Phi(\Omega)\coloneqq \inf_{A\subseteq\overline{\Omega}} \Phi(A).
\end{align*}
Here, $A$ is taken over sets of finite perimeter such that $\abs{A}_\alpha,\abs{\Omega\setminus A}_\alpha>0$.
\end{definition}

Again we define $\lambda_2$ variationally.
\begin{definition}
Let $u\in W^{1,2}(\Omega)$. The \textit{Rayleigh quotient} of $u$ is
\begin{align*}
R(u) &= \frac{\int_\Omega \rho^\gamma\norm{\grad u}^2\,dx}{\int_\Omega \rho^\alpha u^2\,dx}
\end{align*}
and the \textit{second eigenvalue of the Laplacian} on $\Omega$ with weight $\rho$ is
\begin{align*}
\lambda_2(\Omega) &\coloneqq \inf_{u\in W^{1,2}}\set{R(u) \smid \int_\Omega \rho^\alpha u \,dx = 0,\, u\neq 0}.
\end{align*}
\end{definition}

We now prove a weighted Cheeger inequality in higher dimensions. The proof is almost identical to the proof in one dimension. The only change in the proof is replacing the change of variables formula with a coarea formula.
\begin{theorem} \label{thm:Acheeger_n}
Let $\Omega\subseteq\RR^n$ be a bounded convex domain with Lipschitz boundary. Let $\rho:{\Omega}\to\RR_{>0}$ be a Lipschitz function. Then,
\begin{align*}
\Phi^2(\Omega) \leq 4 \norm{\rho^{\beta - \frac{\alpha+\gamma}{2}}}^2_\infty \lambda_2(\Omega).
\end{align*}
In particular, when $(\alpha,\beta,\gamma) = (1,2,3)$ we have
\begin{align*}
\Phi^{2}(\Omega) \leq 4\lambda_2(\Omega).
\end{align*}
\end{theorem}
\begin{proof}
Let $w\in W^{1,2}$ nonzero with $\int_\Omega \rho^\alpha w\,dx = 0$. Let $v = w+a1$ where $a$ is chosen such that $\abs{\set{v<0}}_\alpha = \abs{\set{v>0}}$. Note that
\begin{align*}
R(w) &= \frac{\int_\Omega \rho^\gamma \abs{\grad w}^2\,dx}{\int_\Omega \rho^\alpha w^2\,dx}\\
&\geq \frac{\int_\Omega \rho^\gamma \abs{\grad w}^2\,dx}{\int_\Omega \rho^\alpha w^2\,dx+ a^2\abs{\Omega}_\alpha}\\
&= R(v).
\end{align*}
Without loss of generality, the function $u = \max(v,0)$ satisfies $R(u)\leq R(v)$.

Let $\Omega_0 = \set{v>0}$. Let $g = u^2$. Noting that $\grad g = 2u\grad u$ a.e., we can apply Cauchy-Schwarz to obtain
\begin{align*}
\int_{\Omega_0}\rho^\beta \abs{\grad g}\,dx &= 2\int_{\Omega_0}\rho^\beta \abs{u}\abs{\grad u}\,dx\\
&\leq 2 \sqrt{\int_{\Omega_0} \rho^{2\beta - \alpha} \abs{\grad u}^2\,dx}\sqrt{\int_{\Omega_0} \rho^{\alpha} u^2\,dx}\\
&\leq 2 \norm{\rho^{\beta - \frac{\alpha+\gamma}{2}}}_\infty \sqrt{\int_{\Omega_0} \rho^\gamma \abs{\grad u}^2\,dx}\sqrt{\int_{\Omega_0} \rho^{\alpha} u^2\,dx}.
\end{align*}
Then, dividing by $\int_{\Omega_0} \rho^\alpha g\,dx$, we have
\begin{align*}
\frac{\int_{\Omega_0} \rho^\beta \abs{\grad g}\,dx}{\int_{\Omega_0} \rho^\alpha g\,dx} &\leq2 \norm{\rho^{\beta - \frac{\alpha+\gamma}{2}}}_\infty \sqrt{R(w)}.
\end{align*}
Let $A_t = \set{g>t}$. Then, by the weighted coarea formula,
\begin{align*}
\int_{\Omega_0} \rho^\beta \abs{\grad g}\,dx = \int_0^\infty \abs{\boundary A_t}_\beta\,dt.
\end{align*}
Writing $g(x) = \int_0^{g(x)} 1 \,dt$ and applying Tonelli's thoerem, we rewrite the denominator
\begin{align*}
\int_{\Omega_0} \rho^\alpha g\,dx = \int_{0}^\infty \abs{A_t}_\alpha\,dt.
\end{align*}
Thus, by averaging, there exists some $t^*$ such that
\begin{align*}
\Phi(\Omega) &\leq \Phi(A_{t^*})\\
&\leq \frac{\int_{\Omega_0} \rho^\beta \abs{\grad g}\,dx}{\int_{\Omega_0} \rho^\alpha g\,dx}\\
&\leq2 \norm{\rho^{\beta - \frac{\alpha+\gamma}{2}}}_\infty \sqrt{R(w)}.
\end{align*}
Optimizing over the set $\set{w\in W^{1,2}\smid w\neq 0,\,\int_\Omega \rho^\alpha w\,dx = 0}$ completes the proof.
\end{proof}

We now prove our weighted Buser-type inequality.

\begin{theorem} \label{thm:Abuser_n}
  Let $\Omega\subseteq\RR^n$ be a
  bounded convex domain with Lipschitz boundary. Let
  $\rho:\overline{\Omega}\to\RR_{>0}$ be a Lipschitz function with
  Lipschitz constant $L$. Then,
\begin{align*}
\lambda_2(\Omega) \leq 2 C(\phi,\beta)\linf{\rho^{\gamma-\beta-1}}(1+L) 
  \max\left(L \Phi(A), C(\beta)\linf{\rho^{\alpha+1-\beta}} \Phi(A)^2 \vph\right)
\end{align*}
\end{theorem}
% \begin{proof}
% Let $A\subseteq\overline{\Omega}$ be a set of finite perimeter such that $\abs{A}_\alpha,\abs{\Omega\setminus A}_\alpha>0$. We will show that there exists a $u$ with small Rayleigh quotient compared to $\Phi(A)$. Let $B = \Omega\setminus A$. Without loss of generality $\abs{A}_\alpha\leq \abs{B}_\alpha$ and hence $\Phi(A) = \frac{\abs{\boundary A}_\beta}{\abs{A}_\alpha}$. For notational convenience we will write $\Phi = \Phi(A)$ in this proof.

% Let
% \begin{align*}
% u(x) = \begin{cases}
% 	\abs{A}_\alpha/\abs{\Omega}_\alpha & x\in B\\
% 	-\abs{B}_\alpha/\abs{\Omega}_\alpha & x\in A\\
% 	0 & x\notin \Omega.
% \end{cases}
% \end{align*}

% Let $\theta>0$ be a constant independent of $x$ to be picked later and let $\delta = \theta\rho$.

% Let $\phi$ be the standard mollifier i.e. $\phi\in C_c^\infty(\RR^n)$ is a function from $\RR^n\to [0,\infty)$ satisfying $\int_{\RR^n} \phi\,dx = 1$ and $\supp(\phi)\subseteq B(0,1)$. Let $\phi_\epsilon(z) = (1/\epsilon^n)\phi(z/\epsilon)$.
% \begin{align*}
% u_\delta(x) &= \int_{\RR^n} u(x-\delta(x) y) \phi(y)\,dy\\
% &= \int_{\RR^n} u(z)\phi_{\delta(x)}(x-z)\,dz.
% \end{align*}

% At times we will need other smoothed versions of $u$. By lemma \alex{todo}, there exists a sequence $u_k \in BV(\Omega)\cap C^\infty(\Omega)$ such that $u_k\to u$ in $L^1(U)$ and
% \begin{align*}
% \int_\Omega \rho^\beta \abs{\grad u_k}\,dx \to \abs{\boundary A}_\beta
% \end{align*}
% as $k\to\infty$.

% We begin by bounding the numerator in $\mc R(u_\delta)$. We split the numerator into three terms $\rho^\gamma \abs{\grad u_\delta}^2 = \rho^{\gamma - \beta - 1} \left(\rho \abs{\grad{u_\delta}}\right)\left(\rho^\beta \abs{\grad{u_\delta}}\right)$. We will bound the first two terms uniformly in terms of $n,\theta, L$. We will show that the third term looks like ``$\grad u$.''

% A calculation shows
% \begin{align*}
% \grad u_\delta(x) &= \grad_x\int u(z)\phi_{\delta(x)}(x-z)\,dz\\
% &= \int u(z)\grad_x\left(\phi_{\delta(x)}(x-z)\right)\,dz\\
% &= \int u(z)\grad_x\left(\frac{1}{\delta(x)^n}\phi\left(\frac{x-z}{\delta(x)}\right)\right)\,dz\\
% % &= \int u(z)\left(\frac{-n}{\delta}\phi_{\delta}(x-z)\grad \delta + \frac{1}{\delta^n}\grad \phi\left(\frac{x-z}{\delta}\right)\frac{\delta I - \grad \delta \otimes (x-z)}{\delta^2}\right)\,dz\\
% &= \int u(z)\left(\frac{-n}{\delta}\phi_{\delta}(x-z)\grad \delta + \frac{1}{\delta^{n+1}}\left(I - \grad \delta \otimes \frac{x-z}{\delta}\right)\right)\grad \phi\left(\frac{x-z}{\delta}\right)\,dz.
% \end{align*}
% Multiplying by $\rho(x)$ and taking the magnitude of the resulting vector, we deduce
% \begin{align*}
% \rho(x)\abs{\grad u_\delta(x)} &\leq \norm{\frac{\rho}{\delta}}_\infty \norm{u}_\infty
% \int n\phi_{\delta}(x-z)\abs{\grad \delta} + \frac{1}{\delta^{n}}(1+\abs{\grad \delta})\abs{\grad \phi\left(\frac{x-z}{\delta}\right)}\,dz.
% \end{align*}
% Note that $\int \phi_\delta(x-z)\,dz = 1$ and $\int \frac{1}{\delta^n}\abs{\grad \phi\left(\tfrac{x-z}{\delta}\right)}\,dz = \int \abs{\grad \phi(z)}\,dz$ is a constant depending only on $n$. Define $c_n = n + \int\abs{\grad \phi(z)}\,dz$. Thus we can bound $\rho \abs{\grad u_\delta}$ uniformly as
% \begin{align*}
% \rho(x)\abs{\grad u_\delta(x)} &\leq c_n \norm{\frac{\rho}{\delta}}_\infty \norm{u}_\infty (1+\norm{\grad \delta}_\infty)\\
% &\leq c_n (1+L)/\theta.
% \end{align*}
% Next bound $\grad u_\delta$ by $\grad u_k$.
% \begin{align*}
% \grad u_\delta(x) &= \grad_x \int u(x-\delta(x)y) \phi(y)\,dy\\
% &= \lim_{k\to\infty} \grad_x \int u_k(x-\delta(x)y)\phi(y)\,dy\\
% &= \lim_{k\to\infty} \int (I - y\otimes \grad \delta(x))\grad u_k(x-\delta(x)y)\phi(y)\,dy.
% \end{align*}
% We multiply this bound by $\rho^\beta(x)$ and use the estimate $\rho(x-\delta(x)y) \geq (1-L\theta)\rho(x)$ to obtain
% \begin{align*}
% \rho^\beta(x)\abs{\grad u_\delta(x)} &\leq (1+\norm{\grad\delta}_\infty)\lim_{k\to\infty} \int \rho^\beta(x)\abs{\grad u_k(x-\delta(x)y)}\phi(y)\,dy\\
% &\leq \frac{1+\norm{\grad\delta}_\infty}{(1-L\theta)^\beta}\lim_{k\to\infty} \int \rho^\beta(x-\delta(x)y)\abs{\grad u_k(x-\delta(x)y)}\phi(y)\,dy\\
% &= \frac{1+\norm{\grad\delta}_\infty}{(1-L\theta)^\beta}\lim_{k\to\infty} \int \rho^\beta(z)\abs{\grad u_k(z)}\phi_{\delta(x)}(x-z)\,dz.
% \end{align*}
% Integrating this estimate over $\Omega$ and applying Lemma \alex{todo}
% \begin{align*}
% \int_\Omega \rho^\beta(x) \abs{\grad u_\delta(x)}\,dx &\leq \frac{1+\norm{\grad\delta}_\infty}{(1-L\theta)^\beta} \lim_{k\to\infty} \int_{\Omega} \int \rho^\beta(z)\abs{\grad u_k(z)}\phi_{\delta(x)}(x-z)\,dz\,dx\\
% &\leq \frac{1+\norm{\grad\delta}_\infty}{(1-L\theta)^{\beta + 1}} \lim_{k\to\infty} \int_{\RR^n} \rho^\beta(z)\abs{\grad u_k(z)}\,dz\\
% &= \text{\alex{?}} \frac{1+\norm{\grad\delta}_\infty}{(1-L\theta)^{\beta + 1}} \abs{\boundary A}_\beta.
% \end{align*}
% Thus we can bound the numerator of $\mc R(u_\delta)$ as
% \begin{align*}
% \int_\Omega \rho^{\gamma} \abs{\grad u_\delta}^2 \,dx &\leq c_n\norm{\rho^{\gamma-\beta-1}}_\infty\frac{(1+L)(1+L\theta)}{\theta(1-L\theta)^{\beta+1}} \abs{\boundary A}_\beta. 
% \end{align*}

% \end{proof}

% \section{Proof of Theorem \ref{thm:buser_n}}
% \label{app:proof}

Classical calculus will be used to compute the integrals over the
boundary of the sets in Definition \ref{def:betaBdy}.  Specifically,
given a bounded convex domain $\Omega \subset \Re^d$
% with Lipschitz boundary
and a set $A \subset \Omega$ with finite perimeter, let $u\coloneqq \chi_A$, the characteristic function of $A$, i.e., $u(x) =
1$ if $x \in A$ and zero otherwise.  Then there exists a sequence of
functions $\{u_n\}_{n=1}^\infty \subset C^\infty(\Re^d)$ with $u_n
\rightarrow u$ in $\Lone$ for which \cite{EvansMeasure15}
% $$
% \int_{\partial A} \rho^\beta
% = \lim_{n \rightarrow \infty} \int_\Omega |\nabla u_n| \rho^\beta
% \eqqcolon  \int_\Omega |\nabla u| \rho^\beta.
% $$
\begin{equation} \label{eqn:Aapprox}
\abs{\boundary A}_\beta
= \lim_{n \rightarrow \infty} \int_\Omega  \rho^\beta|\nabla u_n|
\eqqcolon  \int_\Omega  \rho^\beta|\nabla u|.
\end{equation}
Interchanging $A$ and $\Omega \setminus A$ if necessary, it follows that
$\Phi(A)$ defined in Definition \ref{def:betaBdy} can be written as
% $$
% \Phi(A) 
% = \frac{\int_\Omega \rho^\beta |\nabla u|}{\int_\Omega \rho^\alpha u}
% = \frac{\lone{\rho^\beta \nabla u}}{\lone{\rho^\alpha u}}
% \qquad
% \text{ where } u = \chi_A.
% $$
$$
\Phi(A) 
= \frac{\int_\Omega \rho^\beta |\nabla u|}{\int_\Omega \rho^\alpha u}
= \lim_{n\to\infty}\frac{\int_\Omega \rho^\beta\abs{\grad u_n}}{\int_\Omega\rho^\alpha \abs{u_n}}.
$$
To prove Theorem \ref{thm:Abuser_n} we construct an approximation $u_\theta$ of
$u$ for which the numerator and denominator of the Raleigh quotient, $R(u_\theta)$,
approximate respectively the numerator and denominator of this expression.
Specifically, $u_\theta$ will constructed as a mollification of $u$,
\begin{equation} \label{eqn:Autheta}
u_\theta(x) 
\coloneqq \int_{B(0,1)} \!\!\! u(x-\theta \rho(x) y) \phi(y) \, dy
= \int_{\Re^n} \! u(z) \phi_{\theta \rho(x)}(x-z) \, dz,
\quad \text{ where } \quad
\phi_{\eta}(z) 
= \frac{1}{\eta^d} \phi\left(\frac{z}{\eta}\right),
\end{equation}
with $\theta > 0$ a parameter to be chosen and $\phi:\Re^d \rightarrow
[0,\infty)$ a smooth function supported in the unit ball $B(0,1)=\{x
\in \Re^d| |x| < 1\}$ with unit mass $\int_{\Re^d} \phi = 1$. When
$\rho$ is constant it follows from the Tonelli theorem that
$\lone{u_\theta} \leq \lone{u}$; when $\rho$ is not constant Lemma
\ref{lem:lonetheta} shows that the latter still bounds the former.


\begin{proof} (of Theorem \ref{thm:Abuser_n})
Fix $A \subset \Omega$ with $|A|_\alpha \leq |\Omega|_\alpha / 2$ and let
$u(x) = \chi_A(x)$ be the characteristic function of $A$. Setting 
$\ubar$ to be the weighted average of $u$,
$$
\ubar 
= \frac{\int_\Omega \rho^\alpha u}{\int_\Omega \rho^\alpha}
= \frac{\int_A \rho^\alpha}{\int_\Omega \rho^\alpha}
= \frac{|A|_\alpha}{|\Omega|_\alpha} \in [0,1/2],
\qquad \text{ then } \qquad
\int_\Omega \rho^\alpha (u-\ubar) = 0,
$$
and
$$
\lone{\rho^\alpha(u-\ubar)} 
= \int_\Omega \rho^\alpha |u-\ubar| = |A|_\alpha (1-\ubar)
= 2 \int_\Omega \rho^\alpha |u-\ubar|^2.
$$
Since $|A|_\alpha = \lonea{u}$ and $1-\ubar \in [0,1/2]$ it follows that
\begin{equation} \label{eqn:APhiA}
  (1/2) \frac{\lone{\rho^\beta \nabla u}}{\lone{\rho^\alpha(u-\ubar)}} 
  \leq \Phi(A) = \frac{\lone{\rho^\beta \nabla u}}{\lone{\rho^\alpha u}}
  \leq \frac{\lone{\rho^\beta \nabla u}}{\lone{\rho^\alpha(u-\ubar)}}.
\end{equation}
In the calculations below we omit the limiting argument with smooth
approximations of $u$ in equation \eqnref{:Aapprox} which justify
formula involving $\nabla u$. In particular, only the $\Lone$ norm
of $\rho^\beta |\nabla u|$ has meaning; the $\Ltwo$ norm is undefined.

Next, let $u_\theta$ be the mollification of (an extension of) $u$
given by equation \eqnref{:Autheta}. Then $u_\theta(x)$ is a local
average average of $u$ so $u_\theta(x) \geq 0$, $\linf{u_\theta} \leq
1$ and $\linf{u-u_\theta} \leq 1$.  Letting $L$ denote the Lipschitz
constant of $\rho$, the parameter $\theta$ will to be chosen 
less than $1/(2L)$ so that that Lemma \ref{lem:lonetheta} is applicable
with constant $c = 1/2$.

The remainder of the proof constructs an upper bound on the numerator
$\int_\Omega \rho^\gamma |\nabla u_\theta|^2$ of the Raleigh quotient
for $u_\theta - \ubar_\theta$ by $\lone{\rho^\beta \nabla u}$ and to
lower bound the denominator $\int_\Omega \rho^\alpha (u_\theta -
\ubar_\theta)^2$ by $\lone{\rho^\alpha (u-\ubar)}$. The conclusion
of the theorem then follows from equation \eqnref{:APhiA}.

\begin{itemize}
\item {\em Upper Bounding the Numerator:} To bound the $L^2$ norm in
  the numerator of the Raleigh quotient by the $L^1$ norm in the
  numerator of the expression for $\Phi(A)$ it is necessary to obtain
  uniform bound on $\nabla u_\theta$. To do this take the gradient of
  the second representation of $u_\theta$ in equation \eqnref{:Autheta}
  to get
  $$
  \nabla u_\theta(x) 
  = \int u(z) \left\{
    \frac{-d}{\theta \rho} \phi_{\delta}(x-z) \nabla \rho
    + \frac{1}{(\theta \rho)^{d+1}} 
    \left( I 
      + \nabla \rho \otimes \frac{x-z}{\theta \rho} \right)
    \nabla \phi \left(\frac{x-z}{\theta \rho} \right) \right\} \, dz.
  $$
  Multiplying by $\rho$ and noting that $|u(x)| \leq 1$ gives the bound
  $$
  |\rho(x) \nabla u_\theta(x)|
  \leq \frac{C(\phi)}{\theta} (1+L) \linf{u},
  \quad \text{ where } \quad
  C(\phi) = \int_{\Re^d} |\nabla \phi(y)| \, dy,
  $$
  and $L = \linf{\nabla \rho}$ is the Lipschitz constant for $\rho$. 

  Next, taking the gradient of the first representation of $u_\theta$
  in equation \eqnref{:Autheta} and using the chain rule shows
  $$
  \nabla u_\theta(x) 
  = \int_{\Re^d} 
  (I - \theta \nabla \rho \otimes y) \nabla u(x-\theta \rho u) \phi(y) \, dy,
  $$
  so
  $$
  \rho^\beta(x) \nabla u_\theta(x) 
  = \int_{\Re^d} 
  (I - \theta \nabla \rho \otimes y)
  \frac{\rho^\beta(x)}{\rho^\beta(x-\theta \rho y)}
  \rho^\beta(x-\theta \rho y) \nabla u(x-\theta \rho y) \phi(y) \, dy.
  $$
  The ratio in the integrand is bounded using the Lipschitz assumption
  on $\rho$ (and $|y| \leq 1$),
  \begin{equation} \label{eqn:ArhoRatio}
    \frac{\rho(x)}{\rho(x-\theta \rho y)}
    \leq \frac{\rho(x)}{\rho(x) - L \theta \rho(x)}
    = \frac{1}{1 - L \theta} \leq 2,
    \qquad \text{ when } \theta < 1 / (2L).
  \end{equation}
  Similarly, the $\ell^2$ matrix norm of $I - \theta \nabla \rho \otimes
  y$ is bounded by $3/2$, and and application of Lemma
  \ref{lem:lonetheta} then shows
  $$
  \lone{\rho^\beta \nabla u_\theta} 
  \leq C(\beta) \lone{\rho^\beta \nabla u},
  \qquad \text{ when } \theta < 1 / (2L).
  $$
  Combining the two estimates gives an upper bound for the Raleigh
  quotient 
  $$
  \int_\Omega \rho^\beta |\nabla u_\theta|^2
  = \int_\Omega \rho^{\gamma-\beta-1} \,
  \rho |\nabla u_\theta| \, \rho^\beta |\nabla u_\theta|
  \leq C(\phi,\beta) \linf{\rho^{\gamma-\beta-1}} \frac{1+L}{\theta} 
  \linf{u} \lone{\rho^\beta \nabla u},
  $$
  under the assumption $\theta \leq 1/(2L)$. Since $\linf{u} \leq 1$
  it can be dropped from the right hand side.

\item {\em Lower Bound on the Denominator:} Let $\ubar$ and
  $\ubar_\theta$ be the $\rho^\alpha$--weighted averages of $u$ and
  $u_\theta$ and let $\ltwoa{.}$ denote the $L^2$ space with this
  weight. Using the property that subtracting the average from a
  function reduces the $L^2$ norm it follows that
  $$
  \ltwoa{u_\theta - \ubar_\theta}
  \geq \ltwoa{u-\ubar} - \ltwoa{u_\theta - u - (\ubar_\theta-0)}
  \geq \ltwoa{u-\ubar} - \ltwoa{u_\theta - u}.
  $$
  If $a \geq b-c$ then $a^2 \geq b^2/2 - c^2$, so a lower bound for
  the denominator of the Raleigh quotient
  \begin{equation} \label{eqn:Autmu}
    \ltwoa{u_\theta - \ubar_\theta}^2
    \geq (1/2) \ltwoa{u}^2 - \ltwoa{u_\theta - u}^2
    \geq (1/4) \lonea{u} - \lonea{u_\theta - u},
  \end{equation}
  where the identity $\ltwoa{u}^2 = \lonea{u}/2$ and bound
  $\linf{u_\theta - u} \leq 1$ were used in the last step.

  To estimate the difference $\lonea{u_\theta - u}$ use the fundamental
  theorem of calculus to write
  \begin{eqnarray*}
    u_\theta(x) - u(x)
    &=& \int (u(x - \theta \rho y) - u(x)) \phi(y) \, dy \\
    &=& \int \! \int_0^1
    -\theta \rho \nabla u(x - t \theta \rho y).y \phi(y) \, dy \, dt \\
    &=& \int \! \int_0^1
    \frac{-\theta \rho(x)}{\rho^\beta(x-t\theta \rho y)} 
    \rho^\beta(x - t\delta y) \nabla u(x - t \delta y).y \phi(y) \, dy \, dt,
  \end{eqnarray*}
  so that 
  $$
  \rho^{\alpha}(x) ( u_\delta(x) - u(x) )
  = \int \! \int_0^1
  \frac{-\theta \rho^{\alpha+1}(x)}{\rho^\beta(x-t\delta y)} 
  \rho^\beta(x - t\delta y) 
  \nabla u(x - t \theta \rho y).y \phi(y) \, dy \, dt.
  $$
  Equation \eqnref{:ArhoRatio} bounds the ratio $\rho(x)/ \rho(x - t\delta y)$
  and an application of Lemma \ref{lem:lonetheta} then shows
  $$
  \lonea{u_\theta - u}
  \leq C(\beta) \linf{\rho^{\alpha+1-\beta}} \theta \lone{\rho^\beta \nabla u}
  \qquad \text{ when } \theta < 1 / (2L).
  $$
  Using this estimate in \eqnref{:Autmu} gives a lower bound on the
  denominator of the Raleigh quotient,
  $$
  \ltwoa{u_\theta - \ubar_\theta}^2
  \geq (1/4) \lonea{u} 
  - C(\beta) \theta \linf{\rho^{\alpha+1-\beta}} \lone{\rho^\beta \nabla u},
  \qquad \text{ when } \theta < 1 / (2L).
  $$
  {\em Bounding the Raleigh Quotient:} Combining the two steps above
  provides an upper bound for the Raleigh quotient of $u_\theta - \ubar_\theta$,
  \begin{eqnarray*}
    \lambda_2 
    &\leq& \frac{\int_\Omega \rho^\gamma |\nabla u_\theta|^2}
    {\int_\Omega \rho^\alpha (u_\theta - \ubar_\theta)^2} \\
    &\leq&  \frac{C(\phi,\beta)}{\theta}
    \frac{\linf{\rho^{\gamma-\beta-1}} (1+L) \lone{\rho^\beta \nabla u}}
    {\lonea{u} 
      - C(\beta) \theta \linf{\rho^{\alpha+1-\beta}} \lone{\rho^\beta \nabla u}} \\
    &\leq&  \frac{C(\phi,\beta)}{\theta}
    \frac{\linf{\rho^{\gamma-\beta-1}} (1+L)}
    {1 - C(\beta) \theta \linf{\rho^{\alpha+1-\beta}} \Phi(A)} \Phi(A).
  \end{eqnarray*}
  Selecting $\theta = (1/2)
  \min\left(1/(C(\beta)\linf{\rho^{\alpha+1-\beta}} \Phi(A)), 1/L
  \right)$ shows
  \[
  \lambda_2 \leq 2 C(\phi,\beta)\linf{\rho^{\gamma-\beta-1}}(1+L) 
  \max\left(L \Phi(A), C(\beta)\linf{\rho^{\alpha+1-\beta}} \Phi(A)^2 \vph\right).
  \qedhere
  \]
\end{itemize}
\end{proof}

