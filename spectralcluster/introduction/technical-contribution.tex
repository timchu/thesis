% \subsection{Differences between our work and past work}\label{sec:cheeger-buser-difficulties}
% Our work differs from past work in the following key ways:
% 
% \begin{enumerate}
% \item Our work differs from past practical work on spectral sweep cuts
% cuts~\cite{TrillosRate15, TrillosVariational15, ShiMalik97,
%   NgSpectral01}, as those methods perform what we call a $(\alpha = 1,
%     \gamma = 2)$-sweep cut. These sweep cuts have no theoretical
%     guarantees, much less a guarantee on their $(\alpha, \beta)$
%     sparsity. Lemma~\ref{lem:converse} shows that no Cheeger and Buser inequality can
%     simultaneously hold for any setting of $\beta$ when $\alpha = 1,
%     \beta = 2$. 
%     
%     We will further show that using a $(\alpha=1,
%     \gamma=2)$-sweep cut can lead to undesirable cuts of $1$-Lipschitz
%     probability densities, with poor sparsity guarantees.
% 
% \item We note that probability density Cheeger-Buser is not easily implied by
%   graph or manifold Cheeger-Buser. For a lengthier discussion on this,
%     see Appendix~\ref{app:notgraph}.
% \item We
% do not require any assumptions on our probability density
% except that it is Lipschitz. Past work on Cheeger-Buser inequality
% for densities focused on log-concave distributions, or
% mixtures thereof~\cite{Lee18survey, Ge2018}.  
% \item For our work, the probability
% density $\rho$ is not required to be bounded away from $0$.
% This is a sharp departure from many existing results: past results on
% partitioning probability densities required a positive lower bound on
% $\rho$~\cite{von2008consistency, TrillosRate15}.
% The strongest results in fields like linear elliptic partial
% differential equations depend on $\rho$ being bounded away from
% $0$ ~\cite{w17}.
% \end{enumerate}
% Our work is the first spectral sweep cut algorithm 
% that guarantees a sparse cut on Lipschitz densities $\rho$, without requiring strong
% parametric assumptions on $\rho$.

\subsubsection{Technical Contribution}
The key technical contribution of our proof is proving Buser's
inequality on Lipschitz probability densities via
mollification~\cite{s38, f44} with disks of varying radius. 
This chapter is the first time mollification with disks of varying radius
have been used. We emphasize that the most difficult part of our paper
is proving the Buser inequality.

Mollification has a long history in mathematics dating back to Sergei
Sobolev's celebrated proof of the Sobolev embedding
theorem~\cite{s38}. It is one of the key tools in numerical
analysis, partial differential equations, fluid mechanics, and functional
analysis~\cite{f44, lw01, ss09, m03}, and analogs of mollification have been
used in computational complexity settings~\cite{dkn09}.  Informally
speaking, mollification is used to
create a series of smooth functions approximating a non-smooth
function,  by  convolving the original function with a
smooth function supported on a disk.  
Notably, an approach using convolution is used by Buser in~\cite{Buser82} to
prove the original Buser's inequality, albeit with an intricate
pre-processing step on any given cut.

To prove Buser's inequality on Lipschitz probability density functions $\rho$, we
will show that given a cut $C$ with low $(\alpha, \beta)$-sparsity, we can find a function $u$ with low $(\alpha, \gamma)$-Rayleigh
quotient. We build $u$ by starting with the indicator function $I_C$ for
cut $C$ (which is $1$ on one side of the cut and $0$ on the other).
Next, we mollify this function with disks of varying radii. In
particular, for each point $r$ in the domain of $\rho$, we
spread out the point mass $I_C(r)$ over a disk of radius
proportional to $\rho(r)L$, where L is the Lipschitz constant of $\rho$.
The resulting function $u$ obtained by `spreading out' $I_C$ will have
low $(\alpha, \gamma)$-Rayleigh quotient.

For all past uses of mollification, the disks on which the smooth
convolving function is supported (we call this the mollification
    disk) have the same radius throughout
the manifold. The use of a uniform radius disk is critical for most uses and proofs in mollification.  
Our contribution is to allow the disks to vary in
radius across our density.  This variation in radius allow us to
deal with functions that approach $0$ (and explains the importance
    of the density being Lipschitz). No mollification disks
centered anywhere in our probability density will intersect
the $0$-set of the density. This overcomes significant
hurdles in many results for functional analysis and PDEs, as
many past significant results related to partial
differential equations rely on having a positive lower
bound on the density~\cite{w17, TrillosRate15}.

Proving our Buser inequality using
mollification by disks of various radius requires a fairly
delicate proof with many pages of calculus. Our key technical
lemma is a bound on how the $l_1$ norm of a mollified function
when the mollification disks have various radius, which can be
found in Section~\ref{sec:key_lemma}. 
