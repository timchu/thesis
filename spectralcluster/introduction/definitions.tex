\subsection{Definitions}\label{sec:definitions}
To establish our Cheeger and Buser inequalities, we define new
notions of isoperimetric constant (equivalently, sparsity), Rayleigh
quotients, eigenvalues, eigenvectors, and sweep cuts. Appropriate
definitions will let us establish basic spectral theory in the Lipschitz
probability density setting.

\vspace{2 mm}

\begin{definition} Let $\rho$ be a probability density function with domain
  $\mathbb{R}^d$, and let $A$ be a
  subset of $\mathbb{R}^d$.

  The \textbf{$(\alpha, \beta)$-sparsity} of the cut induced by $A$ is
  denoted by $\Phi(A)$. It is defined as $(d-1)$ dimensional integral of $\rho^\beta$ on the cut, divided
  by the $d$ dimensional integral of $\rho^\alpha$ on the side of the
  cut where this integral is smaller.  \end{definition}

For nice smooth cuts this
  intuitive definition is fine but a more general and precise
  definition using total variation is given in
  definition~\ref{def:betaBdy}

  \begin{definition}
The \textbf{$(\alpha, \beta)$-isoperimetric constant} of
  $\rho$ is defined as the infimum of $\Phi(A)$ over all $A$.
  \end{definition}

\vspace{2 mm}

\begin{definition}
The \textbf{$(\alpha, \gamma)$-Rayleigh quotient} of $u$ with
respect to $\rho:\Re^d \to \Re_{\geq 0}$ is:

\[
  R_{\alpha, \gamma}(u) \coloneqq \frac{\int_{\Re^d} \rho^\gamma
  |\nabla u|^2}{\int_{\Re^d} \rho^{\alpha}|u|^2} 
\]

A \textbf{$(\alpha, \gamma)$-principal eigenvalue} of $\rho$ is
$\lambda_2$, where:

\[ \lambda_2 := \inf_{\int \rho^\alpha u = 0} R_{\alpha, \gamma}(u). \]

Define a \textbf{$(\alpha, \gamma)$-principal eigenfunction} of
$\rho$ to be a function $u$ such that $R_{\alpha, \gamma}(u) =
\lambda_2$, if such $u$ exists.
\end{definition}
\vspace{2 mm}

Now we define a sweep cut for a given function with respect to a
a positive valued function supported on $\mathbb{R}^d$:

\vspace{2 mm}
\begin{definition} Let $\alpha, \beta$ be two real numbers, and $\rho$ be
  any function from $\Re^d$ to $\Re_{\geq 0}$.
  Let $u$ be any function from $\Re^d \to \Re$, and let $C_{t, u}$ be the cut
  defined by the set $\{s \in \Re^d \; | \; u(s) > t\}$. 
  
  The \textbf{sweep-cut} algorithm
  for $u$ with respect to $\rho$ returns the cut $C_{t,u}$ of minimum $(\alpha,
  \beta)$ sparsity, where this sparsity is measured with respect to $\rho$.

When $u$ is a $(\alpha, \gamma)$-principal eigenfunction, the sweep cut is called
a \textbf{$(\alpha, \gamma)$-spectral sweep cut} of $\rho$.  
\end{definition}

\textbf{Additional Definitions:}

A function $\rho: \Re^d \to \Re_{\geq 0}$ is
\textbf{$L$-Lipschitz} if $|\rho(x)-\rho(y)|_2 \leq L|x-y|_2$ for
all $x, y \in \Re^d.$

A function is $\rho:\Re^d \to \Re_{\geq 0}$ is
\textbf{$\alpha$-integrable} if $\int_{\Re^d} \rho^\alpha$ is
well defined and finite. Throughout this chapter, we assume $\rho$
is always $\alpha$-integrable.

\vspace{3 mm}
Our definitions depend on three constants: $\alpha$, $\beta$, and
$\gamma$. Informally, these constants can be thought of as the mass
constant, the cut constant, and the spring constant respectively. In the
graph and manifold setting, the spring and cut constant are the same. One key
contribution of our definitions is the decoupling of the cut and spring
constants which will allow us to get tight Cheeger and Buser results.


