\subsection{Theorems}

%Unfamiliar terms like $(\alpha, \beta)$-sparsity, $(\alpha, \gamma)$-spectral sweep cuts, and $(\alpha,\beta)$-principal eigenvalue are defined in Section~\ref{sec:definitions}.

\begin{theorem}\label{thm:Cheeger-Buser}
  \textbf{Probability Density Cheeger and Buser:}

  Let $\rho:\mathbb{R}^d \rightarrow \mathbb{R \geq 0}$ be an $L$-Lipschitz
  density function. Let $\alpha = \beta - 1 = \gamma - 2$.

  Let $\Phi$ be the infimum $(\alpha,\beta)$-sparsity of a cut through
  $\rho$, and let $\lambda_2$ be the $(\alpha,\gamma)$-principal eigenvalue of
  $\rho$. Then:
  \[ \Phi^2/4 \leq \lambda_2 \]
  and 
  \[\lambda _2 \leq O_{\alpha, \beta}(d \max(L \Phi, \Phi^2)).\]
  The first inequality is \textbf{Probability Density Cheeger}, and
  the second inequality is \textbf{Probability Density Buser}.
\end{theorem}
In particular, a Cheeger and Buser inequality exist when $\alpha = 1,
\beta = 2, \gamma=3$. Note that we don't need $\rho$ to have a total mass of $1$ for any
of our proofs. The overall probability mass of $\rho$ can be arbitrary.


Theorem~\ref{thm:Cheeger-Buser} has partial converses:

\begin{lemma} \label{lem:cheeger-converse} If $\alpha + \gamma >
  2\beta$, the Cheeger inequality in Theorem~\ref{thm:Cheeger-Buser}
  does not hold.
\end{lemma}

\begin{lemma} \label{lem:buser-converse} If $\gamma \geq 1$ and $\gamma -1 < \beta$, then the
  Buser inequality in Theorem~\ref{thm:Cheeger-Buser} does not
  hold.
\end{lemma}

In particular, if $\alpha = \beta = \gamma = 1$, the Buser inequality
fails.  If $\alpha = 1, \gamma =2$, no
Cheeger-Buser inequality can hold for any $\beta$. These settings
encompass most past work on sparse cuts and eigenvectors in probability
densities, as mentioned in Section~\ref{sec:past-prob}.

We apply these inequalities to show that spectral sweep cuts of
Lipschitz probability density functions give sparse cuts of the density.
This contrasts with past work on sweep cuts in probability
densities, as mentioned in Section~\ref{sec:past-prob}.

\begin{theorem} \label{thm:sweep-cut} 
  \textbf{Spectral Sweep Cuts give Sparse Cuts:}

  Let $\rho:\Re^d \to \Re_{\geq 0}$ be an $L$-Lipschitz probability
  density function, and let $\alpha = \beta - 1 = \gamma - 2$.
  
  The $(\alpha,\gamma)$-spectral sweep cut of $\rho$ has 
  $(\alpha, \beta)$ sparsity $\Phi$ satisfying:
  \[ 
  \Phi_{OPT} \leq \Phi \leq O(\sqrt{dL\Phi_{OPT}} ).
  \]
  Here, $\Phi_{OPT}$ refers to the optimal $(\alpha,\beta)$ sparsity of
  a cut on $\rho$. 
\end{theorem}

In words, the spectral sweep cut of the
  $(\alpha, \gamma)$  eigenvector gives a provably good approximation to
  the sparsest $(\alpha, \beta)$ cut, as long as $\beta = \alpha+1$ and
  $\gamma = \alpha+2$.   We will also show that this theorem is not possible
  for past definitions of eigenvectors and sparsity on probability
  density functions.
