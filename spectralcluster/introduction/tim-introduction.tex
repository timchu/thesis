\section{Introduction}

Clustering, the task of
partitioning data into groups, is one of the fundamental tasks
in machine learning~\cite{bishopBook, NgSpectral01, e96, kk10clustering, sw18}.
The data that practitioners seek to cluster is commonly modeled as samples from a probability density function,
an assumption foundational in theory, statistics, and AI~\cite{bishopBook,
TrillosContin16, bc17focs, mv10, Ge2018}. 
A clustering algorithm on data drawn from a probability density function
should ideally converge to a partitioning of the
probability density function, as the number of samples grows
large~\cite{von2008consistency, TrillosVariational15,
TrillosContin16}.
 
% As the sample count grows, simple tasks like 2-way clustering will
% ideally converge to a cut of the underlying probability density.
% Surprisingly, there are almost no clustering algorithms in the machine
% learning literature that give any guarantees on the cut quality of the
% probability density, absent strong parametric assumptions on the
% density function~\footnote{Strong parametric assumptions may include:
% being a mixture of Gaussians, or a mixture of log-concave
% distributions.}. 

In this paper, we outline a new strategy for
clustering. We then make progress on implementing this strategy. We
build clustering algorithms by addressing the following two questions:

\begin{enumerate}
  \item How can we partition probability density functions? How can we
    do this so
    that two data points drawn from the same part of the partition are
    likely to be similar, and two data points drawn from different parts of the partition
    are likely to be dissimilar?

  \item What clustering algorithms converge to such a partition, 
    as the number of samples from the density function grows large?
\end{enumerate}

In this paper, we address both points, with emphasis on the first.  We
focus on the special case of $2$-way partitioning, which can be seen as
finding a good cut on the probability density function. First, we
propose a new notion of sparse (or isoperimetric)
cuts on density functions. We call this an $(\alpha, \beta)$-sparse cut,
for real parameters $\alpha$ and $\beta$. Next, we propose a new notion of spectral
sweep cuts on probability densities, called a $(\alpha,
\gamma)$-spectral sweep cut, for real parameters $\alpha$ and $\gamma$.
We show that a $(\alpha, \gamma)$-spectral sweep cut provably
approximates an $(\alpha, \beta)$-sparse cut when
$\beta = \alpha+1$ and $\gamma=\alpha+2$. In particular, $\alpha = 1, \beta = 2,
\gamma=3$ is such a setting.
Our result holds for any $L$-Lipschitz
probability density function on $\mathbb{R}^d$, for any $d$. 
We will give evidence that partitioning probability densities via
$(\alpha=1, \beta=2)$-sparse cuts agrees with the machine learning
intuition that data drawn from the same part of the partition are
similar, and data drawn from different parts of the partition are
dissimilar.  
% To our
% knowledge, this is the first result we know of that gives
% theoretical guarantees on a partitioning method for probability density
% functions, where these density functions do not have any strong
% parametric assumptions\footnote{Examples of parametric assumptions
% include: assuming that a density function
% is the mixture of finitely many log-concave functions, or a mixture of
% Gaussians.}.

The key mathematical contribution of this paper is a new Cheeger and Buser
inequality for probability density functions, which we use to prove
that $(\alpha,\gamma)$-spectral sweep cuts approximate $(\alpha,
\beta)$-sparse cuts on probability
density functions for the aforementioned settings of $\alpha, \beta, \gamma$.
These inequalities
are inspired by the Cheeger and Buser inequalities on graphs and
manifolds~\cite{AlonM84, Cheeger70, Buser82}, which have received
considerable attention in graph algorithms and machine
learning~\cite{ChungBook97, SpielmanTeng2004, Orecchia08, Orecchia2011,
kw16, belkin2004semisup}. These new inequalities do not directly follow
from either the graph or manifold Cheeger inequalities,
something we detail in Section~\ref{sec:cheeger-buser-difficulties}.
We note that our Cheeger and Buser inequalities for probability density
functions require a careful definition of eigenvalue and
sparse/isoperimetric cut for density functions: existing definitions
lead to false inequalities. 
% The Cheeger and Buser inequality for
% probability density functions is the first new Cheeger and Buser
% inequality since Alon and Milman's development of graph Cheeger and
% Buser~\cite{AlonM84}, and opens up natural conjectures about a potential
% Cheeger and Buser inequality for
% probability density functions supported on manifolds.

Finally, our paper will present a discrete $2$-way clustering algorithm that we
suspect converges to the $(\alpha=1, \gamma=3)$-spectral sweep cut as the
number of data points grows large.  Our
algorithm bears similarity to classic
spectral clustering methods, with some important differences. We note
that we do not prove convergence of this discrete clustering method to
our method for cutting probability densities, and leave this for future work.

% We briefly note some difficulties that we overcome in our work: first,
% graph and manifold Cheeger inequalities do \textit{not} imply Cheeger
% inequalities on Lipschitz density functions. This will be discussed at
% length in Section~\ref{sec:contributions}. In fact, using most
% standard definitions of sparse cuts and eigenfunctions of denstiy
% functions, the Cheeger inequalities are false. Thus, our key
% contribution is not only new mathematical foundations for Cheeger
% inequalities on probability density functions, but also the appropriate
% definitions of quantities like sparse cuts and eigenfunction in this
% setting (which are necessary for the inequalities to hold). For a list of
% mathematical tools necessary to build these inequalities in the
% probability density setting, see Section~\ref{sec:contributions}.

\subsection{Contributions}

Our paper has three core contributions: 
\begin{enumerate}
  \item A natural method for cutting probability density
    functions, based on a new notion of sparse cuts on density functions.
  \item A Cheeger and Buser inequality for probability density
    functions, and
  \item A clustering algorithm operating on samples, which heuristically
    approximates a spectral sweep cut on the density function when the
    number of samples grows large.
\end{enumerate}
We emphasize that our primary contributions are points 1 and 2, which
are formally stated in
Theorems~\ref{thm:sweep-cut} and~\ref{thm:Cheeger-Buser} respectively.  
Our clustering algorithm on samples, which is designed to approximate
the $(\alpha, \gamma)$-spectral sweep cut on the density function as the number of samples
grows large, is of secondary importance. 

We now state our two main theorems. Unfamiliar terms like 
$(\alpha, \beta)$-sparsity, $(\alpha, \gamma)$-spectral sweep cuts, and
$(\alpha,\beta)$-principal eigenvalue are defined in Section~\ref{sec:definitions}.
\begin{theorem} \label{thm:sweep-cut} 
  \textbf{Spectral Sweep Cuts give Sparse Cuts:}

  Let $\rho:\Re^d \to \Re_{\geq 0}$ be an $L$-Lipschitz probability
  density function, and let $\beta = \alpha+1$ and $\gamma = \alpha+2$.
  
  The $(\alpha,\gamma)$-spectral sweep cut of $\rho$ has 
  $(\alpha, \beta)$ sparsity $\Phi$ satisfying:
  \[ 
  \Phi_{OPT} \leq \Phi \leq O(\sqrt{dL\Phi_{OPT}} ).
  \]
  Here, $\Phi_{OPT}$ refers to the optimal $(\alpha,\beta)$ sparsity of
  a cut on $\rho$. 
\end{theorem}

In words, the spectral sweep cut of the
  $(\alpha, \gamma)$  eigenvector gives a provably good approximation to
  the sparsest $(\alpha, \beta)$ cut, as long as $\beta = \alpha+1$ and
  $\gamma = \alpha+2$.  
  Proving this result is a straightforward application of two new
  inequalities we present, which we will refer to as as the Cheeger and Buser inequalities
  for probability density functions.

\begin{theorem}\label{thm:Cheeger-Buser}
  \textbf{Probability Density Cheeger and Buser:}

  Let $\rho:\mathbb{R}^d \rightarrow \mathbb{R \geq 0}$ be an $L$-Lipschitz
  density function. Let $\alpha = \beta - 1 = \gamma - 2$.

  Let $\Phi$ be the infimum $(\alpha,\beta)$-sparsity of a cut through
  $\rho$, and let $\lambda_2$ be the $(\alpha,\gamma)$-principal eigenvalue of
  $\rho$. Then:
  \[ \Phi^2/4 \leq \lambda_2 \]
  and 
  \[\lambda _2 \leq O_{\alpha, \beta}(d \max(L \Phi, \Phi^2)).\]
  The first inequality is \textbf{Probability Density Cheeger}, and
  the second inequality is \textbf{Probability Density Buser}.
\end{theorem}
Note that we don't need $\rho$ to have a total mass of $1$ for any
of our proofs. The overall probability mass of $\rho$ can be arbitrary.

Finally, we give a discrete algorithm~\textsc{1,3-SpectralClutering} for clustering data points into
two-clusters.
We conjecture, but do not prove, that
\textsc{1,3-SpectralClustering} converges to the the $(\alpha=1,
\gamma=3)$-spectral sweep cut of the probability density function $\rho$
as the number of samples grows large.

\begin{algorithm}
  \textsc{1,3-SpectralClustering}

  \textbf{Input:} Point $s_1, \ldots s_n \in \mathbb{R}^d$, and similarity measure
  $K:\mathbb{R}^d, \mathbb{R}^d\rightarrow \mathbb{R}$.
  \begin{enumerate}
    \item Form the affinity matrix $A' \in \mathbb{R}^{n \times n}$,
      where $A'_{ij} = K(s_i, s_j)$ for $i \not= j$ and $A_{ii} = 0$ for
      all $i$.
    \item Define $D$ to be the diagonal matrix whose $(i, i)$ element is
      the sum of $A$'s $i^{th}$. Let $L$ be the Laplacian formed
      from the adjacency matrix $D^{1/2}AD^{1/2}$.
    \item Let $u$ be the principal eigenvector of $L$. Find the value
      $t$ where $t := \argmin_s \Phi_{\{u(v) > t\}}$, where $\Phi_S$ is
      the graph conductance of the cut defined by set $S$.
  \end{enumerate}
  \textbf{Output:} Clusters $G_1 = \{v : u(v) > t\}, G_2 = \{v : u(v)
  \leq t\}$.
\end{algorithm}
We note that this strongly resembles the unnormalized spectral
clustering based on the work of Shi and Malik~\cite{ShiMalik97} and Ng,
Weiss, and Jordan~\cite{NgSpectral01}. The major difference is that we build our
Laplacian from the matrix $D^{1/2}AD^{1/2}$ rather than $A$ (which is
the case for unnormalized spectral
clustering~\cite{von2007tutorial, TrillosVariational15}), or
$D^{-1/2}AD^{-1/2}$ (which is the case for normalized spectral
clustering~\cite{von2007tutorial, TrillosVariational15}).
%%% DEFINITION %%%%
\subsubsection{Definitions}\label{sec:definitions}
In this subsection, we define $(\alpha, \beta)$ sparsity, $(\alpha,
\gamma)$ eigenvalues/Rayleigh quotients, and $(\alpha, \gamma)$ sweep
cuts. 

\vspace{2 mm}
\begin{definition} Let $\rho$ be a probability density function with domain
  $\mathbb{R}^d$, and let $A$ be a
  subset of $\mathbb{R}^d$.

  The \textbf{$(\alpha, \beta)$-sparsity} of the cut defined by $A$ is the
  integral of
  $\rho^\beta$ over the cut, divided by the integral of
  $\rho^\alpha$ on the side of the cut where this integral is
  smaller.

\end{definition}
\vspace{2 mm}

\begin{definition}
The \textbf{$(\alpha, \gamma)$-Rayleigh quotient} of $u$ with
respect to $\rho:\Re^d \to \Re_{\geq 0}$ is:

\[
  R_{\alpha, \gamma}(u) \coloneqq \frac{\int_{\Re^d} \rho^\gamma
  |\nabla u|^2}{\int_{\Re^d} \rho^{\alpha}|u|^2} 
\]

A \textbf{$(\alpha, \gamma)$-principal eigenvalue} of $\rho$ is
$\lambda_2$, where:

\[ \lambda_2 := \inf_{\int \rho^\alpha u = 0} R_{\alpha, \gamma}(u). \]

Define a \textbf{$(\alpha, \gamma)$-principal eigenfunction} of
$\rho$ to be a function $u$ such that $R_{\alpha, \gamma}(u) =
\lambda_2$.
\end{definition}
\vspace{2 mm}

Now we define a sweep cut for a given function with respect to a
a positive valued function supported on $\mathbb{R}^d$:

\vspace{2 mm}
\begin{definition} Let $\alpha, \beta$ be two real numbers, and $\rho$ be
  any function from $\Re^d$ to $\Re_{\geq 0}$.
  Let $u$ be any function from $\Re^d \to \Re$, and let $C_t$ be the cut
  defined by the set $\{s \in \Re^d | u(s) > t\}$. 
  
  The \textbf{sweep-cut} algorithm
  for $u$ with respect to $\rho$ returns the cut $C_t$ of minimum $(\alpha,
  \beta)$ sparsity, where this sparsity is measured with respect to $\rho$.

When $u$ is a $(\alpha, \gamma)$-principal eigenfunction, the sweep cut is called
a \textbf{$(\alpha, \gamma)$-spectral sweep cut} of $\rho$.  
\end{definition}
\subsubsection{Additional Definitions}
A function $\rho: \Re^d \to \Re_{\geq 0}$ is
\textbf{$L$-Lipschitz} if $|\rho(x)-\rho(y)|_2 \leq L|x-y|_2$ for
all $x, y \in \Re^d.$

A function is $\rho:\Re^d \to \Re_{\geq 0}$ is
\textbf{$\alpha$-integrable} if $\int_{\Re^d} \rho^\alpha$ is
well defined and finite. Throughout this paper, we assume $\rho$
is always $\alpha$-integrable.

% \subsection{Do $(\alpha, \beta)$-sparse cuts give meaningful cuts of a
% density function?}
% 
% In this section, we suggest that a $(\alpha, \beta)$ sparse cut captures
% machine learning intuition that two data points drawn from the same side of
% the cut are more similar, and two data points drawn from opposite sides of
% the cut are less similar.
% 
% According to machine learning intuition, the ideal cut of a density
% function should partition it into two pieces of relatively high
% probability mass, while cutting through a low amount of probability
% mass.  
% 
% We note that a $(\alpha=1,\beta=2)$-sparse cut agrees with the machine
% learning intuition that an ideal cut should cut through regions
% of low probability density, while splitting the density function
% into two regions of comparatively large density. Moreover,
% choosing $\beta$ to be $2$ instead of $1$
% biases towards longer cuts through lower density regions,
% which agrees with machine learning intuition and practice. We
% will detail the advantages and disadvantages of a $(1,2)$-sparse cut in
% Section~\ref{sec:changing-beta}.


%These inequalities critically depend on the right notion of $(\alpha, \beta)$-sparsity and the principal $(\alpha, \gamma)$-eigenfunctions, and the right relations between $\alpha, \beta$, and $\gamma$.


% The contributions of our paper can be summed up as follows: first, we define the notion of an $(\alpha, \beta)$-sparse cut for a probability density function, which captures the notion of a cut that separates the density into two pieces of high probability mass while cutting through a region of low probability mass. We will then show that $(1,2)$-sparse cuts satisfy this intuition, and have additional desirable properties in machine learning.  Next, we give a spectral clustering-based algorithm that generates a cut of provably low $(1,2)$-sparsity for all Lipschitz probability density functions, as the number of samples grows large.  To prove this, we will state and prove a Cheeger and Buser inequality on probability density functions, which depends critically on the right notion of $(\alpha, \beta)$-sparsity and principal $(\alpha, \beta)$ eigenvalue.  Finally, we show that classical spectral clustering algorithms (like those of Ng.  et al) generate a cut of bad $(\alpha, \beta)$-sparsity for simple $1$-Lipschitz distributions, for any $\alpha, \beta > 0$.

% We note that our Cheeger and Buser inequalities on density functions lead to a natural new conjecture inspired by common machine learning models: is there a Cheeger and Buser inequality for density functions supported on manifolds?  Modeling data as samples from a probability density on a manifold is one of the most influential models in machine learning, and such a theorem would generalize our inequalities and the existing Cheeger/Buser inequalities on manifolds.
%  More generally, we will present clustering algorithms that converge to cuts of provably low $(\alpha, \alpha+1)$-sparsity on the underlying density.  

% % !TeX root = main.tex

\subsection{Definitions and Preliminaries} % (fold)
\label{sec:definitions}
In this section, we establish additional definitions for this chapter. These
are mostly of interest for our spanner and persistent homology results, and are not strictly
necessary for Theorem~\ref{thm:NN}.

%  \tim{Anything on wireless networks here? Or other prelims?}
\vspace{3 mm}

\noindent \textbf{Spanners:} For real value $t \geq 1$, a $t$-spanner of
a weighted graph $G$ is a subgraph $S$ such that $d_G(x,y) \leq d_S(x,y)
\leq t\cdot d_G(x,y)$ where $d_G$ and $d_S$ represent the shortest path
distance functions between vertex pairs in $G$ and $S$. Spanners
of Euclidean distances, and general graph distances, have been
studied extensively, and their importance as a data structure is
well established.
~\cite{Chew1986, Vaidya1991, Callahan1993,HarPeled13}.

\vspace{3 mm}
\noindent \textbf{$k$-nearest neighbor graphs:} The $k$-nearest neighbor graph
($k$-NN graph) for a set of objects $V$ is a graph with vertex set $V$
and an edge from $v\in V$ to its $k$ most similar objects in $V$, under
a given distance measure. In this chapter, the underlying distance
measure is Euclidean, and the edge weights are Euclidean distance
squared.
$k$-NN
graph constructions are a key data structure in machine
learning~\cite{Dong11, Chen11}, clustering~\cite{vL09}, and manifold learning~\cite{tenenbaum00global}.

\vspace{3 mm}
\noindent \textbf{Gabriel Graphs:} The Gabriel graph is a graph where
two vertices $p$ and $q$ are joined by an edge if and only if the disk
with diameter $pq$ has no other points of $S$ in the interior. The
Gabriel graph is a subgraph of the Delaunay
triangulation~\cite{SridharMaster}, and a
$1$-spanner of the edge-squared metric~\cite{SridharMaster}. Gabriel
graphs will be used in the proof of
Theorem~\ref{thm:distribution-spanner}.

\vspace{3 mm}
\noindent \textbf{Persistent Homology:}
  Persistent homology is a popular tool in computational geometry and topology to ascribe quantitative topological invariants to spaces that are stable with respect to perturbation of the input.
  In particular, it's possible to compare the so-called persistence diagram of a function defined on a sample to that of the complete space~\cite{chazal08towards}.
  These two aspects of persistence theory---the intrinsic nature of topological invariants and the ability to rigorously compare the discrete and the continuous---are both also present in our theory of nearest neighbor metrics.
  Indeed, our primary motivation for studying these metrics was to use them as inputs to persistence computations for problems such as persistence-based clustering~\cite{chazal13persistence} or metric graph reconstruction~\cite{aanjaneya12metric}.

% section definitions (end)

% \input{introduction/theorem}
\subsection{Past Work}\label{sec:past-work}

\subsubsection{Cheeger and Buser Inequalities for Graphs and Manifolds}

Cheeger and Buser inequalities for graphs are the foundation of spectral
graph theory~\cite{ChungBook97}. These inequalities were first
discovered by Alon and Millman~\cite{AlonM84} based on similar
inequalities in the manifold setting~\cite{Cheeger70, Buser82}. These
inequalities have been applied to graph partitioning,
random walks, and spectral graph theory~\cite{ChungBook97,
kw16, Orecchia08, Louis12, Lee2014, Orecchia2011, 
  SpielmanTeng2004}. In the graph setting, the Buser inequality is
  trivial~\cite{ChungBook97}, while the Cheeger inequality is
  mathematically substantial~\cite{AlonM84}.

  Cheeger inequality for manifolds have been extensively used in
  Riemannian geometry~\cite{Cheeger70, belkin2004semisup,
  belkin2005towards}.  Buser's inequality on manifolds is 
  mathematically non-trivial, unlike the graph case. This inequality
  depends on a Ricci curvature term, and it is false if the manifold has
  unbounded Ricci curvature~\cite{Buser82, ledoux2004spectral}. This
  inequality has been applied to diffusion
  processes on Manifolds~\cite{ledoux2004spectral}, machine
  learning~\cite{belkin2004semisup, grady2006isoperimetric}, and more~\cite{ledoux2004spectral}.

  For formal definitions of Cheeger and Buser inequalities for graphs
  and manifolds, refer to~\cite{AlonM84} and~\cite{Buser82}
  respectively.

\subsubsection{Eigenvalues, Sweep Cuts, and Isoperimetry on Probability
Densities}\label{sec:past-prob}

Recently, eigenvalues and sparse cuts have been used in the probability
density setting~\cite{Lee18, Lee18survey}, in connection with the
Kannan-Lovasz-Simonovits conjecture. There is a Cheeger and Buser inequality in
this setting, as long as strong parametric assumptions (such as log
concavity) are given~\cite{Lee18survey}. These inequalities use
what we call $(\alpha=1, \gamma=1)$ eigenvectors, and $(\alpha=1, \beta=1)$
isoperimetric constants.  
We will show in our paper that any Buser
inequality using $(\alpha = 1, \beta = 1, \gamma=1)$ must fail for
simple Lipschitz densities.
No Cheeger-Buser inequality was previously known for
probability densities without strong
parametric assumptions.

In another line of work by Von Luxburg et al and Rosasco et
al~\cite{von2008consistency,rosasco2010learning}, the authors use
perturbation theory results to show that the second eigenvalue of
a graph Laplacian generated from samples on a probability
density converges to what we call the $(\alpha = 1, \gamma =
2)$-principal eigenvalue of the density. Trillos et
al.~\cite{TrillosRate15,TrillosVariational15} improved the convergence
rate and showed that the (extensions of the) eigenvectors of the graph
Laplacian approach the eigenfunctions of the weighted Laplacian
operator for a probability density. 

These results show that
spectral clustering algorithms like the one in
\cite{NgSpectral01} can be thought of as taking an iid sample
from a distribution, constructing a graph Laplacian, and computing its
fundamental eigenvector as an approximation for finding the
eigenfunction over the original
distribution. The eigenfunction that they end up approximating is
the $(\alpha = 1, \gamma = 2)$ eigenfunction. The clustering algorithm
then takes a sweep cut with respect to this eigenfunction.

Unfortunately, sweep cut
algorithms based on the $(1, 2)$ eigenfunction can produce cuts
of probability densities with bad isoperimetry properties. See
Theorem~\ref{thm:counterexample}. The strength of our Cheeger and Buser
inequalities are that they will imply new sweep cut algorithms on
probability densities, with provably good isoperimetry.
% \textbf{Spectral Clustering and Sweep Cut Algorithms on Data}
% 
% The spectral clustering algorithms of Shi and Malik~\cite{ShiMalik97}
% and those of Ng, Jordan, and Weiss~\cite{NgSpectral01} are some of the
% most popular clustering algorithms on data (over 10,000 citations).
% If we want to split data points into two clusters, their algorithm works
% as follows: for $n$ data points, compute an $n \times n$ matrix $M$ on
% the data, and compute the principal eigenvector $e$ of the matrix. Then, find a
% threshold value $t$ such that all points $p$ where $e(p) \leq t$
% are considered to be on one side of the cut, and all other points
% where $e(p)  > t$ are on the other. Often, the matrix is a
% Laplacian matrix of some graph built from the
% data~\cite{von2007tutorial}.  
% 
% Von Luxburg, Belkin, and Bosquet~\cite{von2008consistency} proved that
% if the data is modeled as $n$ i.i.d samples from a probability density
% $\rho$, the matrix $M$ is a Laplacian matrix with certain structural
% assumptions, and certain regularity
% assumptions on $\rho$ hold, then classical spectral clustering
% algorithms converge to a $(\alpha = 1, \gamma = 2)$-spectral sweep cut
% on $\rho$~\footnote{We note that these authors used different terminology to describe this
% result, as their papers did
% not define $(\alpha, \gamma)$-spectral sweep cuts.}. These results were refined in~\cite{rosasco2010learning,
% TrillosRate15, TrillosVariational15}. We note that there are no sparsity
% guarantees known for a $(\alpha = 1, \gamma=2)$-spectral sweep cut, and
% we show $1$-Lipschitz examples of $\rho$ where this spectral sweep cut leads to undesirable
% behavior.
% 

\subsection{Theorems}

%Unfamiliar terms like $(\alpha, \beta)$-sparsity, $(\alpha, \gamma)$-spectral sweep cuts, and $(\alpha,\beta)$-principal eigenvalue are defined in Section~\ref{sec:definitions}.

\begin{theorem}\label{thm:Cheeger-Buser}
  \textbf{Probability Density Cheeger and Buser:}

  Let $\rho:\mathbb{R}^d \rightarrow \mathbb{R \geq 0}$ be an $L$-Lipschitz
  density function. Let $\alpha = \beta - 1 = \gamma - 2$.

  Let $\Phi$ be the infimum $(\alpha,\beta)$-sparsity of a cut through
  $\rho$, and let $\lambda_2$ be the $(\alpha,\gamma)$-principal eigenvalue of
  $\rho$. Then:
  \[ \Phi^2/4 \leq \lambda_2 \]
  and 
  \[\lambda _2 \leq O_{\alpha, \beta}(d \max(L \Phi, \Phi^2)).\]
  The first inequality is \textbf{Probability Density Cheeger}, and
  the second inequality is \textbf{Probability Density Buser}.
\end{theorem}
In particular, a Cheeger and Buser inequality exist when $\alpha = 1,
\beta = 2, \gamma=3$. Note that we don't need $\rho$ to have a total mass of $1$ for any
of our proofs. The overall probability mass of $\rho$ can be arbitrary.


Theorem~\ref{thm:Cheeger-Buser} has partial converses:

\begin{lemma} \label{lem:cheeger-converse} If $\alpha + \gamma >
  2\beta$, the Cheeger inequality in Theorem~\ref{thm:Cheeger-Buser}
  does not hold.
\end{lemma}

\begin{lemma} \label{lem:buser-converse} If $\gamma \geq 1$ and $\gamma -1 < \beta$, then the
  Buser inequality in Theorem~\ref{thm:Cheeger-Buser} does not
  hold.
\end{lemma}

In particular, if $\alpha = \beta = \gamma = 1$, the Buser inequality
fails.  If $\alpha = 1, \gamma =2$, no
Cheeger-Buser inequality can hold for any $\beta$. These settings
encompass most past work on sparse cuts and eigenvectors in probability
densities, as mentioned in Section~\ref{sec:past-prob}.

We apply these inequalities to show that spectral sweep cuts of
Lipschitz probability density functions give sparse cuts of the density.
This contrasts with past work on sweep cuts in probability
densities, as mentioned in Section~\ref{sec:past-prob}.

\begin{theorem} \label{thm:sweep-cut} 
  \textbf{Spectral Sweep Cuts give Sparse Cuts:}

  Let $\rho:\Re^d \to \Re_{\geq 0}$ be an $L$-Lipschitz probability
  density function, and let $\alpha = \beta - 1 = \gamma - 2$.
  
  The $(\alpha,\gamma)$-spectral sweep cut of $\rho$ has 
  $(\alpha, \beta)$ sparsity $\Phi$ satisfying:
  \[ 
  \Phi_{OPT} \leq \Phi \leq O(\sqrt{dL\Phi_{OPT}} ).
  \]
  Here, $\Phi_{OPT}$ refers to the optimal $(\alpha,\beta)$ sparsity of
  a cut on $\rho$. 
\end{theorem}

In words, the spectral sweep cut of the
  $(\alpha, \gamma)$  eigenvector gives a provably good approximation to
  the sparsest $(\alpha, \beta)$ cut, as long as $\beta = \alpha+1$ and
  $\gamma = \alpha+2$.   We will also show that this theorem is not possible
  for past definitions of eigenvectors and sparsity on probability
  density functions.

