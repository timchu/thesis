\section{Introduction}

The Cheeger and Buser inequalities relate isoperimetric cuts with eigenvalues.
Up to a constant factor, the Cheeger inequality lower bounds the
fundamental eigenvalue of a Laplacian with the square of
the isoperimetric constant~\cite{Cheeger70, AlonM84}, and the Buser inequality upper
bounds the eigenvalue with the isoperimetric constant~\cite{Buser82, AlonM84}.

These inequalities appear in two settings: the graph
setting~\cite{AlonM84} and the manifold setting~\cite{Cheeger70,
Buser82}.  In graphs, these inequalities are the foundation of spectral
graph theory~\cite{ChungBook97, AlonM84}. Spectral graph theory
has provided surprising insights on problems including maximum
flow~\cite{CKMST}, fast Laplacian solving~\cite{KMP}, expander
decompositions~\cite{wulff17expander, sw19expander}, sparse
cuts~\cite{AlonM84, arv04, chawla05sparse}, and longstanding
mathematical conjectures like the Kadison Singer conjecture~\cite{MSS}.
In this setting, the Buser inequality is trivial, and the Cheeger
inequality is mathematically substantial~\cite{AlonM84,ChungBook97}.  In
manifold theory, the Cheeger and Buser inequalities have proven useful
for probability theory on manifolds~\cite{ledoux2004spectral}, machine
learning~\cite{belkin2004semisup}, and more~\cite{belkin2005towards, ledoux2004spectral}.
Unlike in the graph setting, the Buser inequality on manifolds is highly
nontrivial, and historically came as a surprise to manifold
theorists~\cite{Buser82, ledoux2004spectral}.

This chapter introduces Cheeger and Buser inequalities in a new setting:
Lipschitz probability density functions. 
Here, a
probability density function refers to a function $\rho: \RR^d
\rightarrow \RR_{\geq 0}$ where $\int_{\RR^d} \rho = 1$. A Lipschitz
probability density refers to a probability density $\rho$ where
$|\rho(x)-\rho(y)| < L\|x-y\|_2$ for some constant $L$.
Cheeger inequalities
have been used in the probability density setting in the
past~\cite{Lee18survey}, but primarily in the setting where there are
strong parametric assumptions on the density (such as when the density
is Gaussian or log-concave)~\cite{Lee18survey}. 
We show that for general Lipschitz probability densities,  
either the Cheeger or the Buser inequality must fail using past
definitions of eigenvalue and isoperimetric constant.  

Since past definitions of eigenvalue and isoperimetric constant are
inadequate in the probability density setting, we present new
definitions of these quantities for which a Cheeger and Buser inequality
will hold. Using our new definitions, the Cheeger inequality can be
proven using standard techniques. Akin to the manifold setting, the
Buser inequality here is more difficult to prove. New mathematical ideas
are required to prove the Buser inequality. We prove both the
Cheeger and Buser inequalities on Lipschitz probability densities, using
our new definitions.

\subsubsection{Applications}
We will use our inequalities to show that a spectral sweep cut of a
Lipschitz
probability density functions will partition it into two pieces with
good sparsity guarantees.  We then discuss potential applications to
machine learning and spectral clustering. 

Spectral clustering~\cite{ShiMalik97,
NgSpectral01} is one of the most widely used techniques in machine
learning~\cite{von2007tutorial}. It is known to have a close connection to
past definitions for eigenvalues of a probability density
function~\cite{TrillosVariational15}. However, spectral clustering is
not known to have any good theoretical sparsity guarantees on partition quality as the number
of samples grows large,
in part due to the lack of a Cheeger and Buser inequality for past
definitions of eigenvalues.  Our new Cheeger and Buser inequalities may
motivate theoretically principled spectral clustering methods.

\subsection{Definitions}\label{sec:definitions}
To establish our Cheeger and Buser inequalities, we define new
notions of isoperimetric constant (equivalently, sparsity), Rayleigh
quotients, eigenvalues, eigenvectors, and sweep cuts. Appropriate
definitions will let us establish basic spectral theory in the Lipschitz
probability density setting.

\vspace{2 mm}

\begin{definition} Let $\rho$ be a probability density function with domain
  $\mathbb{R}^d$, and let $A$ be a
  subset of $\mathbb{R}^d$.

  The \textbf{$(\alpha, \beta)$-sparsity} of the cut induced by $A$ is
  denoted by $\Phi(A)$. It is defined as $(d-1)$ dimensional integral of $\rho^\beta$ on the cut, divided
  by the $d$ dimensional integral of $\rho^\alpha$ on the side of the
  cut where this integral is smaller.  \end{definition}

For nice smooth cuts this
  intuitive definition is fine but a more general and precise
  definition using total variation is given in
  definition~\ref{def:betaBdy}

  \begin{definition}
The \textbf{$(\alpha, \beta)$-isoperimetric constant} of
  $\rho$ is defined as the infimum of $\Phi(A)$ over all $A$.
  \end{definition}

\vspace{2 mm}

\begin{definition}
The \textbf{$(\alpha, \gamma)$-Rayleigh quotient} of $u$ with
respect to $\rho:\Re^d \to \Re_{\geq 0}$ is:

\[
  R_{\alpha, \gamma}(u) \coloneqq \frac{\int_{\Re^d} \rho^\gamma
  |\nabla u|^2}{\int_{\Re^d} \rho^{\alpha}|u|^2} 
\]

A \textbf{$(\alpha, \gamma)$-principal eigenvalue} of $\rho$ is
$\lambda_2$, where:

\[ \lambda_2 := \inf_{\int \rho^\alpha u = 0} R_{\alpha, \gamma}(u). \]

Define a \textbf{$(\alpha, \gamma)$-principal eigenfunction} of
$\rho$ to be a function $u$ such that $R_{\alpha, \gamma}(u) =
\lambda_2$, if such $u$ exists.
\end{definition}
\vspace{2 mm}

Now we define a sweep cut for a given function with respect to a
a positive valued function supported on $\mathbb{R}^d$:

\vspace{2 mm}
\begin{definition} Let $\alpha, \beta$ be two real numbers, and $\rho$ be
  any function from $\Re^d$ to $\Re_{\geq 0}$.
  Let $u$ be any function from $\Re^d \to \Re$, and let $C_{t, u}$ be the cut
  defined by the set $\{s \in \Re^d \; | \; u(s) > t\}$. 
  
  The \textbf{sweep-cut} algorithm
  for $u$ with respect to $\rho$ returns the cut $C_{t,u}$ of minimum $(\alpha,
  \beta)$ sparsity, where this sparsity is measured with respect to $\rho$.

When $u$ is a $(\alpha, \gamma)$-principal eigenfunction, the sweep cut is called
a \textbf{$(\alpha, \gamma)$-spectral sweep cut} of $\rho$.  
\end{definition}

\textbf{Additional Definitions:}

A function $\rho: \Re^d \to \Re_{\geq 0}$ is
\textbf{$L$-Lipschitz} if $|\rho(x)-\rho(y)|_2 \leq L|x-y|_2$ for
all $x, y \in \Re^d.$

A function is $\rho:\Re^d \to \Re_{\geq 0}$ is
\textbf{$\alpha$-integrable} if $\int_{\Re^d} \rho^\alpha$ is
well defined and finite. Throughout this chapter, we assume $\rho$
is always $\alpha$-integrable.

\vspace{3 mm}
Our definitions depend on three constants: $\alpha$, $\beta$, and
$\gamma$. Informally, these constants can be thought of as the mass
constant, the cut constant, and the spring constant respectively. In the
graph and manifold setting, the spring and cut constant are the same. One key
contribution of our definitions is the decoupling of the cut and spring
constants which will allow us to get tight Cheeger and Buser results.



\subsection{Theorems}

%Unfamiliar terms like $(\alpha, \beta)$-sparsity, $(\alpha, \gamma)$-spectral sweep cuts, and $(\alpha,\beta)$-principal eigenvalue are defined in Section~\ref{sec:definitions}.

\begin{theorem}\label{thm:Cheeger-Buser}
  \textbf{Probability Density Cheeger and Buser:}

  Let $\rho:\mathbb{R}^d \rightarrow \mathbb{R \geq 0}$ be an $L$-Lipschitz
  density function. Let $\alpha = \beta - 1 = \gamma - 2$.

  Let $\Phi$ be the infimum $(\alpha,\beta)$-sparsity of a cut through
  $\rho$, and let $\lambda_2$ be the $(\alpha,\gamma)$-principal eigenvalue of
  $\rho$. Then:
  \[ \Phi^2/4 \leq \lambda_2 \]
  and 
  \[\lambda _2 \leq O_{\alpha, \beta}(d \max(L \Phi, \Phi^2)).\]
  The first inequality is \textbf{Probability Density Cheeger}, and
  the second inequality is \textbf{Probability Density Buser}.
\end{theorem}
In particular, a Cheeger and Buser inequality exist when $\alpha = 1,
\beta = 2, \gamma=3$. Note that we don't need $\rho$ to have a total mass of $1$ for any
of our proofs. The overall probability mass of $\rho$ can be arbitrary.


Theorem~\ref{thm:Cheeger-Buser} has partial converses:

\begin{lemma} \label{lem:cheeger-converse} If $\alpha + \gamma >
  2\beta$, the Cheeger inequality in Theorem~\ref{thm:Cheeger-Buser}
  does not hold.
\end{lemma}

\begin{lemma} \label{lem:buser-converse} If $\gamma \geq 1$ and $\gamma -1 < \beta$, then the
  Buser inequality in Theorem~\ref{thm:Cheeger-Buser} does not
  hold.
\end{lemma}

In particular, if $\alpha = \beta = \gamma = 1$, the Buser inequality
fails.  If $\alpha = 1, \gamma =2$, no
Cheeger-Buser inequality can hold for any $\beta$. These settings
encompass most past work on sparse cuts and eigenvectors in probability
densities, as mentioned in Section~\ref{sec:past-prob}.

We apply these inequalities to show that spectral sweep cuts of
Lipschitz probability density functions give sparse cuts of the density.
This contrasts with past work on sweep cuts in probability
densities, as mentioned in Section~\ref{sec:past-prob}.

\begin{theorem} \label{thm:sweep-cut} 
  \textbf{Spectral Sweep Cuts give Sparse Cuts:}

  Let $\rho:\Re^d \to \Re_{\geq 0}$ be an $L$-Lipschitz probability
  density function, and let $\alpha = \beta - 1 = \gamma - 2$.
  
  The $(\alpha,\gamma)$-spectral sweep cut of $\rho$ has 
  $(\alpha, \beta)$ sparsity $\Phi$ satisfying:
  \[ 
  \Phi_{OPT} \leq \Phi \leq O(\sqrt{dL\Phi_{OPT}} ).
  \]
  Here, $\Phi_{OPT}$ refers to the optimal $(\alpha,\beta)$ sparsity of
  a cut on $\rho$. 
\end{theorem}

In words, the spectral sweep cut of the
  $(\alpha, \gamma)$  eigenvector gives a provably good approximation to
  the sparsest $(\alpha, \beta)$ cut, as long as $\beta = \alpha+1$ and
  $\gamma = \alpha+2$.   We will also show that this theorem is not possible
  for past definitions of eigenvectors and sparsity on probability
  density functions.

\subsection{Past Work}\label{sec:past-work}

\subsubsection{Cheeger and Buser Inequalities for Graphs and Manifolds}

Cheeger and Buser inequalities for graphs are the foundation of spectral
graph theory~\cite{ChungBook97}. These inequalities were first
discovered by Alon and Millman~\cite{AlonM84} based on similar
inequalities in the manifold setting~\cite{Cheeger70, Buser82}. These
inequalities have been applied to graph partitioning,
random walks, and spectral graph theory~\cite{ChungBook97,
kw16, Orecchia08, Louis12, Lee2014, Orecchia2011, 
  SpielmanTeng2004}. In the graph setting, the Buser inequality is
  trivial~\cite{ChungBook97}, while the Cheeger inequality is
  mathematically substantial~\cite{AlonM84}.

  Cheeger inequality for manifolds have been extensively used in
  Riemannian geometry~\cite{Cheeger70, belkin2004semisup,
  belkin2005towards}.  Buser's inequality on manifolds is 
  mathematically non-trivial, unlike the graph case. This inequality
  depends on a Ricci curvature term, and it is false if the manifold has
  unbounded Ricci curvature~\cite{Buser82, ledoux2004spectral}. This
  inequality has been applied to diffusion
  processes on Manifolds~\cite{ledoux2004spectral}, machine
  learning~\cite{belkin2004semisup, grady2006isoperimetric}, and more~\cite{ledoux2004spectral}.

  For formal definitions of Cheeger and Buser inequalities for graphs
  and manifolds, refer to~\cite{AlonM84} and~\cite{Buser82}
  respectively.

\subsubsection{Eigenvalues, Sweep Cuts, and Isoperimetry on Probability
Densities}\label{sec:past-prob}

Recently, eigenvalues and sparse cuts have been used in the probability
density setting~\cite{Lee18, Lee18survey}, in connection with the
Kannan-Lovasz-Simonovits conjecture. There is a Cheeger and Buser inequality in
this setting, as long as strong parametric assumptions (such as log
concavity) are given~\cite{Lee18survey}. These inequalities use
what we call $(\alpha=1, \gamma=1)$ eigenvectors, and $(\alpha=1, \beta=1)$
isoperimetric constants.  
We will show in our paper that any Buser
inequality using $(\alpha = 1, \beta = 1, \gamma=1)$ must fail for
simple Lipschitz densities.
No Cheeger-Buser inequality was previously known for
probability densities without strong
parametric assumptions.

In another line of work by Von Luxburg et al and Rosasco et
al~\cite{von2008consistency,rosasco2010learning}, the authors use
perturbation theory results to show that the second eigenvalue of
a graph Laplacian generated from samples on a probability
density converges to what we call the $(\alpha = 1, \gamma =
2)$-principal eigenvalue of the density. Trillos et
al.~\cite{TrillosRate15,TrillosVariational15} improved the convergence
rate and showed that the (extensions of the) eigenvectors of the graph
Laplacian approach the eigenfunctions of the weighted Laplacian
operator for a probability density. 

These results show that
spectral clustering algorithms like the one in
\cite{NgSpectral01} can be thought of as taking an iid sample
from a distribution, constructing a graph Laplacian, and computing its
fundamental eigenvector as an approximation for finding the
eigenfunction over the original
distribution. The eigenfunction that they end up approximating is
the $(\alpha = 1, \gamma = 2)$ eigenfunction. The clustering algorithm
then takes a sweep cut with respect to this eigenfunction.

Unfortunately, sweep cut
algorithms based on the $(1, 2)$ eigenfunction can produce cuts
of probability densities with bad isoperimetry properties. See
Theorem~\ref{thm:counterexample}. The strength of our Cheeger and Buser
inequalities are that they will imply new sweep cut algorithms on
probability densities, with provably good isoperimetry.
% \textbf{Spectral Clustering and Sweep Cut Algorithms on Data}
% 
% The spectral clustering algorithms of Shi and Malik~\cite{ShiMalik97}
% and those of Ng, Jordan, and Weiss~\cite{NgSpectral01} are some of the
% most popular clustering algorithms on data (over 10,000 citations).
% If we want to split data points into two clusters, their algorithm works
% as follows: for $n$ data points, compute an $n \times n$ matrix $M$ on
% the data, and compute the principal eigenvector $e$ of the matrix. Then, find a
% threshold value $t$ such that all points $p$ where $e(p) \leq t$
% are considered to be on one side of the cut, and all other points
% where $e(p)  > t$ are on the other. Often, the matrix is a
% Laplacian matrix of some graph built from the
% data~\cite{von2007tutorial}.  
% 
% Von Luxburg, Belkin, and Bosquet~\cite{von2008consistency} proved that
% if the data is modeled as $n$ i.i.d samples from a probability density
% $\rho$, the matrix $M$ is a Laplacian matrix with certain structural
% assumptions, and certain regularity
% assumptions on $\rho$ hold, then classical spectral clustering
% algorithms converge to a $(\alpha = 1, \gamma = 2)$-spectral sweep cut
% on $\rho$~\footnote{We note that these authors used different terminology to describe this
% result, as their papers did
% not define $(\alpha, \gamma)$-spectral sweep cuts.}. These results were refined in~\cite{rosasco2010learning,
% TrillosRate15, TrillosVariational15}. We note that there are no sparsity
% guarantees known for a $(\alpha = 1, \gamma=2)$-spectral sweep cut, and
% we show $1$-Lipschitz examples of $\rho$ where this spectral sweep cut leads to undesirable
% behavior.
% 

% \input{spectralcluster/introduction/theorem}
% \subsection{Differences between our work and past work}\label{sec:cheeger-buser-difficulties}
% Our work differs from past work in the following key ways:
% 
% \begin{enumerate}
% \item Our work differs from past practical work on spectral sweep cuts
% cuts~\cite{TrillosRate15, TrillosVariational15, ShiMalik97,
%   NgSpectral01}, as those methods perform what we call a $(\alpha = 1,
%     \gamma = 2)$-sweep cut. These sweep cuts have no theoretical
%     guarantees, much less a guarantee on their $(\alpha, \beta)$
%     sparsity. Lemma~\ref{lem:converse} shows that no Cheeger and Buser inequality can
%     simultaneously hold for any setting of $\beta$ when $\alpha = 1,
%     \beta = 2$. 
%     
%     We will further show that using a $(\alpha=1,
%     \gamma=2)$-sweep cut can lead to undesirable cuts of $1$-Lipschitz
%     probability densities, with poor sparsity guarantees.
% 
% \item We note that probability density Cheeger-Buser is not easily implied by
%   graph or manifold Cheeger-Buser. For a lengthier discussion on this,
%     see Appendix~\ref{app:notgraph}.
% \item We
% do not require any assumptions on our probability density
% except that it is Lipschitz. Past work on Cheeger-Buser inequality
% for densities focused on log-concave distributions, or
% mixtures thereof~\cite{Lee18survey, Ge2018}.  
% \item For our work, the probability
% density $\rho$ is not required to be bounded away from $0$.
% This is a sharp departure from many existing results: past results on
% partitioning probability densities required a positive lower bound on
% $\rho$~\cite{von2008consistency, TrillosRate15}.
% The strongest results in fields like linear elliptic partial
% differential equations depend on $\rho$ being bounded away from
% $0$ ~\cite{w17}.
% \end{enumerate}
% Our work is the first spectral sweep cut algorithm 
% that guarantees a sparse cut on Lipschitz densities $\rho$, without requiring strong
% parametric assumptions on $\rho$.

\subsubsection{Technical Contribution}
The key technical contribution of our proof is proving Buser's
inequality on Lipschitz probability densities via
mollification~\cite{s38, f44} with disks of varying radius. 
This chapter is the first time mollification with disks of varying radius
have been used. We emphasize that the most difficult part of our paper
is proving the Buser inequality.

Mollification has a long history in mathematics dating back to Sergei
Sobolev's celebrated proof of the Sobolev embedding
theorem~\cite{s38}. It is one of the key tools in numerical
analysis, partial differential equations, fluid mechanics, and functional
analysis~\cite{f44, lw01, ss09, m03}, and analogs of mollification have been
used in computational complexity settings~\cite{dkn09}.  Informally
speaking, mollification is used to
create a series of smooth functions approximating a non-smooth
function,  by  convolving the original function with a
smooth function supported on a disk.  
Notably, an approach using convolution is used by Buser in~\cite{Buser82} to
prove the original Buser's inequality, albeit with an intricate
pre-processing step on any given cut.

To prove Buser's inequality on Lipschitz probability density functions $\rho$, we
will show that given a cut $C$ with low $(\alpha, \beta)$-sparsity, we can find a function $u$ with low $(\alpha, \gamma)$-Rayleigh
quotient. We build $u$ by starting with the indicator function $I_C$ for
cut $C$ (which is $1$ on one side of the cut and $0$ on the other).
Next, we mollify this function with disks of varying radii. In
particular, for each point $r$ in the domain of $\rho$, we
spread out the point mass $I_C(r)$ over a disk of radius
proportional to $\rho(r)L$, where L is the Lipschitz constant of $\rho$.
The resulting function $u$ obtained by `spreading out' $I_C$ will have
low $(\alpha, \gamma)$-Rayleigh quotient.

For all past uses of mollification, the disks on which the smooth
convolving function is supported (we call this the mollification
    disk) have the same radius throughout
the manifold. The use of a uniform radius disk is critical for most uses and proofs in mollification.  
Our contribution is to allow the disks to vary in
radius across our density.  This variation in radius allow us to
deal with functions that approach $0$ (and explains the importance
    of the density being Lipschitz). No mollification disks
centered anywhere in our probability density will intersect
the $0$-set of the density. This overcomes significant
hurdles in many results for functional analysis and PDEs, as
many past significant results related to partial
differential equations rely on having a positive lower
bound on the density~\cite{w17, TrillosRate15}.

Proving our Buser inequality using
mollification by disks of various radius requires a fairly
delicate proof with many pages of calculus. Our key technical
lemma is a bound on how the $l_1$ norm of a mollified function
when the mollification disks have various radius, which can be
found in Section~\ref{sec:key_lemma}. 

