\subsection{Main Theorems}
We are now ready to state a formal version of
Theorem~\ref{thm:informal}.
\begin{theorem}\label{thm:formal} (Distributional Cheeger)
For an $L$-Lipschitz function $\rho: \Re^d
  \rightarrow \Re_{\geq 0}$, for variables $(\alpha, \beta, \gamma)$
  where $\beta = \alpha +1$ and $\gamma = \alpha+2$, we have:

\[ \Phi^2/4 \leq \lambda_2 \leq 3 \cdot 2^{\beta+1}d \cdot
\max(L\Phi ,2^{\beta + 1}\Phi^2). \]
  where $\lambda_2$ is the $(\alpha, \gamma)$ eigenvalue of $\rho$,
  and $\Phi$ is the $(\alpha, \beta)$-sparsity of the $(\alpha,
  \beta)$-isoperimetric
  cut for $\rho$.
\end{theorem}
When $(\alpha = 1, \beta = 2, \gamma = 3)$, we have:
\[ \Phi^2/4 \leq \lambda_2 \leq 24d 
\max(L\Phi, 8\Phi^2). \]

We denote the lower bound on $\lambda_2$ as the \textbf{Cheeger
Inequality}, as this is the bound that Cheeger
proved for manifolds~\cite{Cheeger70}. We call the upper bound on
$\lambda_2$ the \textbf{Buser inequality}
~\cite{Buser82}.
Assuming constant dimension, both the Buser inequality and the
Cheeger inequality in Theorem~\ref{thm:formal}
are tight up to constant factors. We will see this in
Section~\ref{sec:examples}.

Note that theorem~\ref{thm:formal} has a partial converse: when $\alpha, \beta,
\gamma$ are set improperly, no such Cheeger-like statement can
hold.
\begin{lemma}\label{lem:converse}
If $\gamma - 2 \leq \alpha$, then
  there exists $\rho: \Re^d \to \Re_{\geq 0}$ which is a $1$-Lipschitz function in
  $1$ dimension, such that either:
  \[
    \lambda_2 =  o(\Phi^2)
    \]
  or 
  \[ 
    \max(\Phi, \Phi^2) = o(\lambda_2).
  \]
\end{lemma}
Together, Theorem~\ref{thm:formal} and Lemma~\ref{lem:converse}
give restrictions on what $(\alpha, \beta, \gamma)$ must
equal, for both a Cheeger and Buser inequality to hold on an $L$-Lipschitz
distribution.
