\subsection{Definitions}\label{sec:definitions}
Using the upcoming definitions, we make
Theorem~\ref{thm:informal} precise. These definitions are
inspired by the definition of fundamental eigenvalues and
isoperimetric cuts for graphs and manifolds.

\subsubsection{Isoperimetric Cuts of Distributions}
Let $\alpha$ and $\beta$ be two arbitrary non-negative constants.
Let $\rho: \Re^d \rightarrow \Re_{\geq 0}$.

Let $A \subset \bbR^d$. 
Let $u\coloneqq \chi_A$, the characteristic function of $A$, i.e., $u(x) =
1$ if $x \in A$ and zero otherwise.  We say that $A$ has
\textbf{finite $\beta$-perimeter} if there exists a sequence of
functions $\{u_n\}_{n=1}^\infty \subset C^\infty(\Re^d)$ with $u_n
\rightarrow u$ in $\Lone$ for which \cite{EvansMeasure15}
\[
\abs{\boundary A}_\beta
= \lim_{n \rightarrow \infty} \int  \rho^\beta|\nabla u_n|
\eqqcolon  \int  \rho^\beta|\nabla u|
\]
exists and is finite.

Assuming $\int_{A} \rho^\alpha u \leq \int_{\bbR^d - A}
\rho^\alpha u $ (and interchanging $A$ and $B$ if otherwise), we define the
\textbf{$(\alpha, \beta)$-sparsity} of the cut defined by $A$
to be $\Phi_{\alpha, \beta}(A)$ where:
\[
  \Phi_{\alpha, \beta} (A) 
:= \frac{\int \rho^\beta |\nabla u|}{\int \rho^\alpha u}
= \lim_{n\to\infty}\frac{\int \rho^\beta\abs{\grad u_n}}{\int \rho^\alpha \abs{u_n}}.
\]
The \textbf{$(\alpha, \beta)$-isoperimetric cut} of the
distribution $\rho$ is the set $A \subset \Re^d$ that minimizes $\Phi_{\alpha,
\beta}(A)$. When $\alpha, \beta$ are clear from context, we
simply call it the isoperimetric cut and denote it as $\Phi$.
\subsubsection{Eigenvalues and $(\alpha, \gamma)$-Rayleigh Quotients of Distributions}
Define the \textbf{$(\alpha, \gamma)$-Rayleigh quotient} of $u$ with
respect to $\rho:\Re^d \to \Re_{\geq 0}$ as:

\[
  R_{\alpha, \gamma}(u) \coloneqq \frac{\int_{\Re^d} \rho^\gamma
  |\nabla u|^2}{\int_{\Re^d} \rho^{\alpha}|u|^2} 
\]

Define a \textbf{$(\alpha, \gamma)$-principal eigenvalue} of $\rho$ to be
$\lambda_2$, where:

\[ \lambda_2 := \inf_{\int \rho^\alpha u = 0} R_{\alpha, \gamma}(u). \]

When $(\alpha, \gamma)$ is clear, we simply refer to the above
terms as Rayleigh quotient and eigenvalue respectively. In this
case, we denote $R_{\alpha, \gamma}(u) = R(u)$.

Finally, define a \textbf{$(\alpha, \gamma)$-principal eigenfunction} of
$\rho$ to be a function $u$ such that $R_{\alpha, \gamma}(u) =
\lambda_2$.

We note that the eigenvalue $\lambda_2$ can also be defined in
terms of a partial differential equation: 
\begin{align*}
-\div\left(\rho^\gamma \grad f\right) = \lambda \rho^\alpha f,
\end{align*}
However, for our
purposes, the Rayleigh quotient definition suffices.

Throughout our paper, we will use the variables $\alpha, \beta,
\gamma$. It may be helpful for the reader to set these numbers to be $1, 2,$ and $3$
repsectively, as this is the simplest setting of these variables
in which we can recover our main theorems.

\subsubsection{Additional Definitions}
A function $\rho: \Re^d \to \Re_{\geq 0}$ is
\textbf{$L$-Lipschitz} if $|\rho(x)-\rho(y)|_2 \leq L|x-y|_2$ for
all $x, y \in \Re^d.$

A function is $\rho:\Re^d \to \Re_{\geq 0}$ is
\textbf{$\alpha$-integrable} if $\int_{\Re^d} \rho^\alpha$ is
well defined and finite. Throughout this paper, we assume $\rho$
is always $\alpha$-integrable.
