\section{Paper Organization}

In Section~\ref{sec:examples}, we go over example $1$-D
distributions that show that either Cheeger or Buser inequality
must fail for past definitions of sparsity and eigenfunctions.
These examples motivate our new definitions.  We will prove Lemma~\ref{lem:cheeger-converse} and
Lemma~\ref{lem:buser-converse} in this section.  

We prove the Buser inequality in Section~\ref{sec:buser},
   via a rather extensive series of calculus computations.
Our proof relies on a key technical lemma, which is presented in
Section~\ref{sec:key_lemma}.  The Buser inequality is by far the most difficult part of
our proof. 

We prove the Cheeger inequality in Section~\ref{sec:cheeger}.
The proof in this section implies that the $(\alpha, \alpha+1)$
sparsity of the $(\alpha, \alpha+2)$ spectral sweep cut of a
probability density function $\rho$ is provably close to the
$(\alpha, \alpha+2)$ principal eigenvalue of $\rho$.  We note
that this inequality does not depend on the Lipschitz nature of
the probability density function.

In Section~\ref{sec:sweep_cut}, we prove
Theorem~\ref{thm:sweep-cut}, which shows that a $(\alpha,
    \alpha+2)$ spectral sweep cut has $(\alpha, \alpha+1)$
sparsity which provably approximates the optimal $(\alpha,
    \alpha+1)$ sparsity.


In Section~\ref{sec:counterexample}, we show an example Lipschitz
probability density where the $(\alpha = 1, \gamma=2)$ spectral sweep
cut has bad $(1,\beta)$ sparsity for any $\beta < 10$, and will
lead to an undesirable cut (from a clustering point of view) on this density function.
This is important since the spectral clustering
algorithm of Ng et al~\cite{NgSpectral01} is known to converge
to a $(\alpha=1, \gamma=2)$ spectral sweep cut on the underlying
probability density function, as the number of
samples grows large~\cite{TrillosVariational15}. 

Finally, we state conclusions and open problems in Section~\ref{sec:conclusion}.

In the appendix, we note that the Cheeger and Buser inequalities for probability
densities are not easily implied by graph or manifold
Cheeger-Buser. We also provide a simplified version of Cheeger's and
Buser's inequality for probability densities, in the $1$-dimensional case. This
may make easier reading for those unfamiliar with technical
multivariable mollification.
