\section{Cheeger Inequality for Probability Density Functions}\label{sec:cheeger}

In this section, we prove the Cheeger inequality from
Theorem~\ref{thm:Cheeger-Buser}. That is a weighted Cheeger inequality in higher dimensions.
This is the easier to prove than Buser's inequality, which
contrasts with what happens in the graph case (the graph Buser
    inequality is trivial).

For a simplified proof of the Cheeger inequality for distributions
in one-dimension, see Appendix~\ref{sec:one_dim}.

As we will see from simple counterexamples in
Section~\ref{sec:examples}, the Cheeger-direction does not
hold for all setting of $(\alpha,\beta,\gamma)$. The proof we give is
requires fewer assumptions than the Buser inequality for
probability densities. One, the Cheeger inequality is
independent of the Lipschitz constant of $\rho$ and two, the proof
also holds when $\rho$ is supported on a set
$\Omega \subset \mathbb{R}^d$.

The proof is almost identical to the proof in one dimension and only a
slight modification of standard proofs The only change in the proof is
replacing the change of variables formula with a co-area formula.  Let
$\rho:\Omega \to \RR_>$ be an Lipschitz density function that is
$(\alpha,\beta,\gamma)$-integrable over an open set $\Omega \subseteq
\RR^d$.  Note a stronger hypothesis  on $\Omega$ is that it is the support of
$\rho$ when  $\rho:\RR^d \to \RR_\leq$. 

\begin{theorem}
\label{thm:cheeger_n}
Let $\rho:{\Omega}\to\RR_{>0}$ be a Lipschitz function. Then,
\begin{align*}
\Phi^2 \leq 4 \norm{\rho^{\beta - \frac{\alpha+\gamma}{2}}}^2_\infty \lambda_2.
\end{align*}
In particular, when $(\alpha,\beta,\gamma) = (1,2,3)$ we have
\begin{align*}
\Phi^{2} \leq 4\lambda_2.
\end{align*}
\end{theorem}
Here, $\Phi$ is the optimal $(\alpha,\beta)$-sparsity of a cut
through $\rho$. We note that we can say something a little stronger:
\begin{theorem}
\label{thm:cheeger-sweep}
Let $\rho:{\Omega}\to\RR_{>0}$ be a Lipschitz function. 
Let $\Phi_{(\alpha,\beta,\gamma)}$ be the $(\alpha,\beta)$
sparsity of the $(\alpha, \gamma)$ spectral sweep cut. If $\alpha = \beta-1 = \gamma-2$,  then:
\begin{align*}
\Phi_{(\alpha, \beta, \gamma)}^2 \leq 4 \lambda_2
\end{align*}
\end{theorem}
\begin{proof} (of both theorems): 
Let $w\in W^{1,2}$, functions whose gradient is square integrable, nonzero with $\int_\Omega \rho^\alpha w\,dx = 0$. Let $v = w+a1$ where $a$ is chosen such that $\abs{\set{v<0}}_\alpha = \abs{\set{v>0}}$. Note that
\begin{align*}
R(w) &= \frac{\int_\Omega \rho^\gamma \abs{\grad w}^2\,dx}{\int_\Omega \rho^\alpha w^2\,dx}\\
&\geq \frac{\int_\Omega \rho^\gamma \abs{\grad w}^2\,dx}{\int_\Omega \rho^\alpha w^2\,dx+ a^2\abs{\Omega}_\alpha}\\
&= R(v).
\end{align*}
Without loss of generality, the function $u = \max(v,0)$ satisfies $R(u)\leq R(v)$.

Let $\Omega_0 = \set{v>0}$. Let $g = u^2$. Noting that $\grad g = 2u\grad u$ a.e., we can apply Cauchy-Schwarz to obtain
\begin{align*}
\int_{\Omega_0}\rho^\beta \abs{\grad g}\,dx &= 2\int_{\Omega_0}\rho^\beta \abs{u}\abs{\grad u}\,dx\\
&\leq 2 \sqrt{\int_{\Omega_0} \rho^{2\beta - \alpha} \abs{\grad u}^2\,dx}\sqrt{\int_{\Omega_0} \rho^{\alpha} u^2\,dx}\\
&\leq 2 \norm{\rho^{\beta - \frac{\alpha+\gamma}{2}}}_\infty \sqrt{\int_{\Omega_0} \rho^\gamma \abs{\grad u}^2\,dx}\sqrt{\int_{\Omega_0} \rho^{\alpha} u^2\,dx}.
\end{align*}
Then, dividing by $\int_{\Omega_0} \rho^\alpha g\,dx$, we have
\begin{align*}
\frac{\int_{\Omega_0} \rho^\beta \abs{\grad g}\,dx}{\int_{\Omega_0} \rho^\alpha g\,dx} &\leq2 \norm{\rho^{\beta - \frac{\alpha+\gamma}{2}}}_\infty \sqrt{R(w)}.
\end{align*}
Let $A_t = \set{g>t}$. Then, by the weighted co-area formula,
\begin{align*}
\int_{\Omega_0} \rho^\beta \abs{\grad g}\,dx = \int_0^\infty \abs{\boundary A_t}_\beta\,dt.
\end{align*}
Writing $g(x) = \int_0^{g(x)} 1 \,dt$ and applying Tonelli's theorem, we rewrite the denominator
\begin{align*}
\int_{\Omega_0} \rho^\alpha g\,dx = \int_{0}^\infty \abs{A_t}_\alpha\,dt.
\end{align*}
Thus, by averaging, there exists some $t^*$ such that
\begin{align*}
\Phi &\leq \Phi(A_{t^*})\\
&\leq \frac{\int_{\Omega_0} \rho^\beta \abs{\grad g}\,dx}{\int_{\Omega_0} \rho^\alpha g\,dx}\\
&\leq2 \norm{\rho^{\beta - \frac{\alpha+\gamma}{2}}}_\infty \sqrt{R(w)}.
\end{align*}
Optimizing over the set $\set{w\in W^{1,2}\smid w\neq 0,\,\int_\Omega \rho^\alpha w\,dx = 0}$ completes the proof.
\end{proof}

