% \section{Proof of Theorem \ref{thm:buser_n}}
% \label{app:proof}

Classical calculus will be used to compute the integrals over the
boundary of the sets in Definition \ref{def:betaBdy}.  Specifically,
given a bounded convex domain $\Omega \subset \Re^d$
% with Lipschitz boundary
and a set $A \subset \Omega$ with finite perimeter, let $u\coloneqq \chi_A$, the characteristic function of $A$, i.e., $u(x) =
1$ if $x \in A$ and zero otherwise.  Then there exists a sequence of
functions $\{u_n\}_{n=1}^\infty \subset C^\infty(\Re^d)$ with $u_n
\rightarrow u$ in $\Lone$ for which \cite{EvansMeasure15}
% $$
% \int_{\partial A} \rho^\beta
% = \lim_{n \rightarrow \infty} \int_\Omega |\nabla u_n| \rho^\beta
% \eqqcolon  \int_\Omega |\nabla u| \rho^\beta.
% $$
\begin{equation} \label{eqn:Aapprox}
\abs{\boundary A}_\beta
= \lim_{n \rightarrow \infty} \int_\Omega  \rho^\beta|\nabla u_n|
\eqqcolon  \int_\Omega  \rho^\beta|\nabla u|.
\end{equation}
Interchanging $A$ and $\Omega \setminus A$ if necessary, it follows that
$\Phi(A)$ defined in Definition \ref{def:betaBdy} can be written as
% $$
% \Phi(A) 
% = \frac{\int_\Omega \rho^\beta |\nabla u|}{\int_\Omega \rho^\alpha u}
% = \frac{\lone{\rho^\beta \nabla u}}{\lone{\rho^\alpha u}}
% \qquad
% \text{ where } u = \chi_A.
% $$
$$
\Phi(A) 
= \frac{\int_\Omega \rho^\beta |\nabla u|}{\int_\Omega \rho^\alpha u}
= \lim_{n\to\infty}\frac{\int_\Omega \rho^\beta\abs{\grad u_n}}{\int_\Omega\rho^\alpha \abs{u_n}}.
$$
To prove Theorem \ref{thm:Abuser_n} we construct an approximation $u_\theta$ of
$u$ for which the numerator and denominator of the Raleigh quotient, $R(u_\theta)$,
approximate respectively the numerator and denominator of this expression.
Specifically, $u_\theta$ will constructed as a mollification of $u$,
\begin{equation} \label{eqn:Autheta}
u_\theta(x) 
\coloneqq \int_{B(0,1)} \!\!\! u(x-\theta \rho(x) y) \phi(y) \, dy
= \int_{\Re^n} \! u(z) \phi_{\theta \rho(x)}(x-z) \, dz,
\quad \text{ where } \quad
\phi_{\eta}(z) 
= \frac{1}{\eta^d} \phi\left(\frac{z}{\eta}\right),
\end{equation}
with $\theta > 0$ a parameter to be chosen and $\phi:\Re^d \rightarrow
[0,\infty)$ a smooth function supported in the unit ball $B(0,1)=\{x
\in \Re^d| |x| < 1\}$ with unit mass $\int_{\Re^d} \phi = 1$. When
$\rho$ is constant it follows from the Tonelli theorem that
$\lone{u_\theta} \leq \lone{u}$; when $\rho$ is not constant Lemma
\ref{lem:lonetheta} shows that the latter still bounds the former.


\begin{proof} (of Theorem \ref{thm:Abuser_n})
Fix $A \subset \Omega$ with $|A|_\alpha \leq |\Omega|_\alpha / 2$ and let
$u(x) = \chi_A(x)$ be the characteristic function of $A$. Setting 
$\ubar$ to be the weighted average of $u$,
$$
\ubar 
= \frac{\int_\Omega \rho^\alpha u}{\int_\Omega \rho^\alpha}
= \frac{\int_A \rho^\alpha}{\int_\Omega \rho^\alpha}
= \frac{|A|_\alpha}{|\Omega|_\alpha} \in [0,1/2],
\qquad \text{ then } \qquad
\int_\Omega \rho^\alpha (u-\ubar) = 0,
$$
and
$$
\lone{\rho^\alpha(u-\ubar)} 
= \int_\Omega \rho^\alpha |u-\ubar| = |A|_\alpha (1-\ubar)
= 2 \int_\Omega \rho^\alpha |u-\ubar|^2.
$$
Since $|A|_\alpha = \lonea{u}$ and $1-\ubar \in [0,1/2]$ it follows that
\begin{equation} \label{eqn:APhiA}
  (1/2) \frac{\lone{\rho^\beta \nabla u}}{\lone{\rho^\alpha(u-\ubar)}} 
  \leq \Phi(A) = \frac{\lone{\rho^\beta \nabla u}}{\lone{\rho^\alpha u}}
  \leq \frac{\lone{\rho^\beta \nabla u}}{\lone{\rho^\alpha(u-\ubar)}}.
\end{equation}
In the calculations below we omit the limiting argument with smooth
approximations of $u$ in equation \eqnref{:Aapprox} which justify
formula involving $\nabla u$. In particular, only the $\Lone$ norm
of $\rho^\beta |\nabla u|$ has meaning; the $\Ltwo$ norm is undefined.

Next, let $u_\theta$ be the mollification of (an extension of) $u$
given by equation \eqnref{:Autheta}. Then $u_\theta(x)$ is a local
average average of $u$ so $u_\theta(x) \geq 0$, $\linf{u_\theta} \leq
1$ and $\linf{u-u_\theta} \leq 1$.  Letting $L$ denote the Lipschitz
constant of $\rho$, the parameter $\theta$ will to be chosen 
less than $1/(2L)$ so that that Lemma \ref{lem:lonetheta} is applicable
with constant $c = 1/2$.

The remainder of the proof constructs an upper bound on the numerator
$\int_\Omega \rho^\gamma |\nabla u_\theta|^2$ of the Raleigh quotient
for $u_\theta - \ubar_\theta$ by $\lone{\rho^\beta \nabla u}$ and to
lower bound the denominator $\int_\Omega \rho^\alpha (u_\theta -
\ubar_\theta)^2$ by $\lone{\rho^\alpha (u-\ubar)}$. The conclusion
of the theorem then follows from equation \eqnref{:APhiA}.

\begin{itemize}
\item {\em Upper Bounding the Numerator:} To bound the $L^2$ norm in
  the numerator of the Raleigh quotient by the $L^1$ norm in the
  numerator of the expression for $\Phi(A)$ it is necessary to obtain
  uniform bound on $\nabla u_\theta$. To do this take the gradient of
  the second representation of $u_\theta$ in equation \eqnref{:Autheta}
  to get
  $$
  \nabla u_\theta(x) 
  = \int u(z) \left\{
    \frac{-d}{\theta \rho} \phi_{\delta}(x-z) \nabla \rho
    + \frac{1}{(\theta \rho)^{d+1}} 
    \left( I 
      + \nabla \rho \otimes \frac{x-z}{\theta \rho} \right)
    \nabla \phi \left(\frac{x-z}{\theta \rho} \right) \right\} \, dz.
  $$
  Multiplying by $\rho$ and noting that $|u(x)| \leq 1$ gives the bound
  $$
  |\rho(x) \nabla u_\theta(x)|
  \leq \frac{C(\phi)}{\theta} (1+L) \linf{u},
  \quad \text{ where } \quad
  C(\phi) = \int_{\Re^d} |\nabla \phi(y)| \, dy,
  $$
  and $L = \linf{\nabla \rho}$ is the Lipschitz constant for $\rho$. 

  Next, taking the gradient of the first representation of $u_\theta$
  in equation \eqnref{:Autheta} and using the chain rule shows
  $$
  \nabla u_\theta(x) 
  = \int_{\Re^d} 
  (I - \theta \nabla \rho \otimes y) \nabla u(x-\theta \rho u) \phi(y) \, dy,
  $$
  so
  $$
  \rho^\beta(x) \nabla u_\theta(x) 
  = \int_{\Re^d} 
  (I - \theta \nabla \rho \otimes y)
  \frac{\rho^\beta(x)}{\rho^\beta(x-\theta \rho y)}
  \rho^\beta(x-\theta \rho y) \nabla u(x-\theta \rho y) \phi(y) \, dy.
  $$
  The ratio in the integrand is bounded using the Lipschitz assumption
  on $\rho$ (and $|y| \leq 1$),
  \begin{equation} \label{eqn:ArhoRatio}
    \frac{\rho(x)}{\rho(x-\theta \rho y)}
    \leq \frac{\rho(x)}{\rho(x) - L \theta \rho(x)}
    = \frac{1}{1 - L \theta} \leq 2,
    \qquad \text{ when } \theta < 1 / (2L).
  \end{equation}
  Similarly, the $\ell^2$ matrix norm of $I - \theta \nabla \rho \otimes
  y$ is bounded by $3/2$, and and application of Lemma
  \ref{lem:lonetheta} then shows
  $$
  \lone{\rho^\beta \nabla u_\theta} 
  \leq C(\beta) \lone{\rho^\beta \nabla u},
  \qquad \text{ when } \theta < 1 / (2L).
  $$
  Combining the two estimates gives an upper bound for the Raleigh
  quotient 
  $$
  \int_\Omega \rho^\beta |\nabla u_\theta|^2
  = \int_\Omega \rho^{\gamma-\beta-1} \,
  \rho |\nabla u_\theta| \, \rho^\beta |\nabla u_\theta|
  \leq C(\phi,\beta) \linf{\rho^{\gamma-\beta-1}} \frac{1+L}{\theta} 
  \linf{u} \lone{\rho^\beta \nabla u},
  $$
  under the assumption $\theta \leq 1/(2L)$. Since $\linf{u} \leq 1$
  it can be dropped from the right hand side.

\item {\em Lower Bound on the Denominator:} Let $\ubar$ and
  $\ubar_\theta$ be the $\rho^\alpha$--weighted averages of $u$ and
  $u_\theta$ and let $\ltwoa{.}$ denote the $L^2$ space with this
  weight. Using the property that subtracting the average from a
  function reduces the $L^2$ norm it follows that
  $$
  \ltwoa{u_\theta - \ubar_\theta}
  \geq \ltwoa{u-\ubar} - \ltwoa{u_\theta - u - (\ubar_\theta-0)}
  \geq \ltwoa{u-\ubar} - \ltwoa{u_\theta - u}.
  $$
  If $a \geq b-c$ then $a^2 \geq b^2/2 - c^2$, so a lower bound for
  the denominator of the Raleigh quotient
  \begin{equation} \label{eqn:Autmu}
    \ltwoa{u_\theta - \ubar_\theta}^2
    \geq (1/2) \ltwoa{u}^2 - \ltwoa{u_\theta - u}^2
    \geq (1/4) \lonea{u} - \lonea{u_\theta - u},
  \end{equation}
  where the identity $\ltwoa{u}^2 = \lonea{u}/2$ and bound
  $\linf{u_\theta - u} \leq 1$ were used in the last step.

  To estimate the difference $\lonea{u_\theta - u}$ use the fundamental
  theorem of calculus to write
  \begin{eqnarray*}
    u_\theta(x) - u(x)
    &=& \int (u(x - \theta \rho y) - u(x)) \phi(y) \, dy \\
    &=& \int \! \int_0^1
    -\theta \rho \nabla u(x - t \theta \rho y).y \phi(y) \, dy \, dt \\
    &=& \int \! \int_0^1
    \frac{-\theta \rho(x)}{\rho^\beta(x-t\theta \rho y)} 
    \rho^\beta(x - t\delta y) \nabla u(x - t \delta y).y \phi(y) \, dy \, dt,
  \end{eqnarray*}
  so that 
  $$
  \rho^{\alpha}(x) ( u_\delta(x) - u(x) )
  = \int \! \int_0^1
  \frac{-\theta \rho^{\alpha+1}(x)}{\rho^\beta(x-t\delta y)} 
  \rho^\beta(x - t\delta y) 
  \nabla u(x - t \theta \rho y).y \phi(y) \, dy \, dt.
  $$
  Equation \eqnref{:ArhoRatio} bounds the ratio $\rho(x)/ \rho(x - t\delta y)$
  and an application of Lemma \ref{lem:lonetheta} then shows
  $$
  \lonea{u_\theta - u}
  \leq C(\beta) \linf{\rho^{\alpha+1-\beta}} \theta \lone{\rho^\beta \nabla u}
  \qquad \text{ when } \theta < 1 / (2L).
  $$
  Using this estimate in \eqnref{:Autmu} gives a lower bound on the
  denominator of the Raleigh quotient,
  $$
  \ltwoa{u_\theta - \ubar_\theta}^2
  \geq (1/4) \lonea{u} 
  - C(\beta) \theta \linf{\rho^{\alpha+1-\beta}} \lone{\rho^\beta \nabla u},
  \qquad \text{ when } \theta < 1 / (2L).
  $$
  {\em Bounding the Raleigh Quotient:} Combining the two steps above
  provides an upper bound for the Raleigh quotient of $u_\theta - \ubar_\theta$,
  \begin{eqnarray*}
    \lambda_2 
    &\leq& \frac{\int_\Omega \rho^\gamma |\nabla u_\theta|^2}
    {\int_\Omega \rho^\alpha (u_\theta - \ubar_\theta)^2} \\
    &\leq&  \frac{C(\phi,\beta)}{\theta}
    \frac{\linf{\rho^{\gamma-\beta-1}} (1+L) \lone{\rho^\beta \nabla u}}
    {\lonea{u} 
      - C(\beta) \theta \linf{\rho^{\alpha+1-\beta}} \lone{\rho^\beta \nabla u}} \\
    &\leq&  \frac{C(\phi,\beta)}{\theta}
    \frac{\linf{\rho^{\gamma-\beta-1}} (1+L)}
    {1 - C(\beta) \theta \linf{\rho^{\alpha+1-\beta}} \Phi(A)} \Phi(A).
  \end{eqnarray*}
  Selecting $\theta = (1/2)
  \min\left(1/(C(\beta)\linf{\rho^{\alpha+1-\beta}} \Phi(A)), 1/L
  \right)$ shows
  \[
  \lambda_2 \leq 2 C(\phi,\beta)\linf{\rho^{\gamma-\beta-1}}(1+L) 
  \max\left(L \Phi(A), C(\beta)\linf{\rho^{\alpha+1-\beta}} \Phi(A)^2 \vph\right).
  \qedhere
  \]
\end{itemize}
\end{proof}
